\documentclass[11pt]{beamer}
\usetheme{CambridgeUS}                     %Bergen, CambridgeUS
\usepackage[utf8]{inputenc}
\usepackage{amsmath}
\usepackage{amsfonts}
\usepackage{amssymb}
%\usepackage{helvet}
%\author{}
%\title{}
%\setbeamercovered{transparent} 
%\setbeamertemplate{navigation symbols}{} 
%\logo{} 
%\institute{} 
%\date{} 
%\subject{} 
%\begin{frame}
%\titlepage
%\end{frame}

%\begin{frame}
%\tableofcontents
%\end{frame}
\setbeamercolor{title}{fg=blue, bg=magenta}

\usecolortheme[RGB={128,128,255}]{structure} 
\setbeamercolor*{block title example}{fg=red!50!black,bg= green!80!black!50}
\setbeamercolor*{block body example}{fg=green!20!black, bg= green!15}
\setbeamercolor*{block body alerted}{fg= orange!50!black, bg= orange!15}
\setbeamercolor*{block title alerted}{fg=yellow!50, bg= orange!50!red}
\usepackage[skins]{tcolorbox}


\begin {document}
\begin{frame}{Sandwich box a la Beamer}

\begin{columns}[c]
\begin{column}{.3\linewidth} %décale la colonne de gauche
\begin{block}{Block}Text\end{block} %Bloc de texte classique qui suit le thème
\begin{exampleblock}{Example  block}Text\end{exampleblock} % Bloc d´ exemple classique
\begin{alertblock}{Alert block}Text\end{alertblock} % Bloc d´ alerte classique
\end{column}
\begin{column}{.4\linewidth}

% Without beamer skin: 

\begin{tcolorbox}[
left=1mm,right=1mm,top=1mm,bottom=1mm,middle=1mm,
skin=bicolor,
arc=5pt, %arrrondit les bords
bottomrule=0pt,
leftrule=00pt, % Met de la boite à gauche 
rightrule=0pt,
toprule=0pt,
colback=block body alerted.bg, %Colorie la deuxième case
colbacklower=block body example.bg, %Colorie la dernière case
collower=block title.bg,
colframe=block title.bg!1!black, %structure, %blue!75!black,
frame style={left color=block title.bg,
right color=block title example.bg!100!black}, %% framestyle right and left pour avoir le dégradé
% better shadow than beamer skin
fuzzy shadow={1mm}{-1mm}{-.25mm}{.5pt}{structure!20!black}% blue!30!black!80} % On personnalise les ombres 
,title=Testing Sans utiliser -beamer ]
A bicolor box
\tcblower
a la \dotfill \alert{no} Beamer mode 
\end{tcolorbox}

\bigskip

% With beamer skin: 


\begin{tcolorbox}[
left=1mm,right=1mm,top=1mm,bottom=1mm,middle=1mm, % place le titre
beamer, % Warning: must be before of bicolor skin
skin=bicolor,
colback=block body.bg, %mystère
colbacklower=block body example.bg, % Colorie l'intérieur de la colonne la plus basse
collower=block body alerted.fg, %Colorie le text de la colonne la plus basse
colupper=block body example.fg, %Colorie le text de la colonne en dessous le titre
%colframe=block title.bg,
coltitle=block title alerted.fg, %Couleur du titre
frame style={draw=none,fill=none, left color=block title.bg!2!blue, right color=block title.bg!20!black},
interior style={left color=block body alerted.bg, right color=block title alerted.bg}, %colorie l'intérieur de la deuxième colonne
,title=Pas mal celle la ]
A bicolor box
\tcblower %séparateur de colonne
a la Beamer mode

\end{tcolorbox}

\end{column}
\end{columns}
\end{frame}
\end{document}
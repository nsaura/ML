\documentclass[11pt, xcolor=svgnames]{beamer}
\usetheme[faculty=fi]{fibeamer}
\usetheme{Madrid}
%\usetheme{Warsaw}                %Bergen, CambridgeUS, warsaw
\usepackage[utf8]{inputenc}
\usepackage[skins]{tcolorbox}
\usepackage{tikz}      % Diagrams
\usetikzlibrary{calc, shapes, backgrounds}
%\usepackage{helvet}
%\usepackage{amsmath}
%\usepackage{amsfonts}
%\usepackage{amssymb}
\usepackage{ragged2e}

\setbeamercovered{transparent} 
\setbeamertemplate{navigation symbols}{} 

\setbeamercolor{title}{fg=black, bg=DarkRed!50!yellow!50!}
\usecolortheme[RGB={128,128,255}]{structure} 
\setbeamercolor*{block title example}{fg=red!50!black,bg= green!80!black!128}
\setbeamercolor*{block body example}{fg=green!20!black, bg= green!15}
\setbeamercolor*{block body alerted}{fg= orange!50!black, bg= orange!15}
\setbeamercolor*{block title alerted}{fg=yellow!50, bg= orange!50!red}
\frenchspacing

\AtBeginSection[]{% Print an outline at the beginning of sections
    \begin{frame}<beamer>
      \frametitle{Outline for Section \thesection}
      \tableofcontents[currentsection]
    \end{frame}}

\graphicspath{{../pic/},{./}}


%%% Couleurs %%%
\xdefinecolor{brick}{named}{DarkRed}
\xdefinecolor{navy}{named}{Navy}
\xdefinecolor{midblue}{named}{MidnightBlue}
\xdefinecolor{dsb}{named}{DarkSlateGray}

%%% 	Raccourcis 	%%%
\newcommand{\keps}{$k-\varepsilon$}
\newcommand\bk{\color{black}}
\newcommand\dred{\color{brick}}
\newcommand\navy{\color{navy}}
\newcommand\midblue{\color{midblue}}
\newcommand\dsb{\color{dsb}}

\setbeamertemplate{enumerate items}[default]
\addtobeamertemplate{frametitle}{\vskip-1ex}{}

\begin{document}

\title{Oral defence} %% that will be typeset on the
\subtitle{\color{brick} On the road to phd} %% title page.
\author{Nathaniel Saura}
%\logo{} 
\institute{École Polytechnique Universitaire de Lyon (EPUL)} 
%\subject{} 

\begin{darkframes}

\frame{\maketitle}

\begin{frame}{Au sommaire de la présentation}
\tableofcontents
\end{frame}

\section{intro}
\begin{frame}
\begin{columns}
\begin{column}{.5\textwidth}
\begin{enumerate}
  \item<1,3> Ploum!
  \item<3> Paf
  \item Plum...
  \end{enumerate}
\end{column}
\begin{column}{.5\textwidth}
\RaggedRight
\vspace{-3cm}
%\begin{figure}

%\includegraphics[scale=0.5]{Fig-4-Example-of-lattice-Boltzmann-grid-D3Q19.png}

%\end{figure}
\end{column}
\end{columns} 
\vspace{1cm}
texte normal pasogji dpafgjda pfogkjdpfo
\end{frame}

\begin{frame}
\begin{columns}
\begin{column}{.5\textwidth}
\begin{block}{bloc 1}
sDO fjdpsofj psdfjpos j
\begin{itemize}
  \item<1,3> Ploum!
  \item<3> Paf
  \item Plum...
  \end{itemize}\end{block}
\end{column}

\begin{column}{.5\textwidth}
%\includegraphics[width=\textwidth]{epsilon.png}
\end{column}
\end{columns}
texte normal pasogji dpafgjda pfogkjdpfo
\end{frame}


\section{intro2}

\begin{frame}{Sandwich box a la Beamer}
\begin{columns}[c]
\begin{column}{.3\linewidth} %décale la colonne de gauche
\begin{block}{Block}Text\end{block} %Bloc de texte classique qui suit le thème
\begin{exampleblock}{Example  block}Text\end{exampleblock} % Bloc d´ exemple classique
\begin{alertblock}{Alert block}Text\end{alertblock} % Bloc d´ alerte classique
\end{column}
\begin{column}{.4\linewidth}
% Without beamer skin: 
\begin{tcolorbox}[
left=1mm,right=1mm,top=1mm,bottom=1mm,middle=1mm,
skin=bicolor,
arc=5pt, %arrrondit les bords
bottomrule=0pt,
leftrule=00pt, % Met de la boite à gauche 
rightrule=0pt,
toprule=0pt,
colback=block body alerted.bg, %Colorie la deuxième case
colbacklower=block body example.bg, %Colorie la dernière case
collower=block title.bg,
colframe=block title.bg!1!black, %structure, %blue!75!black,
frame style={left color=block title.bg,
right color=block title example.bg!100!black}, %% framestyle right and left pour avoir le dégradé
% better shadow than beamer skin
fuzzy shadow={1mm}{-1mm}{-.25mm}{.5pt}{structure!20!black}% blue!30!black!80} % On personnalise les ombres 
,title=Testing Sans utiliser -beamer ]
A bicolor box
\tcblower
a la \dotfill \alert{no} Beamer mode 
\end{tcolorbox}
\bigskip
% With beamer skin: 
\begin{tcolorbox}[
left=1mm,right=1mm,top=1mm,bottom=1mm,middle=1mm, % place le titre
beamer, % Warning: must be before of bicolor skin
skin=bicolor,
colback=block body.bg, %mystère
colbacklower=block body example.bg, % Colorie l'intérieur de la colonne la plus basse
collower=block body alerted.fg, %Colorie le text de la colonne la plus basse
colupper=block body example.fg, %Colorie le text de la colonne en dessous le titre
%colframe=block title.bg,
coltitle=block title alerted.fg, %Couleur du titre
frame style={draw=none,fill=none, left color=block title.bg!2!blue, right color=block title.bg!20!black},
interior style={left color=block body alerted.bg, right color=block title alerted.bg}, %colorie l'intérieur de la deuxième colonne
,title=Pas mal celle la ]
A bicolor box
\tcblower %séparateur de colonne
a la Beamer mode
\end{tcolorbox}
\end{column}
\end{columns}
\end{frame}

\section{part1}

\end{darkframes}


\end{document}

\documentclass[12pt, xcolor=svgnames]{beamer}
\usepackage[utf8x]{inputenc}
\usepackage[french]{babel}
\usepackage{amsmath, amsfonts, epsfig, xspace}
\usepackage{algorithm,algorithmic}
\usepackage{pstricks,pst-node}
\usepackage{multimedia}
\usepackage{sistyle} 
\usepackage[normal,tight,center]{subfigure}
\setlength{\subfigcapskip}{-.5em}
\usetheme{Warsaw}
\usecolortheme{beaver}
\usepackage{beamerthemesplit} 

\xdefinecolor{brick}{named}{DarkRed}
\xdefinecolor{navy}{named}{Navy}
\xdefinecolor{midblue}{named}{MidnightBlue}
\xdefinecolor{dsb}{named}{DarkSlateGray}

\newcommand{\keps}{$k-\varepsilon$}
\newcommand\bk{\color{black}}
\newcommand\brick{\color{brick}}
\newcommand\navy{\color{navy}}
\newcommand\midblue{\color{midblue}}
\newcommand\dsb{\color{dsb}}

%%%%%%%% Math
\newcommand{\vx}{$\vec{x}$}
\newcommand{\vp}{$\vec{p}$}
\newcommand{\DT}{\frac{D}{Dt}}
\newcommand\numberthis{\addtocounter{equation}{1}\tag{\theequation}}
\newcommand{\cia}{c_{i\alpha}}
\newcommand{\cib}{c_{i\beta}}
\newcommand{\cig}{c_{i\gamma}}
\newcommand{\fiun}{f_i^{(1)}}
\newcommand{\fieq}{f_i^{eq}}
\newcommand{\feq}{$f^{\text{eq}}$}
\newcommand{\delab}{\delta_{\alpha \beta}}
\newcommand{\delbc}{\delta_{\beta \gamma}}
\newcommand{\delac}{\delta_{\alpha \gamma}}

%%%%%%%% Cigles
\newcommand{\mdf}{mécanique des fluides}
\newcommand{\ckng}{Chapman-Enskog}
\newcommand{\BeGK}{Bhatnagar, Gross et Krook}
\newcommand{\BGK}{Bhatnagar--Gross--Krook}
\newcommand{\cad}{c'est-à-dire}
\newcommand{\cla}{couche limite atmosphérique}
\newcommand{\qttmvt}{quantité de mouvement}

\usepackage{pifont}
\newcommand{\bwarrow}{\item[\color{DarkRed} \ding{227}]}
\newcommand{\warrow}{\item[\color{blue!50!black!70} \tiny{\ding{109}}]}
\newcommand{\sarrow}{\item[\color{blue!50!black!70!orange!60} \tiny{\ding{55}}]}

%%%%%%%% Autres
\newcommand{\kn}{$K_n$}

\graphicspath{{/home/saura/Documents/LB_2017/pic/},{/home/nsaura/Bureau/Notes/pic/},
{/home/saura/Documents/LB_2017/exploitable/},{/home/nsaura/Bureau/Notes/exploitable/}}


\newcommand{\backupbegin}{
   \newcounter{framenumberappendix}
   \setcounter{framenumberappendix}{\value{framenumber}}
}
\newcommand{\backupend}{
   \addtocounter{framenumberappendix}{-\value{framenumber}}
   \addtocounter{framenumber}{\value{framenumberappendix}} 
}

%%%%%%%% Commandes
\newcommand{\fracpart}[1]{
	% #1: parameter with respect of which you want to derive 
	\frac{\partial}{\partial #1}
}
\newcommand{\fracD}[2]{
	% #1: parameter with respect of which you want to derive 
	\frac{D #1}{D #2}
}
\newcommand{\bepar}[1]{
	\left( #1 \right)  
}

\newcommand{\becro}[1]{
	\left[ #1 \right]  
}

\newcommand{\moyvar}[1]{
	\overline{#1}
}

\newcommand{\moyu}[1]{
	\overline{U}_{#1}  
}

\usepackage{setspace}

    \expandafter\def\expandafter\insertshorttitle\expandafter{%
       \insertshorttitle\hfill%
       \insertframenumber\,/\,\inserttotalframenumber}

\usepackage{geometry}
\geometry{hmargin=0.5cm, vmargin=0cm}

\author[Saura Nathaniel]{Saura Nathaniel \textit{sous la supervision de : \\}  
	\small{Mr.~Louis \textsc{\small Gostiaux} \\ 
	Mr.~Emmanuel \textsc{\small Lévêque}\\
    Mr.~Bastien \textsc{\small Di Pierro}\\
    Mr.~Pietro \textsc{\small Salizzoni} \\
    Mr.~Lionel \textsc{\small Soulhac}}}

\title[Soutenance 5A Polytech - 7 Septembre 2017]{\brick Simulations numériques de dispersion de gaz lourds dans une couche limite turbulente}
\date{}


\institute{Soutenance de stage de 5$^{\text{ème}}$ année de Polytech Lyon filière mécanique.\\
\begin{center}

\begin{minipage}[!ht]{0.9\textwidth}
  \begin{flushleft}
    \includegraphics[scale=0.15]{polytech.jpg} \hfill
    \includegraphics[scale=0.15]{centrale.png} \hfill
    \includegraphics[scale=0.25]{centrale_innov.png} \hfill
    \vspace{2cm}\includegraphics[scale=0.2]{univ_lyon.png} \\[0.5cm]
    \begin{spacing}{1.5}
    \end{spacing}
  \end{flushleft}
\end{minipage}
\end{center}
}
\begin{document}

\maketitle

% \section{Introduction}  % add these to see outline in slides

%\begin{frame}
%  \frametitle{Introduction }
%   \begin{itemize}
%   \bwarrow Problème de dispersion de scalaire actif en gravité dans l'air
%  \begin{itemize}
%  \warrow Masse volumique du scalaire vérifiant $\rho_{scalaire} > \rho_{air}$ \\
%  \warrow Salaire injecté dans le domaine \\
%   \end{itemize}
%  \vspace{1cm}
%  \bwarrow Couche limite turbulente avec le modèle \keps
%	  \begin{itemize}
%  	\warrow Profils de vitesse $U_x$, d'énergie cinétique turbulente $k$ et de sa dissipation $\varepsilon$ initialement issus de la thèse de Hervé Gamel (2015)\\
%  	\warrow Couche limite turbulente atmosphérique \\
%  	\warrow Écoulement pariétal
%  	\end{itemize}
%  \end{itemize}
%
%\end{frame}
%
%% \section{Main Body} % add these to see outline in slides
%
%\begin{frame}
%  \frametitle{Introduction}
%  \begin{itemize}
%	\bwarrow Méthode de Boltzmann sur réseau (LBM)
%  	\begin{itemize}
%		\warrow Code optimisé, parallélisé, orienté objet  		
%  		\warrow Passage D$_3$Q$_{19}$ à D$_3$Q$_{27}$ \\
%  		\warrow Conditions aux limites et couche limite\\
%  	\end{itemize}
%  	\vspace{1cm}
%	\bwarrow Objectifs évoluant au cours du stage  	
%  	\begin{itemize}
%  		\warrow Utilisation des profils théoriques \\
%  		\warrow Domaine sans obstacle \\
%  		\warrow Travaux sur la LBM	
%  	\end{itemize} 
%  \end{itemize}
%\end{frame}

\begin{frame}
\frametitle{Motivations du stage}
\begin{center}
\includegraphics[scale=0.3]{gas_disp.png}
\end{center}

\begin{itemize}
	\bwarrow L'éjection de nuages lourds dans la couche limite atmosphérique est au cœur des problématiques environnementales et sécuritaire
		\begin{itemize}
			\warrow Fuites de gaz industriels
			\warrow Explosions
		\end{itemize}
	\bwarrow L'étude de leur dispersion est fondamentale pour en améliorer la prise en charge \\
\end{itemize}
\end{frame}

\begin{frame}
\frametitle{Outils numériques utilisés}
\begin{itemize}
	\bwarrow Le modèle \keps 
		\begin{itemize}
			\warrow Permet l'obtention rapide de résultats
			\warrow Outil utilisé par les industriels
			\warrow Précision discutable \\[1cm]
		\end{itemize}
	\bwarrow La méthode des réseaux de Boltzmann
		\begin{itemize}
			\warrow Permet l'obtention rapide de résultats
			\warrow Outil en pleine expansion en recherche et en industrie
			\warrow Précision importante à faible nombre de mach 
			\end{itemize}
\end{itemize}
\end{frame}

\begin{frame}
	\frametitle{Plan de la présentation}
	\begin{enumerate}[\color{DarkRed} I.]
		\item Étude \keps $ $ de l'évolution du nuage
		\begin{enumerate}[\dsb 1.]
			\item Simulation et les profils utilisés\\
			\item Richardson, hauteur et vitesse moyenne de transport du nuage\\
			\item Résultats et discussions
		\end{enumerate}
		\vspace{1cm}
		\item La méthode de Boltzmann sur réseaux
		\begin{enumerate}[\dsb 1.]
			\item Hypothèses et éléments de théorie \\
			\item Implémentation : dynamique à deux temps\\
			\item Changement de discrétisation et conditions aux limites  			
		\end{enumerate}
	\end{enumerate}
\end{frame}

% \section{Conclusion} % add these to see outline in slides


\begin{frame}
\frametitle{Écoulement et domaine considérés }
\framesubtitle{La couche limite turbulente}
\begin{itemize}
	\bwarrow Écoulement considéré : couche limite turbulente 
\end{itemize}
\vspace{-0.6cm}
\begin{figure}[!ht]
\centering
\includegraphics[scale=0.4]{BL_est.png}
\end{figure}
\begin{itemize}
	\bwarrow{Constantes caractéristiques du problème issus des expériences menées par Hervé Gamel dans la Soufflerie Atmosphérique i11, utilisées pour les simulations Fluent }	
	\begin{itemize}
		\warrow Vitesse en dehors de la couche limite : $U_\infty = 6\pnt 33$ m/s 
		\warrow Hauteur de rugosité  $z_0 = 1\pnt38$ mm : l'action du sol
		\warrow	Vitesse de frottement $u_* = 0\pnt34$ m/s
		\warrow Hauteur de transition  $\delta = 0\pnt 55$  m.
	\end{itemize}
\end{itemize}
\end{frame}
    
\begin{frame}
  \frametitle{Écoulement et domaine considérés}
  \framesubtitle{Domaine d'étude de la simulation} 

\begin{columns}
\begin{column}{0.55\textwidth}
\centering
\includegraphics[scale=0.26]{spire_im.png}
\end{column}
\begin{column}{0.4\textwidth}
\begin{itemize}
	\warrow \footnotesize{Injection de polluant à $0\pnt 7$ m de l'entrée}
	\warrow \footnotesize{Taille d'injecteur variable (de 1 à 20 cm)}
\end{itemize}
\end{column}
\end{columns}

  
\begin{columns}

\begin{column}{0.55\textwidth}
\vspace{-0.5cm}\begin{flushleft}
	\includegraphics[scale=0.25]{cote.png} 
\end{flushleft}
\end{column}

\begin{column}{0.4\textwidth}
\begin{itemize}	
	\warrow \footnotesize{Injection de profils $k$, $U_x$ et $\varepsilon$ en entrée et pression imposée en sortie}
	\warrow \footnotesize{"Symmetry" en haut et non glissement à la paroi}
	\end{itemize}
\end{column}

\end{columns}
	
\end{frame}

\begin{frame}
\frametitle{Modèle \keps $ $}
\framesubtitle{RANS : Reynolds Averaged Navier-Stokes}
\begin{itemize}
\bwarrow Décomposition de Reynolds : \\ \begin{center}$ \displaystyle \underbrace{X}_{\text{Champ total}} = \underbrace{\overline{X}}_{\text{Champ moyen (temporel)}} + \underbrace{x}_{\text{champ fluctuant : écart à la moyenne}} $ \end{center}
\end{itemize} 
\begin{itemize}
\bwarrow Équation du champs moyen de la vitesse (RANS) stationnaire\\
\begin{center}$\displaystyle \overline{U}_j \dfrac{\partial \overline{U}_i}{\partial x_j} = -\dfrac{\partial \overline{P}}{\partial x_i} + \nu \dfrac{\partial^2 \overline{U}_i}{\partial x_j \partial x_j } - \underbrace{\dfrac{\partial \overline{u_iu_j}}{\partial x_j}}_{\text{Champ fluctuant}}$  \end{center} 

\bwarrow Énergie cinétique turbulente $k$ et sa dissipation $\varepsilon$\\ \begin{center} $k = \dfrac{1}{2} \displaystyle \bepar{\overline{u^2} + \overline{v^2} + \overline{w^2}}$ \\ $\displaystyle \varepsilon \simeq \nu \overline{\dfrac{\partial u_i}{\partial x_j}\dfrac{\partial u_i}{\partial x_j}} $ \end{center}

\end{itemize}

\end{frame}
%\begin{frame}
%\frametitle{Modèle \keps $ $ }
%\framesubtitle{Équation de $k$}
%\begin{itemize}
%	\warrow Modélisation de la production d'ECT $\mathcal{P}$ possible \\ \begin{center}
%$\displaystyle \mathcal{P} = \nu_t \frac{\partial \overline{U}_i}{\partial x_j}\bepar{\frac{\partial \overline{U}_i}{\partial x_j}+ \frac{\partial \overline{U}_j}{\partial x_i}} = \frac{\nu_t}{2} \bepar{\frac{\partial \overline{U}_i}{\partial x_j} + \frac{\partial \overline{U}_j}{\partial x_i}}^2 > 0 $ \\ 
%\end{center}
%
%	\warrow Modélisation des transports d'ECT par fluctuations de vitesse $T_u$ et par la pression $T_p$ \\
%\begin{center}
%$\displaystyle T_u + T_p = -\dfrac{\nu_t}{\boldsymbol{\sigma_k}}\dfrac{\partial k}{\partial x_j}$ 
%\end{center}
%\begin{block}{\color{gray!50!blue!10} L'équation sur $k$ du modèle \keps}
%\begin{center}$\displaystyle \moyu{j} \frac{\partial k}{\partial x_j} = \dfrac{\partial k}{\partial x_j} \becro{\bepar{\nu + \dfrac{\nu_t}{\boldsymbol{\sigma_k}}} \dfrac{\partial k}{\partial x_j}} + ~ \mathcal{P}-  ~\varepsilon  $ \end{center}
%
%\end{block}
%
%\end{itemize}
%\end{frame}

\begin{frame}
\frametitle{Modèle \keps $ $ : modèle à deux équations}
\begin{block}{\color{gray!50!blue!10} L'équation de $k$ et de $\varepsilon$ dans le modèle \keps}
\begin{itemize}
\sarrow $k$ :
\end{itemize}
\begin{equation*} 
\moyu{j} \frac{\partial k}{\partial x_j} = \dfrac{\partial k}{\partial x_j} \becro{\bepar{\nu + \dfrac{\nu_t}{\boldsymbol{\sigma_k}}} \dfrac{\partial k}{\partial x_j}} + ~ \mathcal{P}-  ~\varepsilon
 \end{equation*}

\begin{itemize}
\sarrow $\varepsilon$ :
\end{itemize}
\begin{equation*}
\moyu{j} \dfrac{\partial \varepsilon}{\partial x_j} = \dfrac{\partial}{\partial x_j} \becro{\bepar{\nu + \dfrac{\nu_t}{\boldsymbol{\sigma_\varepsilon}}}\dfrac{\partial \varepsilon}{\partial x_j}} + \dfrac{\varepsilon}{k} \bepar{\dfrac{\boldsymbol{C_{\varepsilon_1}}\mathcal{P}}{\rho} - \boldsymbol{C_{\varepsilon_2}\varepsilon}}
\end{equation*}
\end{block}
\end{frame}

\begin{frame}
\begin{itemize}
	\bwarrow Le modèle \keps $ $ ne calcule que les champs moyens et simule les champs fluctuant à partir de \textbf{constantes} (Launder 1974) ... \\
	
			\begin{table}[!hbt]
		% Center the table
		\begin{center}
		% Table itself: here we have two columns which are centered and have lines to the left, right and in the middle: |c|c|
		\begin{tabular}{|c || c || c || c || c|}
		$C_\mu$ & $C_{\varepsilon_1}$ & $C_{\varepsilon_2}$ & $~\sigma_k~$ & $\sigma_\varepsilon$ \\
		
		\hline
		
		$0 \pnt 09 $ & $1 \pnt 44$ & $1 \pnt 92$ & $ 1 \pnt 0$ & $1 \pnt 31$
		
		\end{tabular} 
		\caption{\footnotesize{Tableau présentant le jeu de constantes classique et couramment accepté pour le \keps.}	}	
		\end{center}
		\end{table}
		
			\bwarrow ... Et de l'hypothèse de Boussinesq (1877) : \\
		\begin{center}$\displaystyle - \overline{u_iu_j} = \nu_t \bepar{\frac{\partial \overline{U}_i}{\partial x_j} + \frac{\partial \overline{U}_j}{\partial x_i}} - \frac{2}{3} k \delta_{ij} $ \end{center}

	\bwarrow Équation d'évolution de la viscosité turbulente \\
\begin{center}
			$ \displaystyle \nu_t = \boldsymbol{C_\mu} \dfrac{k^2}{\varepsilon}$
\end{center}	
	\bwarrow 4 équations pour 4 inconnues 
\end{itemize}
\end{frame}

\begin{frame}
\frametitle{Profils utilisés}
\framesubtitle{Mise en place de la simulation}

\begin{itemize}
	\bwarrow Création du maillage 	 \\[5mm]
	\bwarrow Création des UDF (User Defined Function) en trois étapes
	\begin{itemize}
		\warrow Récupération des données de Gamel et établissement fits polynomiaux \\
			\begin{itemize}
				\sarrow Utilisation des points expérimentaux 
				\sarrow Prolongement : hypothèse laminaire 
			\end{itemize}
		\warrow Écriture de ces polynômes dans des codes C appelés UDF \\[5mm]
	\end{itemize}
	\bwarrow Compilation de ces UDF avant initialisation Fluent
\end{itemize}

\end{frame}
 	
\begin{frame}
\frametitle{Profils expérimentaux et leur approximation}
	\begin{figure}[!ht]
		\centering
		\includegraphics[scale=0.275]{vxx_fitt.png} \hfill
		\includegraphics[scale=0.25]{k_fitt.png}	   \hfill	
		\includegraphics[scale=0.25]{eps_fitt.png} 
	\end{figure}
	  \small{Tracés de $U_x$, de $k$ et de $\varepsilon$ en fonction de la hauteur adimensionnée par $\delta$}
\end{frame} 	
 	
\begin{frame}
\frametitle{Résultats des simulations avec les profils initiaux de Gamel}
	\begin{block}{Critère de validation des résultats}
		Similarité des profils de vitesse entrée/sortie après convergence Fluent.
	\end{block}	
	\pause
	\begin{itemize}
		\bwarrow Comparaisons entrée sortie après convergence : $U_x$
	\end{itemize}
\begin{columns}
\begin{column}{0.5\textwidth}
\vspace{-1.1cm}
\begin{figure}[!ht]
\centering
\includegraphics[scale=0.38]{out_inline_first.png}
\end{figure}
\end{column}
\begin{column}{0.5\textwidth}
\begin{itemize}
	\bwarrow \small{Le profil continue de s'établir} \\
	\bwarrow \small{Ajout d'une source de \qttmvt $ $ pour compenser les frottements au sol n'y change rien} \\ 
	\bwarrow \small{Incompatibilité du niveau de précision des données et du \keps.}
\end{itemize}

\end{column}
\end{columns}
\end{frame}

\begin{frame}
\frametitle{Utilisation des profils théoriques}
\begin{itemize}
	\bwarrow Conservation  des constantes du problèmes ($U_\infty$, $\delta$, $z_0$)
\end{itemize}

\begin{align*}
U_{theo}(y) &= \frac{u_*}{\kappa} \ln\bepar{\frac{y}{z_0}} \\
\varepsilon_{theo} (y) &= \frac{u_*^3}{\kappa y} \\
k_{theo}\ \ \ \ & = \frac{u_*^2}{\sqrt{C_\mu}} 
\end{align*}

\begin{itemize}
\bwarrow Comparaisons Entrées Sorties après convergence de Fluent (ajout de la source \qttmvt)
\end{itemize}

\end{frame}

\begin{frame}
\frametitle{Comparaisons avec les profils théoriques}
	\begin{block}{Critère de validation des résultats}
		Similarité des profils de vitesse entrée/sortie après convergence Fluent.
	\end{block}
\begin{columns}
\begin{column}{0.6\textwidth}
\vspace{-1.1cm}
\begin{figure}[!ht]
\centering
\includegraphics[scale=0.27]{vitesse_comp_theo.png}
\end{figure}
\end{column}
\pause
\begin{column}{0.4\textwidth}
\begin{itemize}
	\bwarrow \small{Le profil est établi} \\ 
	\bwarrow \small{Ajout indispensable d'une source de \qttmvt} \\ 
	\bwarrow \small{Profils gardés pour la suite}
\end{itemize}

\end{column}
\end{columns}
\end{frame}

\begin{frame}
\frametitle{Définition du nombre de Richardson}
\framesubtitle{Un nombre adimensionné, beaucoup d'interprétations}
\begin{block}{Définition formelle du Richardson}
\begin{equation*}
\text{Ri} = \frac{\text{Énergie potentielle}}{\text{Énergie cinétique}} \equiv \frac{\text{Flottabilité}}{\text{Inertie}}
\end{equation*}

\end{block}
\pause
\begin{itemize}
	\bwarrow Nombre adimensionné $\Rightarrow$ nécessité de spécifier les échelles
		\begin{itemize}
		\warrow Énergie potentielle du \textit{\textbf{nuage}} 
			\begin{itemize}
				\sarrow Définition du poids du nuage $g_s' = g \bepar{\rho_s - \rho_a}/\rho_a$
				\sarrow Définition de la hauteur du nuage $h_p$ (slide suivante)
			\end{itemize}
		\warrow Énergie cinétique \textbf{ambiante} caractérisée $u_*$
		\end{itemize}
\end{itemize}
 \begin{center}$\displaystyle \text{Ri} = \dfrac{g'_s~h_p}{u_*^2}$ \end{center}
 
\begin{itemize}
\bwarrow Approximation de Boussinesq : variations de masse volumique négligeables si non multipliées par $g$ et $\rho_{air} = \rho_{env}$
\end{itemize}


\end{frame}

\begin{frame}
\frametitle{Caractérisation de l'évolution du nuage}
\framesubtitle{$h_p$ et $U_p$}

\begin{block}{Définition de la hauteur du nuage $h_p$}
La hauteur du nuage est définie comme la hauteur maximale pour laquelle on mesure une différence significative de masse volumique : \\
\begin{equation*}
h_p = \int^\infty_0 \dfrac{\rho\bepar{x,y} - \rho_a}{\rho_s - \rho_a} ~dy
\end{equation*}
\end{block}

\begin{block}{Définition de la vitesse de transport moyen du nuage $U_p$}
La vitesse de transport moyen du nuage est définie comme la vitesse moyenne de l'écoulement pondérée par les différences de masse volumiques observables :
\begin{equation*}
 U_p(x) = \frac{ \int^{\infty}_0 U(x,y)\bepar{\rho(x,y) - \rho_a} dy}{\int^{\infty}_0 \bepar{\rho(x,y) - \rho_a} dy}
\end{equation*}
\end{block}

\end{frame}

\begin{frame}
\frametitle{Similitude dans les comportements}


\begin{center}
 \includegraphics[scale=0.3]{Ri_Hp_Up.png}
\end{center}

\end{frame}

\begin{frame}
\frametitle{Richardson Vs $h_p$}
\begin{figure}[!ht]
\centering
\includegraphics[scale=0.3]{diff_ustar.png}
\end{figure}

\end{frame}

\begin{frame}
\frametitle{Bilan des actions du Richardson}
\framesubtitle{Richardson : paramètre de Bulk Stability}
\begin{itemize}
	\bwarrow Sur la vitesse :
	\begin{itemize}
		\warrow Corrélation entre $U_p$ et Ri
		\warrow Les zones concentrées peu affectées par $U_p$. Zone supérieure du nuage happée par cette dernière.\\[3mm]
	\end{itemize}
	\pause
	\bwarrow Sur la hauteur du nuage
	\begin{itemize}
		\warrow Force d'Archimède ou agitation turbulente ;
		\warrow Deux facettes du Richardson \\[3mm]
	\end{itemize}
	\pause
	\bwarrow Ri : critère de stabilité à l'étirement/contraction (Bulk Stability)
		\begin{itemize}
		\warrow Pour de forts Richardson, la vitesse de transport est réduite et le nuage est compact, tassé sur lui même 
		\warrow Pour de faibles Richardson c'est l'inverse le nuage est vite emporté et sa dispersion s'accentue
		\end{itemize}
\end{itemize}
\end{frame}

\begin{frame}
\frametitle{Motivations des simulations Boltzmann sur Réseau LBM}
\begin{itemize}
	\bwarrow Simuler un cas de dispersion de nuage lourd 3D instationnaire \\[2mm]
	\bwarrow Massivement parallélisable \\[2cm]
	\bwarrow Détails de la méthode et des premiers travaux constituent la deuxième partie de cette présentation 
\end{itemize}
\end{frame}

\begin{frame}
\frametitle{La méthode de Boltzmann sur Réseau (LBM)}
\framesubtitle{Hypothèses}
\begin{block}{Origine de la LBM}
La LBM est issue de la théorie cinétique des gaz. On suppose que :
\begin{itemize}
\bwarrow Les molécules 
		\begin{itemize}
			\warrow sont petites, $D_{\text{molécules}} \ll l_\mu$ (\textbf{gaz dilué})
			\warrow sont de masse, de forme et de volume identiques
			\warrow se déplacent à une vitesse \textbf{proche} de \textbf{la vitesse du son} \\
		\end{itemize}
\bwarrow La dynamique est 
		\begin{itemize}
			\warrow \textbf{seulement} due aux \textbf{collisions} inter-particule
			\warrow régie par \textbf{l'équation de Boltzmann}
		\end{itemize}
\end{itemize}
\end{block}
\pause
\begin{itemize}
	\bwarrow $f^{(N)}\bepar{\vec{x}, \vec{p}, t}$ : fonction de distribution : probabilité de trouver $N$ particules au point $\bepar{\vec{x}, \vec{p}}$ à l'instant $t$.
	\bwarrow $N$ réplications de $f^{(1)} \equiv f$ pour décrire la dynamique d'un gaz
\end{itemize}
\end{frame}

%\begin{frame}
%\frametitle{L'équation de Boltzmann}
%\begin{block}{l'équation de Boltzmann (BE)}
%À partir d'un développement de Taylor sur l'évolution temporelle de $f$, et en considérant F une force extérieure, la BE s'écrit
%\vspace{-0.3cm} \begin{equation*}
%\frac{\partial f}{\partial t} + \vec{V}\cdot\nabla_{\vec{x}} \left( f \right) + \vec{F} \cdot \nabla_{\vec{p}} \left( f \right) = \Omega\bepar{f}  \text{\hspace{2mm} Avec \hspace{2mm}} \vec{V} = m\vec{p} 
%\end{equation*}
%\end{block}
%\begin{itemize}
%	\bwarrow Niveau de description entre NSE et l'équation de Hamilton : échelle mésoscopique 
%	\bwarrow Possibilité de retrouver l'hydrodynamique ?
%\end{itemize}
%
%\end{frame}

%\begin{frame}
%\frametitle{LBM pour résoudre l'hydrodynamique}


%\begin{itemize}
%	\bwarrow \BeGK $ $ vont simplifier cette expression ainsi que $\Omega\bepar{f}$
%\end{itemize}
%\end{frame}



\begin{frame}
\frametitle{la LBM}
\framesubtitle{Discrétisation}
\begin{columns}
	\begin{column}{0.5\textwidth}
		\begin{itemize}
			\bwarrow Discrétisation de l'espace des phases 
				\begin{itemize}
					\warrow Discrétisation du domaine en réseaux (lattices)
					\warrow Ensemble de Q vitesses possibles muni de poids $\{c_a,w_a\}$
				\end{itemize}
			\bwarrow Calcul des densités discrètes $f_a$ sur chaque site du réseau
			\bwarrow Discrétisation temporelle

		\end{itemize}			
	\end{column}
	
	\begin{column}{0.5\textwidth}
%		\begin{figure}
			\includegraphics[scale=0.3]{D3_Q19.png}

%		\end{figure}			
	\end{column}
\end{columns}
\begin{itemize}
	\bwarrow $\displaystyle \rho_{\text{réseau}} = \sum_{a=0}^{Q-1} f_a$  \hspace{0.2\textwidth} $\displaystyle \vec{u}_{\text{réseau}} = \frac{1}{\rho} \sum_{a=0}^{N_{chemins}}f_a \vec{c}_a $
\end{itemize}
\end{frame}

\begin{frame}
\frametitle{Principe de la LBM}
\framesubtitle{l'équation de Boltzmann discrétisée}
\begin{block}{L'équation de Boltzmann discrétisée (LBE)}
On résout l'équation dite de Boltzmann sur réseaux, sur l'espace discrétisé :
\begin{equation*}
f_a\bepar{\vec{x}_a + \vec{c}_a \Delta t, t + \Delta t} = f_a\bepar{\vec{x}_a, t} + \Delta t \Omega_a\bepar{\vec{x}_a,t}
\end{equation*}
\end{block}

\begin{itemize}
	\bwarrow Introduction de distribution $f$ à l'équilibre \feq
	\begin{itemize}
		\warrow Équilibre en gains/pertes de molécules $i.e.$ énergie gagnée puis perdue
		\warrow Égalisation de la probabilité de présence de particules
		\warrow Prends en compte les lois de conservation de l'hydrodynamique (masse, \qttmvt$ $, énergie...)
	\end{itemize}
	\bwarrow Nécessité de trouver $\Omega$ $\rightarrow$ modèle de \BeGK
\end{itemize}


\end{frame}

\begin{frame}
\frametitle{La LBM dans le modèle de \BeGK}
\begin{block}{LBGK}
	 \BeGK $ $ introduisent en 1954 le LBGK qui consiste principalement à écrire : \\
	\begin{equation*}
	\Omega(f, f)= -\dfrac{1}{\tau} \bepar{f-f^{\text{eq}}}
	\end{equation*}
	\begin{equation*}
	 f^{\text{eq}} = \sum_{i=1}^{Q-1}\rho w_i \left(1 + 3 \frac{\vec{c_i}\cdot \vec{u}}{c^2} +  \frac{9}{2}\frac{\left( \vec{c_i} \cdot \vec{u} \right )^ 2}{c^4} - \frac{3}{2}\frac{\vec{u}\ ^2}{c^2}\right)
	\end{equation*}

\end{block}
\begin{itemize}
\bwarrow Ces deux expressions sont extrêmement importantes 
\end{itemize}

\end{frame}

\begin{frame}
\frametitle{Spécificité du code d'Emmanuel Lévêque}
\begin{itemize}
	\bwarrow Programmation Orientée Objet (POO) 
		\begin{itemize}
			\warrow Classes, attributs, méthodes
			\warrow Surcharges de méthodes pour gagner en lisibilité
		\end{itemize}
	\bwarrow Parallélisé : M(\small{\textit{essage}}) P(\small{\textit{assing}}) I(\small{\textit{nterface}}) 	
	\begin{center} \includegraphics[scale=0.32]{domaine_mpi} \end{center}
	\bwarrow Conditions aux limites tri-périodiques	avec GhostNodes
	\bwarrow Fichier d'entrée : dimensions des grilles, largeur de la tranche...
	\bwarrow Écrit pour la discrétisation D$_3$Q$_{19}$ que nous avons transcrit en D$_3$Q$_{27}$
	
\end{itemize}
\end{frame}

\begin{frame}
\frametitle{ D$_3$Q$_{19}$ et  D$_3$Q$_{27}$}
\begin{center}
\only<1>{\includegraphics[scale=0.32]{D3_Q19.png}}

\only<2>{\includegraphics[scale=0.32]{D3_Q27.png}}
\end{center}
\end{frame}


\begin{frame}
\frametitle{Dynamique en deux temps}
\begin{block}{La collision}
La collision est modélisée par la fonction d'équilibre qui est construite à partir des conservations des moments hydrodynamiques. Elle s'écrit :
 \begin{equation*}
 f^*_a\bepar{\vec{x}_a ,t} = f_a\bepar{\vec{x}_a ,t} - \dfrac{1}{\tau}\bepar{f_a\bepar{\vec{x}_a ,t}, f_a^{\text{eq}}\bepar{\vec{x}_a ,t}}
 \end{equation*}
\end{block}

\begin{block}{La propagation (streaming)}
Propagation des populations dans le sens de leur direction de propagation $\vec{c}_a$
\begin{equation*}
f\left(\vec{x}_a + \vec{c}_a\Delta t,t + \Delta t\right) = f^*_a\bepar{\vec{x}_a ,t}
\end{equation*}

\end{block}
 
\end{frame}

\begin{frame}
\frametitle{Collision et propagation}
\only<1>{
\centering
\includegraphics[scale=0.43]{streaming0.pdf}
}


\only<2>{
\centering
\includegraphics[scale=0.43]{streaming.pdf}
}
\only<3>{
\centering
\includegraphics[scale=0.43]{streaming2.pdf}
}
\only<4>{
\centering
\includegraphics[scale=0.43]{streaming1.pdf}
}


\end{frame}

\begin{frame}
\frametitle{Modifications des conditions aux limites}
\framesubtitle{Simuler une couche limite : modélisation du non glissement }

\begin{itemize}
	\bwarrow Conditions de non glissement au Sud
		\begin{itemize}
			\warrow Suppression communication Nord-Sud \textit{i.e.} de la périodicité Nord-Sud
			\warrow Modélisation de la paroi : annulation de la vitesse\\
	\end{itemize}
\end{itemize}
\begin{columns}
\begin{column}{0.6\textwidth}
	\begin{itemize}
		\bwarrow Le Bounce Back répond à la question de la provenance des populations orientées Nord dans les réseaux les plus proches de la paroi Sud.  
	\end{itemize}
\end{column}

\begin{column}{0.3\textwidth}
	\begin{center}
	\includegraphics[scale=0.4]{post-coll}
	\end{center}
\end{column}
\end{columns}

\end{frame}

\begin{frame}
\frametitle{Modifications des conditions aux limites}
\framesubtitle{Simuler une couche limite : modélisation de la condition de non glissement }

\only<1>{ \begin{itemize}\bwarrow Configuration initiale \end{itemize} \centering \includegraphics[scale=0.5]{precollision_ns.png}}
\only<2>{ \begin{itemize}\bwarrow Configuration intermédiaire : mi parcours propagation dans les Ghostnodes \end{itemize}\centering  \includegraphics[scale=0.4]{intermediaire_ns.png}}
\only<3>{ \begin{itemize}\bwarrow Configuration finale fin de l'itération \end{itemize} \centering \includegraphics[scale=0.5]{noslip.png}
\begin{itemize}
	\bwarrow Les particules ont "rebondi". Sans l'étape intermédiaire, on perd en précision.
\end{itemize}
		}


\end{frame}

\begin{frame}
\frametitle{Modifications des conditions aux limites}
\framesubtitle{Simuler une couche limite : modélisation du glissement }

\begin{itemize}
	\bwarrow Conditions de glissement au Nord
		\begin{itemize}
			\warrow Suppression communication Nord-Sud \textit{i.e.} de la périodicité Nord-Sud
			\warrow Modélisation du glissement : annulation de la composante de vitesse normale à la paroi \\
	\end{itemize}
\end{itemize}
\begin{columns}
\begin{column}{0.6\textwidth}
	\begin{itemize}
		\bwarrow Le Bounce Back spéculaire répond à la question de la provenance des populations orientées Sud dans les réseaux les plus proches de la limite Nord du domaine.
	\end{itemize}
\end{column}

\begin{column}{0.3\textwidth}
	\begin{center}
	\includegraphics[scale=0.4]{post-coll}
	\end{center}
\end{column}
\end{columns}

\end{frame}


%\begin{frame}
%\frametitle{Modifications des conditions aux limites}
%\framesubtitle{Simuler une couche limite : modélisation de la condition de glissement }
%
%\only<1>{ \begin{itemize}\bwarrow Configuration initiale \end{itemize} \centering \includegraphics[scale=0.5]{precollision.png}}
%\only<2>{ \begin{itemize}\bwarrow Configuration intermédiaire : mi parcours propagation dans les Ghostnodes \end{itemize}\centering  \includegraphics[scale=0.4]{intermediaire.png}}
%\only<3>{ \begin{itemize}\bwarrow Configuration finale fin de l'itération : réflexion spéculaire \end{itemize} \centering \includegraphics[scale=0.5]{freeslip.png}
%\begin{itemize}
%	\bwarrow Conservation de la vitesse tangentielle
%\end{itemize}
%		}
%
%
%\end{frame}

\begin{frame}
\frametitle{Validation des conditions aux limites}
\begin{itemize}
	\bwarrow Validation : profil initial uniforme puis établissement de la couche limite
\end{itemize}

\centering
\includegraphics[scale=0.2]{LB-BL}

\end{frame}

%\begin{frame}
%\frametitle{Conclusion}
%\begin{itemize}
%	\bwarrow Stage composé de deux approches différentes
%	\bwarrow Les résultats présentés et les discussions effectuées sont dans la continuité du projet d'étude de dispersion de gaz lourds :
%		\begin{itemize}
%			\warrow Volonté de se rapprocher des résultats de Briggs sur la caractérisation de la vitesse verticale d'entrainement par le nombre de Richardson
%			\warrow Préparation : du code Lattice Boltzmann pour 
%				\begin{itemize}
%					\sarrow Utiliser des méthodes plus élaborées que la LBGK
%					\sarrow Conditions aux limites correspondant à un problème de couche limite écrites (testé pour de hauts Reynolds $\sim$ 500,000)
%				\end{itemize}
%		\end{itemize}
%	\bwarrow Rencontres de nombreux problèmes qui ont ralentis notre progression
%	\bwarrow Excellente expérience pour la suite
%\end{itemize}
%\end{frame}

\begin{frame}
\frametitle{Conclusion de la présentation}
\framesubtitle{Ce stage et sa suite}
\only<1>{
\begin{itemize}

	\bwarrow Deux approches différentes

	\begin{columns}

	\begin{column}{0.45\textwidth}
	\begin{itemize}
		\warrow Études autour du \keps
			\begin{itemize}
				\sarrow Modèle et approximations \\[2mm]
				\sarrow Construction simulation	\\[2mm]
				\sarrow Retrouver les résultats de Briggs (2001) \\[2mm]
			\end{itemize}

	\end{itemize}
	\end{column}

	\begin{column}{0.45\textwidth}
		\begin{itemize}
		\warrow Études autour de la LBM
			\begin{itemize}
				\sarrow Méthode et dynamique \\[2mm]
				\sarrow Éléments sur la précision et la stabilité \\[2mm]
				\sarrow Implémentation des conditions aux limites \\[2mm]
			\end{itemize}
		\end{itemize}
	\end{column}
	\end{columns}
		\end{itemize}
	}
	\only<2>{
	\begin{itemize}
	\bwarrow S'inscrivant dans la suite 	des projets		
	\begin{columns}
	\begin{column}{0.45\textwidth}
			\begin{itemize}
				\warrow \keps
				\begin{itemize}
					\sarrow Problématiques avec obstacles \\[2mm]
					\sarrow Thèse expérimentale à venir pour comparaison \\[2mm]
					\sarrow Détermination de paramètres de contrôle et loi de similitude \\[2mm]
				\end{itemize}
			\end{itemize}		
	\end{column}
	\begin{column}{0.45\textwidth}	
			\begin{itemize}
				\warrow LBM				
				\begin{itemize}
					\sarrow D$_3$Q$_{19}$ à D$_3$Q$_{27}$ pour méthode plus élaborée \\[2mm]
					\sarrow Implémentation de stratification, ou force de flottabilité \\[2mm]
					\sarrow injection de scalaire dans le domaine \\[2mm]
				\end{itemize}
			\end{itemize}
	\end{column}
	\end{columns}
\end{itemize}
}
\end{frame}

\begin{frame}
\frametitle{Merci}
\begin{center}
	\brick \textbf{\textit{\Large{Merci pour votre attention}}}
\end{center}
\end{frame}

\appendix


\begin{frame}[noframenumbering]
\frametitle{Annexe}
\framesubtitle{Le nombre de Knudsen et la limite incompressible}
\begin{itemize}
	\bwarrow Nombre de Knudsen (Kn):
	\begin{itemize}
		\warrow Compare échelles micro sur macro
		\warrow Définit le modèle requis pour la description d'un problème 
	\end{itemize}
\end{itemize}
\begin{exampleblock}{Le Kn indique l'outil de description à utiliser}
Plus les échelles sont proches plus on devra raffiner la description du milieu
\end{exampleblock}
\begin{block}{Expressions de Kn}
\begin{equation*}
\text{Kn} = \frac{l_\mu}{l_M} = \frac{\tau_\mu}{\tau_M} = \frac{\text{Ma}}{\text{Re}}
\end{equation*}
La limite incompressible : ajuster le Mach pour que l'équation de Boltzmann puisse décrire l'hydrodynamique
\end{block}
\end{frame}

\begin{frame}[noframenumbering]
\frametitle{Annexe}
\framesubtitle{La LBM dans le modèle de \BeGK}
\begin{block}{LBGK}
	 \BeGK $ $ introduisent en 1954 le LBGK qui consiste à écrire : \\
	\begin{equation*}
	\Omega(f, f) = -\dfrac{1}{\tau} \bepar{f-f^{\text{eq}}}
	\end{equation*}

	\begin{equation*}
	 f^{\text{eq}} = \sum_{i=1}^{Q-1}\rho w_i \left(1 + 3 \frac{\vec{c_i}\cdot \vec{u}}{c^2} +  \frac{9}{2}\frac{\left( \vec{c_i} \cdot \vec{u} \right )^ 2}{c^4} - \frac{3}{2}\frac{\vec{u}\ ^2}{c^2}\right)
	\end{equation*}

	\begin{equation*}
	c_s = \dfrac{c}{\sqrt{3}} \text{\hspace{5mm} et \hspace{5mm}} c = \dfrac{\Delta x}{\Delta t} = 1
	\end{equation*}	

\end{block}
\end{frame}

\begin{frame}[noframenumbering]
\frametitle{Raisons du passage D$_3$Q$_{19}$ $\rightarrow$ D$_3$Q$_{27}$}

\begin{itemize}
\bwarrow Pour améliorer la stabilité et augmenter le nombre de Mach maximum 
	\begin{itemize}
		\warrow D$_3$Q$_{27}$ admet plus de degrés de liberté que D$_3$Q$_{19}$ 
		\begin{itemize}
			\sarrow Plusieurs vitesses du son possibles
			\sarrow Poids $w_a$ associé à chaque densité de propagation $c_a$, non fixé 
		\end{itemize}
		\warrow Prends en compte des tenseurs d'ordre supérieur corrigeant certaines erreurs et modes spurieux \\[5mm]
	\end{itemize}
\bwarrow Pour utiliser la méthode dite cascadée 
	\begin{itemize}
		\warrow Méthode plus robuste et plus stable, permettant la considération de Mach plus élevés (proche de l'unité) 
		\warrow Nécessitant que $Q = 3^D$
	\end{itemize}
\end{itemize}
 \end{frame}

\begin{frame}[noframenumbering]
\frametitle{La procédure de \ckng $ $ : quelques résultats}
\begin{itemize}
\bwarrow La procédure de Chapman-Enskog permet d'écrire : \\
\begin{center} $\displaystyle 	\nu = c_s^2 \bepar{\tau - \dfrac{\Delta t}{2}} $ \end{center}
\bwarrow Elle permet d'obtenir les solutions de Navier-Stokes avec une erreur d'ordre $\mathcal{O}\bepar{u^3}$
\bwarrow Fonction d'équilibre construite pour $\text{Ma}^2 \ll 1$
\bwarrow Prendre une vitesse $U_{\text{LB}} $ assez faible permet de construire un Mach \textit{numérique faible} et réduire les erreurs. \\[3mm]
\bwarrow Existence de modèle avec plusieurs temps de relaxation
\end{itemize}
\end{frame}

\begin{frame}[noframenumbering]
\frametitle{Modifications des conditions aux limites}
\framesubtitle{Simuler une couche limite : modélisation de la condition de glissement }

\only<1>{ \begin{itemize}\bwarrow Configuration initiale \end{itemize} \centering \includegraphics[scale=0.5]{precollision.png}}
\only<2>{ \begin{itemize}\bwarrow Configuration intermédiaire : mi parcours propagation dans les Ghostnodes \end{itemize}\centering  \includegraphics[scale=0.4]{intermediaire.png}}
\only<3>{ \begin{itemize}\bwarrow Configuration finale fin de l'itération : réflexion spéculaire \end{itemize} \centering \includegraphics[scale=0.5]{freeslip.png}
\begin{itemize}
	\bwarrow Conservation de la vitesse tangentielle
\end{itemize}
		}
\end{frame}

\begin{frame}[noframenumbering]
\frametitle{Annexe}
\framesubtitle{Richardson Vs $U_p$}
\begin{figure}[!ht]
\centering
\includegraphics[scale=0.25]{ri_vs_up_g1.png}
\end{figure}
\end{frame}



\end{document}

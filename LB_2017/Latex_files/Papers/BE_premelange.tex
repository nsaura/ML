\documentclass[a4paper,10pt]{article}
\usepackage[utf8x]{inputenc}
%\usepackage[french]{babel}
\usepackage{amsmath}
\usepackage{amsfonts}
\usepackage{amssymb}
\usepackage{graphicx}		 			% Inclusion des figures 
\usepackage{textcomp}
\usepackage[nointegrals]{wasysym}			% Collection de symboles mathématiques
\usepackage{multicol}					% Pour utiliser \hfill
\usepackage{ifthen}
\usepackage{tabularx}	 				% Gestion avancée des tableaux
%\usepackage[squaren]{SIunits}
%\usepackage{sistyle} 					% Mise en forme des unités
%\usepackage{fancyhdr} 					% Entêtes et pieds de pages personnalisés 
%\usepackage{latexsym}
%\usepackage{xcolor}					% Pour redéfinir des couleurs 
\usepackage[svgnames]{xcolor}				% Voir : http://calque.pagesperso-orange.fr/latex/latexps.html
\usepackage{latexsym}
\usepackage{fontenc}
\usepackage{wrapfig} 					% Pour pouvoir placer une image à côté de texte
\usepackage{cite}	
\usepackage{url} 					% Pour citer les sites internet dans la bilibographies
\usepackage{setspace}
\usepackage[T1]{fontenc}				% Indispendable, présent dans tous les codes exemples
\usepackage[linkcolor=red,colorlinks=true]{hyperref} 	% Hyper ref
\usepackage{listings}					% Pour citer du code
\usepackage[justification=centering]{caption}
\usepackage{subcaption}
\usepackage{subfloat}


\newcommand\pathpic{/home/saura/Documents/Latex_files/Pic/BE_combustion/tp2/}
%\newcommand\spipath{/home/saura/Documents/Stage/Notes_Manip/Pic/Simu_pic/}

%%%%%%%%%% Raccourcis %%%%%%%%%%%
\newcommand{\lp}{\left(}
\newcommand{\rp}{\right)}
\newcommand{\rint}{\int^\infty_{-\infty}}
\newcommand{\dint}{\int^\infty_{0}}
\newcommand{\rsum}{\sum^\infty_{ j = -\infty}}
\newcommand{\cad}{c'est-à-dire}
\newcommand{\qcq}{quelconque}
\newcommand{\keps}{$k-\varepsilon$}
\newcommand{\chq}{$Ch_4$ }
\newcommand{\od}{$O_2$ }

%%% Couleurs %%%
\xdefinecolor{purple}{named}{MediumVioletRed}
\xdefinecolor{brick}{named}{DarkRed}
\xdefinecolor{forest}{named}{DarkMagenta}
\xdefinecolor{dgreen}{named}{DarkOliveGreen}
\xdefinecolor{darker}{named}{SaddleBrown}
\xdefinecolor{MidnightBlue}{named}{MidnightBlue}

\newcommand\brick{\color{brick}}
\newcommand\npurple{\color{forest}}
\newcommand\dgreen{\color{dgreen}}
\newcommand\darker{\color{Maroon}}
\newcommand\dblue{\color{MidnightBlue}}

\newcommand\black{\color{black}}
\newcommand{\bl}{\color{blue}}
\newcommand\red{\color{red}}

\usepackage{caption}
\usepackage[labelfont={color = Indigo, bf}, font = it ,justification=centering]{caption}

%%%%%%%%%%%%%%%%%%%

%opening
\title{ \dblue BE combustion : Flamme de pré-mélange}
\author{ \darker Patacini Lucas \& Saura Nathaniel}


\usepackage{geometry}
\geometry{hmargin=1.4cm, vmargin=1.2cm}

\begin{document}
\maketitle 
\section*{Introduction}
Cette étude propose d'étudier une flamme partiellement pré-mélangée dans un brûleur. Dans l'étude d'une flamme de pré-mélange classique, le comburant et le carburant sont pré-mélangés avant d'être injectés à l'inverse d'une flamme de diffusion où le comburant et la carburant se mélange dans le milieu où se trouve le comburant (air par exemple). \\
Notre cas est légèrement différent puisque après injection du pré-mélange dans le brûleur un second jet d'air apporte de l'air dans le pré-mélange et le confine. L'injection du mélange air/carburant (ici méthane) possède une composante axiale et tourbillonnaire.\\
 Au cours de cette simulation, nous allons tenter d'étudier la flamme provoquée par une telle injection ainsi que différentes caractéristiques comme la répartition des composants, des produits etc.   

\section{Évolutions des constituants au cours de la réaction}
Dans une flamme de pré-mélange, la réaction commence lorsque les produits atteignent une certaine température appelée température d'ignition $T_i$. Lorsque cette valeur est atteinte, le phénomène d'explosion cinétique se produit, et les produits commencent à être formés. \\
La zone précédent cette explosion cinétique est marqué par une évolution de la température plutôt lisse : c'est la zone de préchauffage. Puis la température augmente très vite (explosion cinétique) jusqu'à stagner à la température des produits. C'est dans cette zone d'évolution rapide que se produit les réactions comme on peut le voir dans les deux figures suivantes : 

\begin{figure}[ht!]
\centering
\includegraphics[width = 9cm, height = 5cm]{\pathpic temperature_selon_x.png} \hfill
\includegraphics[width = 9cm, height = 5cm]{\pathpic formation_produit.png}
\caption{À gauche nous avons tracé l'évolution de la température le long de l'axe $x$. À droite, la zone dans laquelle se produisent des réactions. Cette zone commence au même point que celui qui marque le changement de profil d'évolution de la température $x \approx 0.4$.}
\end{figure}

%\pagebreak

\noindent On vérifie avec un contour, la localisation de la création de produits :

\begin{figure}[ht!]
\centering
\includegraphics[width = 9cm, height = 4cm]{\pathpic contour_temperature.PNG} \hfill
\includegraphics[width = 9cm, height = 4cm]{\pathpic formation_produit_contour.png}
\caption{À gauche nous avons tracé le contour de température dans le brûleur. À droite nous avons représenté la zone de formation de produit.}
\label{formation}
\end{figure}

\noindent La figure de droite nous permet de localiser la flamme. Dans la figure de gauche, on distingue la zone de préchauffage (du bleu au vert) qui s'étend sur plus de la moitié du brûleur. Puis dans une zone très courte, la température passe de $1200\ K$ à $\approx 2000\ K$ qui correspond à la zone d'explosion cinétique. La zone rouge à l'aval de la flamme correspond aux zones de produits brûlés. \\
Si nous augmentions la température du jet pré-mélangé, la zone de pré-chauffage s'étendrait moins en espace et la flamme apparaîtrait plus tôt. \\   
Enfin, on remarque que dans le coin supérieur gauche, la température est élevée bien qu'il n'y ait aucune réaction ; et comme nous le voyons figure (\ref{stream}), cette température est due à la recirculation des produits brûlés (donc chauds) dans cette zone.

\begin{figure}[ht!]
\centering
\includegraphics[scale=0.4]{\pathpic contour_stream_function.PNG}
\caption{Tracé de la fonction de courant. Cette valeur permet de suivre le mouvement des particules fluides. Nous observons une zone de circulation dans la partie supérieure qui explique que dans cette zone la température est élevée malgré l'absence de réaction voir figure (\ref{formation}).}
\label{stream}
\end{figure}

On effectue des coupes selon différents $x$ afin d'obtenir l'évolution de la température selon $y$ aux différents endroits du brûleur :

\begin{figure}[ht!]
\centering
\includegraphics[scale=0.4]{\pathpic xy_temperature.jpg}
\caption{}
\label{temperature_densite} 
\end{figure}

\noindent Pour les coupes à $x=0.2$ et $x=0.3$ la température augmente en s'éloignant de l'axe des abscisses car nous sommes dans la zone de recirculation des produits brûlés. La température initiale est celle du mélange injecté.\\
Pour la coupe $x=0.4$, la température initiale est supérieure à celle du mélange injecté, les réactions chimiques ont commencé ce qui augmente alors la température. Entre $y=0.03$ et $y=0.06$, nous mesurons la température de la flamme ; au-delà, la température stagne à une valeur voisine de celle des deux courbes précédentes, c'est la température des produits brûlés.\\
Pour la coupe $x=0.5$, la flamme est très proche de l'axe des abscisse (voir figure \ref{formation}) et il est donc normal de retrouver une température très élevée. Au-delà de $y = 0.06$ on mesure la température de l'air contenant les produit brûlés.\\
Enfin la dernière coupe se situe pour une abscisse pour laquelle, il n'y que des produits brûlés, la température ne varie pas beaucoup sur tout l'axe $y$.

\pagebreak 

\noindent En complément d'analyse on peut visualiser la répartition des constituants dans le brûleur :

\begin{figure}[ht!]
\centering
\includegraphics[width = 9cm, height = 5cm]{\pathpic contour_O2.PNG} \hfill
\includegraphics[width = 9cm, height = 5cm]{\pathpic contour_ch4.PNG}
\caption{À gauche : contour de fraction massique de l'oxygène ; à droite contour de fraction massique du \chq , notre carburant. Le méthane injectée est totalement brûlée, on en retrouve donc plus en dehors de la flamme. La zone d'extinction de sa fraction massique est un bon indicateur de la longueur de la flamme. À gauche la présence d'oxygène en dehors de la limite de flamme s'explique par une combustion pauvre en carburant par rapport à l'apport d'air.  }
\label{contour_reactif}
\end{figure}

\begin{figure}[ht!]
\centering
\includegraphics[width = 9cm, height = 5cm]{\pathpic xy_o2.jpg} \hfill
\includegraphics[width = 9cm, height = 5cm]{\pathpic xy_ch4.jpg}
\caption{Coupes à différents $x$ des fractions massiques de l'$O_2$ (à gauche) et du méthane (à droite). À différents $x$ l'épaisseur de la zone de présence de \chq $ $ est à peu près la même, jusqu'à que l'on ait atteint le bout de la flamme. Pour l'$O_2$, en dehors de la zone de combustion, on observe une fraction massique non nulle causée par la recirculation. }
\label{densite_o2_ch4}
\end{figure}

Dans la figure (\ref{densite_o2_ch4}), on voit que l'épaisseur du jet de \chq $ $ ne varie pas jusqu'à ce qu'on ait atteint la zone de combustion. Cette conservation de l'épaisseur peut être expliquée par la vitesse \textit{swirl} : l'entraînement hélicoïdal induit par cette vitesse limite les effets de diffusion latérale et confine cette flamme.\\
Au niveau de l'oxygène, on distingue deux zones : dans le jet (pour $y \in [0; 0.06]$), la concentration d'$O_2$ est importante (coupe 0.2, 0.3 et 0.4) jusqu'à la fin de la flamme puis la deuxième zone dans laquelle nous avons une fraction massique constante (non-nulle). Cette valeur est assurée par les cellules de convection vues plus haut.\\
Pour des coupes à $x$ plus grand, on observe plutôt la tendance inverse : la fraction massique est très faible (dans la zone  $y \in [0; 0.06]$) puisque l'$O_2$ est presque entièrement brûlé, puis converge dans le reste de la coupe vers une valeur assez proche de celle qu'on avait pour les autres coupes. \\

\noindent Au niveau des produits, voici les contours obtenus :
\begin{figure}[ht!]
\centering
\includegraphics[width = 9cm, height = 4cm]{\pathpic contour_CO2.PNG} \hfill
\includegraphics[width = 9cm, height = 4cm]{\pathpic contour_H2O.PNG}
\caption{Contour des produits : à gauche $C0_2$ à droite $H_2O$. La production de ces éléments n'a pas lieu au même endroit, nous supposons que leur création est dépendante de la température.}
\label{contour_Cod}
\end{figure}

\noindent L'azote ($N_2$) est présent dans les réactifs mais est considérée comme élément inerte (CF cours). Dans les produits il y a également le monoxyde de carbone ($C_0$).

\section{Vitesse et influence}

On commence par regarder le contour de la magnitude du vecteur vitesse (comme demandée) ainsi que les vecteurs vitesses :

\begin{figure}[ht!]
\centering
\includegraphics[width = 13 cm, height = 6cm]{\pathpic contour_vitesse.PNG}
\caption{Tracé du contour de vitesse (magnitude).}
\label{vitesse}
\end{figure} 

\begin{figure}[ht!]
\centering
\includegraphics[width = 13 cm, height = 6cm]{\pathpic vector_vitesse.PNG}
\caption{Tracé des vecteurs de la vitesse pour une meilleure visualisation du champ de vitesse.}
\label{vitesse_vec}
\end{figure} 
\noindent Sur la figure (\ref{vitesse_vec}), on voit que le flux d'air pur injecté est aspiré par le pré-mélange carburant-air injecté en swirl. Cette aspiration stabilise le jet de pré-mélange et assure une forme cylindrique du jet dans les débuts de l'injection. Le jet s'élargit lorsque qu'il traverse la flamme, ce qui crée une augmentation de la vitesse radiale au dépend du swirl. \\
La composante swirl de la vitesse est très importante ; elle augmente la probabilité de contact entre l'air et le carburant, ce qui accélère l'augmentation de la température et donc l'apparition de la flamme. 

\section*{Conclusion}
\indent Dans ce projet, nous avons pu étudier les caractéristiques d'une flamme partiellement pré-mélangée et discuter de l'effet du "swirl". Cette composante de la vitesse impliquait une injection hélicoïdale favorisant le mélange avec le second jet composé d'air pur. \\
 Au travers la figure (\ref{stream}) nous avons pu observer des zones de recirculation de produits chauds mais également de l'oxygène, ce qui nous indiquait une combustion pauvre (\ref{contour_reactif}). Nous avons également mis en évidence l'effet de la température initiale du jet sur la flamme : si on augmente la température initiale, la zone de pré-chauffage s'en retrouverait réduite ce qui donnerait une flamme moins longue et une production de produits plus tôt.


\end{document}
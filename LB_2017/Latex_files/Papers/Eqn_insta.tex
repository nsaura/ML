\documentclass[a4paper,10pt]{article}
\usepackage[utf8]{inputenc}
\usepackage{amsmath, amsfonts, amssymb}
%\sepackage[linkcolor=red,colorlinks=true]{hyperref}

\usepackage{geometry}
\geometry{hmargin=2.5cm, vmargin=1.65cm}

%%%%% Raccourcis %%%%%%
\newcommand{\rhoxt}{\rho(\vec{x},t)}

%opening
\title{Équations pour la présentation}
\author{Nathaniel Saura}

\begin{document}
\maketitle


\section{Vous avez dit incompressible ?}
Lorsqu'on parle d'incompressibilité d'un écoulement, on a souvent en tête 
\begin{equation}
 \nabla.\vec{u} = 0 \label{incompfalse}
\end{equation}
En vérité, la condition d'incompressibilité concerne le volume. Elle s'écrit alors 
\begin{equation}
 \frac{DV}{Dt} = 0 
\end{equation}
Cette équation nous dit que dans un fluide incompressible, le volume d'une particule fluide donnée peut changer de forme, mais il restera constant au cours du temps tout au long de son mouvement.
En creusant un petit peu et en utilisant la définition du volume : $V = \frac{m}{\rhoxt}$, on a 
\begin{equation}
 \frac{DV}{Dt} = \frac{D[m/\rhoxt]}{Dt} = -\frac{m}{\rho^2(\vec{x},t)} \frac{D\rhoxt}{Dt} = 0 \Rightarrow \frac{D\rhoxt}{Dt} = 0 \label{incomptrue}
\end{equation}
Conclure que $\rhoxt$ est une constante serait une erreur (commune) puisque se cache derrière cette équation le terme non linéaire d'advection ; en effet :
\begin{equation}
 \frac{D\rhoxt}{Dt} = \frac{\partial \rhoxt}{\partial t} + \vec{u}.\nabla\rhoxt =0
\end{equation}
On n'aura donc sûrement pas $\rhoxt = cst$. \\
On peut retrouver le caractère solénoïdal de l'écoulement incompressible, grâce à la conservation de la masse que la physique impose par elle-même. Cette égalité s'écrit :
\begin{equation*}
\frac{\partial \rhoxt}{\partial t} + \nabla.(\rhoxt \vec{u}) = 0
\end{equation*}
\begin{align*}
	\intertext{En développant le dernier terme à gauche du signe égal}
	\frac{\partial \rhoxt}{\partial t} + \vec{u}.\nabla \rhoxt &+ \rhoxt \nabla\vec{u} = 0 \\
	\intertext{ Or }
	\frac{D \rhoxt}{D t} = \frac{\partial \rhoxt}{\partial t} &+ \vec{u}.\nabla \rhoxt = 0 \\
	\intertext{D'où} 
	\nabla . \vec{u} &= 0
\end{align*}
Ainsi, en partant de la conservation du volume d'une particule fluide au cours du temps, on a retrouvé la conservation de la masse volumique de cette particule fluide au cours de son épopée et le caractère solénoïdal que cela implique. L'équation de continuité $\nabla .\vec{u} = 0$ est donc une conséquence de l'incompressibilité du volume d'une particule fluide et non sa définition.

%\pagebreak


\end{document}

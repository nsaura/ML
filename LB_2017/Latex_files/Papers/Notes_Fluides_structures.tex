\documentclass[a4paper,10pt]{article}
\usepackage[utf8]{inputenc}
\usepackage{graphicx}		 			% Inclusion des figures 

\usepackage{multicol}					% Pour utiliser \hfill
\usepackage{wrapfig} 					% Pour pouvoir placer une image à côté de texte

\usepackage{amsmath, amsfonts, amssymb}
%\sepackage[linkcolor=red,colorlinks=true]{hyperref}
\usepackage[nointegrals]{wasysym}			% Collection de symboles mathématiques
\usepackage[svgnames]{xcolor}

%%%%%%%%%%%%%%%%%%%
%%% Couleurs %%%
\xdefinecolor{purple}{named}{MediumVioletRed}
\xdefinecolor{brick}{named}{DarkRed}
\xdefinecolor{forest}{named}{DarkMagenta}
\xdefinecolor{dgreen}{named}{DarkOliveGreen}

\newcommand{\brick}{\color{brick}}
\newcommand{\npurple}{\color{forest}}
\newcommand{\dgreen}{\color{dgreen}}

\newcommand{\bl}{\color{blue}}
\newcommand{\bk}{\color{black}}
\newcommand{\rd}{\color{red}}
%%%%%%%%%%%%%%%%%%%

\newcommand{\pathpic}{/home/saura/Documents/Latex_files/Pic/}

\usepackage{geometry}
\geometry{hmargin=1.4cm, vmargin=1cm}

%opening
\title{Notes Interaction fluides structures}
\author{N-S }

\begin{document}
%\maketitle
\vspace{-1cm} \bl
\section{Écoulements polyphasiques} \bk
\noindent \textbf{$\rightarrow$ Interface :\\}
Zone de séparation entre deux phases liquides. Frontières libres, mobiles et déformables, d'épaisseurs négligeables. Elle imposent des conditions aux limites particulières pour les fluides des deux phases. Eau-Air : différents régimes :

\begin{figure}[!ht]
\centering
\includegraphics[height = 5.5cm, width = 8 cm]{\pathpic bubly1.png} \hfill
\includegraphics[height = 5.5cm, width = 8 cm]{\pathpic bubly2.png}
\caption{Différents régimes en conduite verticale}
\end{figure}

\begin{figure}[!ht]
\centering
\includegraphics[height = 5cm, width=9cm]{\pathpic horizontal.png}
\caption{Différents régimes en conduite horizontale}
\end{figure}

\begin{figure}[!ht]
\centering
\includegraphics[height = 6cm, width=8cm]{\pathpic cdr_vert.png} \hfill
\includegraphics[height = 6cm, width=8cm]{\pathpic cdr_horiz.png}
%\caption{On a recourt aux cartes de régimes}
\end{figure}
Avec pour la figure de gauche \bl
\begin{equation*}
j_G = \frac{Q_g}{A} \text{\hspace{2mm} \bk et\bl \hspace{2mm}} j_L = \frac{Q_L}{A}
\end{equation*}
\bk Et pour la figure de gauche :

\begin{figure}[!ht]
\centering
\includegraphics[height = 1.5 cm, width = 3cm]{\pathpic X.png} \hfill
\includegraphics[height = 2 cm, width = 3.5cm]{\pathpic T.png} \hfill
\includegraphics[height = 1.7 cm, width = 3.5cm]{\pathpic F.png} \hfill
\includegraphics[height = 1.5 cm, width = 5.5cm]{\pathpic K.png} 
\end{figure}

\end{document}
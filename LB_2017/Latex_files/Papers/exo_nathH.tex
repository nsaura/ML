\documentclass[a4paper,10pt]{article}
\usepackage[utf8]{inputenc}
\usepackage{amsmath, amsfonts, amssymb}
%\sepackage[linkcolor=red,colorlinks=true]{hyperref}
\usepackage{graphicx} % inclusion des figures

\usepackage[svgnames]{xcolor}

%%%%%%%%%%%%%%%%%%%
%%% Couleurs %%%
\definecolor{orange}{rgb}{0.99,0.69,0.07}
\xdefinecolor{purple}{named}{MediumVioletRed}
\xdefinecolor{brick}{named}{DarkRed}
\xdefinecolor{dmag}{named}{DarkMagenta}
\xdefinecolor{dgreen}{named}{DarkOliveGreen}

\newcommand\black{\color{black}}
\newcommand\brick{\color{brick}}
\newcommand\npurple{\color{dmag}}
\newcommand\dgreen{\color{dgreen}}
%%%%%%%%%%%%%%%%%%%

\usepackage{geometry}
\geometry{hmargin=2.5cm, vmargin=1.65cm}

\newcommand{\subsub}{\subsubsection}
\newcommand{\sub}{\subsection}

\begin{document}

\section{Règle de trois }
\sub{Classiques}
\subsub{}
Si 7 zombies tués me rapportent 820 points combien un zombie me rapporte de points ?\\
Combien de zombies devrais je tuer pour obtenir 1500 points ?(Si le résultat est un nombre décimal, arrondir au supérieur)

\subsub{}
Si mon chauffage consomme 1636 kW par heure, et que je le fais tourner 3 heures par jour pendant 6 mois (de 30 jours) combien de kWH aurais-je consommé en tout ?\\
Si en plus le kWH coûte 0.1135 euros combien aurais-je dépensé à la fin de ces 6 mois ?\\
Enfin, au bout de combien de kWH aurais-je dépassé mon budget mensuel de 150 euros de chauffage ? Convertir ce résultat en heure de fonctionnement du chauffage.

\sub{Conversions-règle de trois}
\subsub{}
On considère qu'un zombie pèse en moyenne 55 kg (avant le passage en boucherie). Au cours d'une partie, le joueur 1 tue 162 zombies et le joueur 2 en tue 143 ; Donner en gramme le poids des zombies tués par joueur puis le poids total.\\
Votre patron décide de vous payer 12 euros pour tout les 150 grammes de zombie tué, combien chaque joueur aura gagné d'argent à la fin de la partie ? 

\subsubsection*{Notion de vitesse}
On définit la vitesse V comme la distance parcourue X en un temps T. Avec ces notations, on écrit
\begin{equation*}
V = \frac{X}{T}
\end{equation*}
La vitesse est exprimée en unité de longueur par unité de temps, c'est-à-dire, si l'unité de longueur est le mètre et que l'unité de temps est la seconde alors la vitesse sera exprimée en mètre par seconde. Si par contre l'unité de longueur est le centimètre et que l'unité de temps est la milliseconde alors la vitesse sera exprimée en centimètre par millisecondes.\\
Pour bien comprendre voici un \npurple{\textbf{exemple}} \black :\\
Je suis sur l'autoroute, limitée à 110 km/H (ici l'unité de longueur est le kilomètre et l'unité de temps est l'heure), mais je roule à 185 km/h parce que je suis pressé. Cette vitesse signifie que je parcours 185 kilomètre à chaque heure que je passe à conduire. On peut s'amuser à convertir cette unité : sachant que dans une heure j'ai 60 minutes et que dans une minute j'ai 60 secondes, je peux dire que dans une heure j'ai 60 fois 60 secondes c'est-à-dire 3600 secondes. De même dans un kilomètre j'ai 1000 mètres donc dans 185 kilomètre j'ai $185 \times 1000$ mètres ainsi, si je veux transformer ma vitesse de km/H à m/s, je calcule :
\begin{equation*}
V = \frac{185 \times 1000}{1 \times 3600} = 51.38 m/s
\end{equation*}
Donc en roulant à 185 km/H je parcours 51.38 mètres en une seconde (c'est ce qui s'appelle dans le monde physique passer en fond de balle).\\
Pour donner des ordres de grandeur, le son dans l'air parcourt 300 mètres en une seconde et la lumière 300 millions de mètres par seconde.\\
Un premier exercice consisterait à convertir la vitesse du son et de la lumière par en kilomètres par heures. \\

\subsub{}
Si Usain Bolt parcourt 100 mètres en 9.81 secondes Calculer sa vitesse globale.\\
\subsub{}
Un zombie énervé peut faire un sprint atteignant les 3 m/s. Si il arrive à courir à cette vitesse pendant 1 heure, combien de kilomètres aura-t-il parcouru ?\\
Si je suis à 7 mètres de lui combien de secondes ai-je pour le tuer avant qu'il ne m'atteigne ? \\
Adapter ce problème à un rampant qui galope à 5 m/s.

%\section{Problèmes}

\pagebreak

\section{Calcul d'aires et de volumes}


\subsub{}
On se base sur la plan figure (\ref{appart}). Les "bedrooms" seront abrégées par B, le "living-room" LR, la "bathroom" Ba et la "kitchen" K. \\
On précise que les aires s'ajoutent entre elles et que les volumes s'ajoutent entre eux \textbf{si ils sont indépendants}. Par exemple si B1 (bedroom 1) à une aire $A_1$ et un volume $V_1$ et que B2 a une aire $A_2$ et un volume $V_2$ alors l'aire totale $A_T$ et le volume total $V_T$ seront égaux à
\begin{equation*}
A_T = A_1 + A_2 \text{\hspace{2 cm}} V_T = V_1 + V_2
\end{equation*}
Par convention, on appellera longueur une mesure \textbf{verticale} et largeur une mesure \textbf{horizontale}. 

\begin{figure}[!h]
\centering
\includegraphics[width=7cm, height = 5cm]{/home/saura/Bureau/plan-d-un-appartement-t3.jpg}
\caption{Plan d'un appart à New-York}
\label{appart}
\end{figure}

On assimile dans cet exercice B1 à un carré de coté $c = 3$ mètres et B2 à un rectangle de longueur 4 mètres et de largueur 2,5 mètres. Calculer l'aire de chaque chambre, puis l'aire totale. \\
Si la hauteur sous plafond est de 3 mètres, calculer les volumes respectifs et le volume total.

\end{document}

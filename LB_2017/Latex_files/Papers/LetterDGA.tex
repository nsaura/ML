\documentclass[french, roman, 11pt]{article}
\usepackage[utf8x]{inputenc}
\usepackage[T1]{fontenc}
\usepackage[french]{babel}
\usepackage{amsmath}
\usepackage{graphicx}
\usepackage{geometry}
\usepackage{enumitem}
\usepackage{array}    %% Pour les tableaux
\geometry{hmargin=2.0cm, vmargin = 2.7cm}
%%% Personnalisation des en-têtes et pieds de pages
\usepackage{fancyhdr}
\pagestyle{fancy}

%\renewcommand{\footrulewidth}{1pt} %%% Une barre en haut
\renewcommand{\headrulewidth}{10pt} %%% Une barre en haut
\fancyfoot[L]{\leftmark}
\fancyfoot[C]{\thepage}

\usepackage[svgnames]{xcolor}
\xdefinecolor{brick}{named}{DarkRed}
\xdefinecolor{navy}{named}{Navy}


\newcommand{\brick}{\color{brick}}
\newcommand{\navy}{\color{navy}}

\usepackage{scrbase}
\newcommand{\pathpic}{/home/saura/Documents/Latex_files/Pic/}

\usepackage[linkcolor=red,colorlinks=true,urlcolor=DarkRed]{hyperref}

%\newcommand{\scidatalogo}{\includegraphics[height=50pt]{\pathpic ministeredefense.png}}
\pagestyle{headings}

\renewcommand{\headrulewidth}{1pt}
\setlength{\headheight}{45pt} 
\lhead{\textsc{\textbf{Nathaniel Saura}} \\ \textbf{+33 (0)7 82 47 09 35 \\ \href{mailto:nathaniel.saura@etu.univ-lyon1.fr}{nathaniel.saura@etu.univ-lyon1.fr}}}
%\rhead{\textsc{\scidatalogo}}

\pagenumbering{gobble} % To remove page counting

\begin{document}
\thispagestyle{fancy}

\noindent \textsc{\textbf{\navy \underline{Candidature pour la proposition de thèse}} : "Simulation numérique instationnaire de dispersion atmosphérique
d’un gaz lourd par la méthode Boltzmann sur réseau ; étude des fluctuations extrêmes de concentration."} \\[4mm]

\noindent À l'intention de Mesdames et Messieurs du Jury, \\[2mm]

\hspace{8 mm }Futur diplômé d'un Master 2 recherche en Mécanique des Fluides et Énergétique à l'École Doctorale \textsc{MEGA}, ainsi que d'un diplôme d'ingénieur Mécanique et Modélisation (École Polytechnique Universitaire de Lyon \textit{\textsc{EPUL}}), je souhaiterais me spécialiser dans la recherche en Mécanique des Fluides et plus particulièrement dans l'utilisation de la méthode des réseaux de Boltzmann couplée avec des méthodes statistiques innovantes pour l'étude de la dispersion atmosphérique de gaz lourds. \\

J'ai effectué un stage de M1 au Laboratoire de Mécanique des Fluides et d'Acoustique (\textsc{LMFA}) au côté de Louis Gostiaux\footnote{http://www.louis.gostiaux.fr/} et d'Emmanuel Lévêque\footnote{http://perso.ens-lyon.fr/emmanuel.leveque/} durant lequel nous avons étudié la génération d'ondes internes spiralantes en utilisant la méthode des réseaux de Boltzmann. \\ 
Ce stage, en plus d'être ma première expérience en recherche, fut l'occasion pour moi de découvrir des outils majeurs dans le monde de la recherche comme Python, le langage \LaTeX $ $ ou encore le système d'exploitation Unix. En outre, j'eus l'occasion de lire plusieurs articles et d'assister à diverses conférences au LMFA sur des sujets orientés mécanique des fluides enrichissant cette expérience. \\ 
Pour mener à bien ce projet, je me suis familiarisé avec la base données \textsc{HDF5}, la technique d'Embedding en faisant appel à des fragments de codes Python au sein d'un code écrit en C++, à la traduction du \textsc{HDF5} en xmf pour un rendu 3D de nos résultats de simulations, ainsi que de logiciels comme paraview ou hdfview entre autres éléments.\\
 Le stage s'étant parfaitement déroulé notamment grâce à un encadrement de qualité, et les résultats de nos simulations étant jugés intéressants, nous avons évoqué une possible publication de nos travaux, que nous finalisons en ce moment. \\
Actuellement, j'effectue un stage comptant à la fois pour le stage de fin d'étude de l'\textsc{EPUL} et pour le stage de recherche du Master \textsc{MEGA} traitant de la dispersion de gaz lourds émis par une source instationnaire. Cette étude se fera majoritairement grâce à la méthode des réseaux de Boltzmann et précédera une thèse expérimentale sur la même problématique. \\
Ce stage est pour moi l'occasion de d'approfondir mes connaissances sur la technique des réseaux de Boltzmann et de me familiariser avec les problématiques liées à la dispersion atmosphérique de gaz lourds.  \\
 
Je souhaiterais aujourd'hui candidater à l'offre de thèse "simulation numérique instationnaire de dispersion atmosphérique
d’un gaz lourd par la méthode Boltzmann sur réseau ; étude des fluctuations extrêmes de concentration" car elle serait dans la suite directe de mon stage actuel.\\
 Je suis de plus persuadé d'avoir les bases requises et la motivation nécessaire pour mener à bien ce projet et exploiter au maximum les différentes méthodes numériques autant dans leur implémentation que dans l'exploitation et l'analyse des données.
\\[2mm] 

\raggedleft \small
Dans l'attente de votre réponse, \\
Veuillez accepter mes plus sincères salutations \\
Nathaniel Saura

\end{document}
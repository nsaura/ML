\documentclass[french, lmr]{article}
\usepackage[utf8x]{inputenc}
\usepackage[T1]{fontenc}
\usepackage[french]{babel}
\usepackage{amsmath}
\usepackage{graphicx}
\usepackage{geometry}
\usepackage{enumitem}
\usepackage{array}    %% Pour les tableaux
\geometry{hmargin=2.5cm, vmargin = 2.3cm}
%%% Personnalisation des en-têtes et pieds de pages
\usepackage{fancyhdr}
\pagestyle{fancy}

%\renewcommand{\footrulewidth}{1pt} %%% Une barre en haut
\renewcommand{\headrulewidth}{10pt} %%% Une barre en haut
\fancyfoot[L]{\leftmark}
\fancyfoot[C]{\thepage}

\usepackage[svgnames]{xcolor}
\xdefinecolor{brick}{named}{DarkRed}
\xdefinecolor{navy}{named}{Navy}

\newcommand{\brick}{\color{brick}}
\newcommand{\navy}{\color{navy}}

\usepackage{scrbase}
\newcommand{\pathpic}{/home/saura/Documents/Latex_files/Pic/}

\usepackage[linkcolor=Chocolate,colorlinks=true]{hyperref}

\newcommand{\scidatalogo}{\includegraphics[scale=0.5]{\pathpic lmllogo.png}}
\pagestyle{headings}

\renewcommand{\headrulewidth}{1pt}
\setlength{\headheight}{15pt} 
\lhead{\textsc{\scidatalogo}}
%\rhead{\textsc{\overleaflogo}}

\pagenumbering{gobble} % To remove page counting
\thispagestyle{fancy}
\begin{document}
\begin{minipage}{455 pt}
\flushright{
\small\navy \textit{Nathaniel \textsc{saura}} \\
50 Bis Rue Saint Antoine \\
Lyon 69003, France \\
(+33) 7 82 47 09 35 \\
2 Mai 2017, \\[3mm]
}
\end{minipage}

\normalsize \color{Black}
\fontfamily{lmr}{À l'intention de Mesdames et Messieurs du Jury, } \\[4 mm]
\hspace{4 mm }Futur diplômé d'un Master 2 recherche en mécanique des fluides et énergétique (École Doctorale \textsc{MEGA}) ainsi que d'un diplôme d'ingénieur mécanique et modélisation (École Polytechnique Universitaire de Lyon \textit{\textsc{EPUL}}), je souhaiterais me spécialiser dans la recherche en mécanique des fluides et plus particulièrement dans l'exploitation de méthodes algorithmiques de pointe pour le développement de modèles de couches limites turbulentes innovants. 

\hspace{4 mm }J'ai effectué un stage de M1 au Laboratoire de Mécanique des Fluides et d'Acoustique (\textsc{LMFA}) au côté de Louis Gostiaux\footnote{http://www.louis.gostiaux.fr/} et d'Emmanuel Lévêque\footnote{http://perso.ens-lyon.fr/emmanuel.leveque/} durant lequel nous avons étudié la génération d'ondes internes spiralantes en utilisant la méthode des \textbf{réseaux de Boltzmann} (LBM pour Lattice Boltzmann Method). \\ 
Ce stage, en plus d'être ma première expérience dans le monde de la recherche, fut l'occasion pour moi de découvrir des outils majeurs de ce domaine comme Python, le langage \LaTeX $ $ ou encore le système d'exploitation Unix. En outre, j'eus l'occasion de lire plusieurs articles et d'assister à diverses conférences au LMFA sur des sujets orientés mécanique des fluides enrichissant cette expérience. \\ 
Pour mener à bien ce stage, je me suis familiarisé avec la base de données \textsc{HDF5}, la technique d'Embedding en faisant appel à des fragments de codes Python au sein d'un code écrit en C++, à la traduction du \textsc{HDF5} en xmf pour un rendu 3D de nos résultats de simulations, ainsi que de logiciels comme Paraview ou HDFview entre autres éléments.\\
 Le stage s'étant parfaitement déroulé notamment grâce à un encadrement de qualité, et les résultats de nos simulations étant jugés intéressants, nous avons évoqué une possible publication de nos travaux, que nous finalisons en ce moment. \\
Actuellement, j'effectue un stage comptant à la fois pour le stage de fin d'étude de l'\textsc{EPUL} et pour le stage de recherche du Master \textsc{MEGA} traitant de la dispersion de gaz lourds (par lourds nous entendons des gaz dont les effets de flottabilités sont pris en compte) émis par une source instationnaire. Cette étude se fera majoritairement grâce à la méthode des \textbf{réseaux de Boltzmann} et précédera une thèse expérimentale sur la même problématique. \\
Entre ces deux stages de recherche, j'ai eu l'occasion d'effectuer un stage en entreprise à l'étranger durant l'été 2016. Durant ce stage, j'ai pu comparer les mondes de la recherche et celui de l'entreprise, confirmant que la recherche de par son universalité, tant par les disciplines, sa versatilité que par sa portée, me correspondait plus largement.  \\

Je suis fortement intéressé par l'idée d'incorporer des concepts algorithmiques dans l'étude de la Turbulence et particulièrement de la couche limite turbulente, dont une meilleure compréhension aurait des impacts technologiques, économiques et écologiques, en plus du défi intellectuel que cela représente. Et, au vue de la quantité de données issue d'expériences en laboratoire, \textit{in situ}, ou de simulations numériques, je suis convaincu que la révolution du \textit{machine learning} jouera un rôle déterminant dans l'exploitation optimale des travaux sus-mentionnés.\\
C'est donc fort de mon expérience et sûr de l'orientation que je veux donner à mon avenir, que je candidate à l'offre de thèse intitulée "Méthode pilotée par les données pour le développement de modèle de turbulence pour le contrôle de couche limite turbulente" dirigée par Mr. Thomas \textsc{GOMEZ}. \\
Ayant contacté Mr. Gomez, je suis convaincu que travailler sur une telle problématique au sein du Laboratoire de Mécanique de Lille à ses côtés sera pour moi une façon de m'épanouir intellectuellement, professionnellement et personnellement. 
\\[2mm] 

\raggedleft \small
Dans l'attente de votre réponse, \\
Veuillez accepter mes plus sincères salutations \\
Nathaniel Saura

\end{document}
\documentclass[a4paper,10pt]{article}
\usepackage[utf8x]{inputenc}
%\usepackage[french]{babel}
\usepackage{amsmath}
\usepackage{amsfonts}
\usepackage{amssymb}
\usepackage{graphicx}		 			% Inclusion des figures 
\usepackage{textcomp}
\usepackage[nointegrals]{wasysym}			% Collection de symboles mathématiques
\usepackage{multicol}					% Pour utiliser \hfill
\usepackage{ifthen}
\usepackage{tabularx}	 				% Gestion avancée des tableaux
%\usepackage[squaren]{SIunits}
%\usepackage{sistyle} 					% Mise en forme des unités
%\usepackage{fancyhdr} 					% Entêtes et pieds de pages personnalisés 
%\usepackage{latexsym}
%\usepackage{xcolor}					% Pour redéfinir des couleurs 
\usepackage[svgnames]{xcolor}				% Voir : http://calque.pagesperso-orange.fr/latex/latexps.html
\usepackage{latexsym}
\usepackage{fontenc}
\usepackage{wrapfig} 					% Pour pouvoir placer une image à côté de texte
\usepackage{cite}	
\usepackage{url} 					% Pour citer les sites internet dans la bilibographies
\usepackage{setspace}
\usepackage[T1]{fontenc}				% Indispendable, présent dans tous les codes exemples
\usepackage[linkcolor=red,colorlinks=true]{hyperref} 	% Hyper ref
\usepackage{listings}					% Pour citer du code
\usepackage[justification=centering]{caption}
\usepackage{subcaption}
\usepackage{subfloat}


\newcommand\pathpic{/home/saura/Documents/Latex_files/Pic/}
%\newcommand\spipath{/home/saura/Documents/Stage/Notes_Manip/Pic/Simu_pic/}

%%%%%%%%%% Raccourcis %%%%%%%%%%%
\newcommand{\lp}{\left(}
\newcommand{\rp}{\right)}
\newcommand{\rint}{\int^\infty_{-\infty}}
\newcommand{\dint}{\int^\infty_{0}}
\newcommand{\rsum}{\sum^\infty_{ j = -\infty}}
\newcommand{\cad}{c'est-à-dire }
\newcommand{\qcq}{quelconque}
\newcommand{\keps}{$k-\varepsilon$}
\newcommand{\eci}{énergie cinétique}

%%% Couleurs %%%
%%%%%%%%%%%%%%%%%%%
%%% Couleurs %%%
\xdefinecolor{purple}{named}{MediumVioletRed}
\xdefinecolor{brick}{named}{DarkRed}
\xdefinecolor{forest}{named}{DarkMagenta}
\xdefinecolor{dgreen}{named}{DarkOliveGreen}
\xdefinecolor{darker}{named}{SaddleBrown}
\xdefinecolor{brick}{named}{FireBrick}
\xdefinecolor{dred}{named}{DarkRed}
\xdefinecolor{indigo}{named}{Indigo}

\newcommand{\brick}{\color{brick}}
\newcommand{\npurple}{\color{forest}}
\newcommand{\dgreen}{\color{dgreen}}
%\newcommand{\sbrown}{\color{sbrown}}
\newcommand{\brik}{\color{brick}}
\newcommand{\indigo}{\color{indigo}}
\newcommand{\dred}{\color{dred}}
\newcommand\darker{\color{Maroon}}

\newcommand\black{\color{black}}
\newcommand{\bl}{\color{blue}}
\newcommand\red{\color{red}}
%%%%%%%%%%%%%%%%%%%
%% Pour la légende des figures 
\usepackage{caption}
\usepackage[labelfont={color = Indigo, bf}, font = it ,justification=centering]{caption}

%%%%%%%%%%%%%%%%%%%

%%% Éléments pour la première page %%%
\newcommand{\reporttitle}{Écoulement turbulent en canal plan : comparaison RANS et DNS }     % Titre
\newcommand{\reportauthor}{\darker \textbf{Saura Nathaniel} - 5A Polytech Mécanique - Master MEGA } % Auteur
\newcommand{\reportsubject}{\bl \indigo \textbf{Turbulence dans les écoulements}} % Sujet
\newcommand{\HRule}{\rule{\linewidth}{0.5mm}}
%%%%%%%%%%%%%%%%%%%
\hypersetup{
    pdftitle={\reporttitle},%
    pdfauthor={\reportauthor},%
    pdfsubject={\reportsubject},%
    pdfkeywords={rapport} {vos} {mots} {clés}
}

\usepackage{geometry}
\geometry{hmargin=1.65cm, vmargin=1.5cm}

\begin{document}
\begin{titlepage}

\begin{center}

\begin{minipage}[t]{0.9\textwidth}
  \begin{flushleft}
    \includegraphics[width=45mm]{\pathpic polytech.jpg} \hfill
%    \includegraphics[width=25mm]{\pathpicNS centrale.png} \hfill
    \includegraphics[scale = 0.55]{\pathpic mega.jpg}
    \begin{spacing}{1.5}
    \end{spacing}
  \end{flushleft}
\end{minipage}
% \begin{minipage}[t]{0.30\textwidth}
%   \begin{center}
%     \includegraphics[width=15mm]{\pathpicNS centrale.png} \\[0.5cm]
%     \begin{spacing}{1.5}
%     \end{spacing}
%   \end{center}
% \end{minipage}
% 
% \begin{minipage}[t]{0.30\textwidth}
%   \begin{flushright}
%     \includegraphics[width=30mm]{\pathpicNS centrale_innov.png} 
%   \end{flushright}	
% \end{minipage} 
\vspace{2cm}

    \emph{ \reportauthor}\\[1cm]
    \large{Filière Mécanique \\
    Année 2016-2017}
    
\vspace{1.5cm}
\textsc{\Large \reportsubject}\\[0.4cm]
\brik \HRule \\[0.4cm]
{\huge \bfseries \reporttitle}\\[0.4cm]
\HRule \\[1.5cm]

\black
%     \reportauthor
\vspace{2cm}

    \emph{\large Enseignant :} \\[5mm]
    Mr.~\textsc{\large \darker Faouzi}

\vfill
\begin{figure}[h!]
  \begin{center}
  \includegraphics[width=6cm]{\pathpic univ_lyon.png} \\[0.5cm]
  \end{center}
  \end{figure}

\end{center}

\end{titlepage}

Trois approches sont possibles lorsque l'on veut étudier un problème dans lequel apparaît le phénomène de turbulence. La première, la simulation directe ou DNS, résout toutes les échelles. Ces simulations numériques sont très lourdes et nécessitent beaucoup de temps de calcul. De plus, plus le Reynolds augmente, plus le temps de calcul est conséquent. Cette exigence a poussé les intéressés à imaginer des méthodes qui prennent en compte le phénomène de turbulence mais dont le temps de calcul reste raisonnable.\\
Deux catégories de méthodes ont été alors mises au point : la méthode des LES pour \textit{Large Eddy Simulations} et la méthode RANS pour \textit{Reynolds-Averaged Navier Stokes}.\\
La première consiste en la résolution des plus grandes échelles exigeant alors un maillage moins précis, les équations que l'on cherche à résoudre sont les équations de Navier-Stokes filtrées pour ne faire apparaître que les grandes structures. Comme dans le cas de la DNS, la résolution se fait dans tout l'espace. Cette méthode reste encore lourde et très peu utilisée dans le monde industriel.\\
Enfin la méthode RANS est la méthode la moins chronophage et donc la plus utilisée dans le monde de l'ingénieur.\\
Elle est basée sur la décomposition de Reynolds des champs $\vec{U} $, $P$ et d'autres (selon le problème étudié : température etc...) qui s'écrit pour un champ $\chi$ quelconque :
\begin{equation}
\vec{\chi} = \vec{\overline{\chi}} + \vec{\tilde{\chi}} 
\end{equation}
On injecte alors ces champs décomposés dans l'équation de Navier-Stokes (ou de convection-diffusion dans un problème de thermique par exemple) et on moyenne les équations ainsi obtenues. Cette méthode et les différents modèles qui la composent sont également basés sur l'hypothèse de Boussinesq permettant d'écrire les tensions de Reynolds moyennes $\overline{u_iu_j}$ telles que :
\begin{equation}
\overline{u_iu_j} = -\nu_t \lp \frac{\partial U_i}{\partial x_j} + \frac{\partial U_j}{\partial x_i} \rp + \frac{2}{3} k \delta_{ij}  \label{nut}
\end{equation}  
Les différents modèles de cette méthode se distinguent notamment dans la façon dont cette viscosité turbulente est calculée. Notons également que $\nu_t$ est supposée isotrope.

\noindent Les modèles RANS donnent accès à des \textbf{grandeurs moyennées} fournissant ainsi le comportement \textbf{moyenné dans le temps du fluide étudié}, ainsi que les ordres de grandeurs des fluctuations turbulentes. On a donc ici aussi une sorte de filtrage (statistique ici) des résultats.\\
Cette méthode ainsi que les différents modèles développés font apparaître des constantes qui ne pourront être évaluées que par l'expérience ou la DNS (voir équation \ref{keps_constantes}).\\
 Dans ce BE, on propose de comparer les différents modèles RANS avec des résultats plus précis issus de DNS pour un problème à géométrie simple : un écoulement dans un canal plan (maillage figure (\ref{maillage})). Nous discuterons également des modèles de traitement des parois.\\
 Nous essaierons d'introduire même brièvement les différents modèles RANS ainsi que quelques unes de leur propriétés et nous tenterons de retrouver ces caractéristiques afin de discuter de la similitude ou des écarts entre les résultats de ces modèles et ceux de la DNS.

\begin{figure} [ht!]
\centering
%\includegraphics[height = 4 cm, width = 7cm]{\pathpic maillage.png}
\includegraphics[scale=0.4]{\pathpic maillage.png}
\caption{}
\label{maillage}
\end{figure}

\vspace{2cm}
Pour traiter cet exercice j'ai essayer de détailler les résultats des modèles et d'interpréter les différences entre les modèles, lorsque cela m'était possible.
\pagebreak
 
 \brik
  \section{Modèle $k-\varepsilon$} \black

Le premier modèle que nous étudions est le modèle \keps . Il consiste à prendre en compte une viscosité turbulente calculée à partir de l'hypothèse de Boussinesq (\ref{nut}) et de deux équations de transports : une en $k$, l'énergie cinétique turbulente, et une en $\varepsilon$ \cad le taux de dissipation de cette énergie cinétique. Enfin dans ce modèle nous relions les	 grandeurs $\nu_t$, $k$ et $\varepsilon$ à partir d'une analyse dimensionnelle de la cascade d'énergie de Kolmogorov :
\begin{equation}
\varepsilon \sim \frac{k^{3/2}}{L_t} \label{kolmo}
\end{equation} 
Et d'autres analyses $\nu_t \sim uL_t$ et $u \sim k^{1/2}$, on trouve alors :  
\begin{equation}
 \mu_t = \rho C_\mu \frac{k^2}{\varepsilon} \text{\hspace{2 mm} avec $C_\mu$ constante sauf en présence de paroi}\label{mut}
\end{equation}

\noindent Nous allons observer différentes variantes du modèle \keps $ $ le standard, le "RNG" et le "Realizable". Si le modèle standard est "le plus simple" les autres sont des raffinements de ce dernier et nous verrons en quoi ils améliorent la précision de ces résultats. Nous commençons donc par le modèle standard avec deux lois de paroi : la normale et l'améliorée :

 \npurple
 
 \subsection{ \keps $ $ standard avec lois de paroi standard et améliorée} \black
Les équations de $k$ et $\varepsilon$ s'écrivent pour ce modèle standard : \\
\indent Pour $k$ :
\begin{equation}
\frac{D\left( \rho k \right)}{Dt} = \frac{\partial }{\partial x_j} \left[ \left( \mu + \frac{\mu_t}{\sigma_k} \right) \frac{\partial k}{\partial x_j} \right] + G_k -\rho \varepsilon \label{k_keps}
\end{equation}

Pour $\varepsilon$ :
\begin{equation}
\frac{D\left( \rho \varepsilon \right)}{Dt} = \frac{\partial }{\partial x_j} \left[ \left( \mu + \frac{\mu_t}{\sigma_\varepsilon} \right) \frac{\partial \varepsilon}{\partial x_j} \right] + C_{\varepsilon 1} \frac{\varepsilon}{k} G_k -\rho C_{\varepsilon 2} \frac{\varepsilon^2}{k} \label{eps_keps}
\end{equation}

Avec les constantes :
\begin{equation}
C_\mu = 0.09 \text{\hspace{2 mm}} C_{\varepsilon 1} = 1.44 \text{\hspace{2 mm}} C_{\varepsilon 2} = 1.92 \text{\hspace{2 mm}} \sigma_k = 1.0 \text{\hspace{2 mm}} \sigma_\varepsilon = 1.3 \label{keps_constantes}
\end{equation}

\noindent C'est le modèle qui fait intervenir le plus de constante. \\ 
\indent La prise en compte des effets de parois est très importante. En effet, les grandeurs mises en jeu proche de la paroi varient rapidement et d'une façon parfois drastique. Pour s'assurer que ces effets soient pris en jeu, il faut au préalable raffiner le maillage proche des parois. Puis lors de la configuration du calcul, il faudra préciser que l'on veut telle ou telle loi de paroi selon les résultats souhaités. Dans le modèle de loi paroi standard, le logiciel résout les équations RANS plus celle de fermeture en suivant les lois universelles proche de la paroi alors que le modèle amélioré diffère plusieurs couches proche de la paroi : les \textit{buffer layer} et \textit{sublayer}, puis  l'\textit{inner layer }et enfin l'\textit{outer layer}.\\
Ces zones sont différenciées selon la valeur du $y^+ = \frac{yu_f}{\nu}$ avec $u_f = \sqrt{\frac{\tau_p}{\rho}}$ la vitesse de frottement et $\tau_p$ la contrainte à la paroi. Le maillage devra être au préalable capable de prendre en compte le raffinement dans la sous couche visqueuse (\textit{viscous sublayer}) mais également le raffinement (moins précis) dans l'\textit{inner layer}, zone de transition entre la sous couche et le reste de l'écoulement.\\
Voici le tracé de la vitesse horizontale $U_x$ dans le cas \keps $ $ pour les deux lois de paroi :
 
 
 \begin{figure}[ht!]
 \centering
 \includegraphics[scale = 0.4]{\pathpic k_e_standardWF_enh_U.png} 
 \caption{ \footnotesize{ Comparaison des résultats pour la vitesse horizontale $U_x$ pour le modèle \keps $ $ avec les lois de paroi standard et améliorée et les résultat de la DNS. Les vitesses sont tracées selon l'axe des $y$. Les deux lois approchent assez bien la courbe de la DNS, même si la loi améliorée donne un résultat plus précis.}}
 \label{std_U_x_std_enh}
 \end{figure}
 \black
Dans la figure (\ref{std_U_x_std_enh}), nous confirmons que pour la vitesse $U_x$, la loi de paroi améliorée approche mieux la réalité. Le modèle amélioré calcule avec plus de précisions les variations de la vitesse à la paroi, appréhendant donc mieux les variations de vitesse dans l'\textit{inner} et donc dans l'\textit{outer layer}. Toutefois, la courbe de vitesse obtenue dans le modèle \keps $ $ standard avec de loi de paroi standard n'est pas très éloignée de la véritable vitesse $U_x$ mais la zone de transition des régimes proche et loin de la paroi n'est pas très bien gérée amenant des erreurs dès $y \approx 30 \ m$. 
Au vue de ce graphe, nous sommes intéressés à tracer la courbe $\frac{dU_x}{dy}$ que nous dénommerons taux de cisaillement (également gradient de $U$). \\
Dans la figure (\ref{std_grad_std_enh}), nous avons tracé l'évolution du gradient de la paroi jusqu'à l'extérieur pour le modèle \keps $ $ standard en différenciant les deux modèles de parois. Pour comparer ces résultats, nous traçons la même courbe obtenue grâce à une DNS. \\
 Lorsque les calculs prennent en compte la \textit{sublayer} et l'\textit{inner layer}, les résultats sont très précis. Dans une loi de paroi standard, les résultats sont très proches néanmoins comme pour $U_x$, nous notons quelques erreurs par rapport à la DNS. De plus dans la zone $y < 30\ m$, l'erreur de la courbe blanche (modèle standard - loi de paroi standard) est la plus grande, nous comprenons donc un peu mieux d'où venait l'erreur sur la vitesse, comme discutée dans la figure (\ref{std_U_x_std_enh}).

\begin{figure}[ht!]
\centering
\includegraphics[scale=0.4]{\pathpic k_e_std_enh_gradu.png}
\caption{\footnotesize{Le modèle \keps$ $ standard avec une loi de paroi standard (blanc) nous permet d'avoir une bonne approximation de ce gradient. Le modèle avec la loi améliorée, l'approximation est meilleure, et l'erreur est presque nulle. On voit que l'erreur relevé figure (\ref{std_U_x_std_enh})}.} 
\label{std_grad_std_enh}
\end{figure}

\noindent Nous continuons notre analyse du modèle standard en traçant les grandeurs turbulentes. \\

Nous identifions dans l'expression de l'évolution de l'énergie cinétique $k$, équation (\ref{k_keps}), le terme de production classique de la méthode RANS $-\rho \overline{u_i u_j} \frac{\partial \overline{u_i}}{\partial x_j}$ avec $G_k$, couplé dans l'équation de $\varepsilon$ (\ref{eps_keps}). De plus tous les termes que l'on retrouve dans les équations classiques RANS faisant intervenir les tensions de Reynolds ont été remplacés par des constantes expérimentales, que ce soit pour $k$ ou $\varepsilon$; Nous allons voir quelles sont les limites de ce modèle au travers plusieurs tracés, nous commençons avec $k$ : 
\begin{figure}[ht!]
\centering
\includegraphics[scale = 0.4]{\pathpic std_enh_K.png}
\caption{Tracés des évolutions de $k$ pour les modèles de paroi standard et amélioré, comparés avec les résultats de la DNS (blanche). Le modèle de parois standard donne des résultats cohérents mais surestimé. Les erreurs sont moins significatives avec le modèle de paroi amélioré (verte).}
\label{std_enh_k}
\end{figure}

Traçons ensuite le taux de dissipation $\varepsilon$ 

\pagebreak

\begin{figure}[ht!]
\centering
\includegraphics[scale=0.4]{\pathpic k_e_std_enh_eps.png}
\caption{\footnotesize{Proche de la paroi (pour de petit $y$), l'approximation de la loi standard n'est pas bonne et le taux de dissipation est surévalué. Pour la loi améliorée, le taux de dissipation est légèrement sous évalué puis surévalué. Passés les $y \approx 50\ m$, les deux courbes convergent vers la courbe de DNS (vert).}}
\label{std_enh_eps}
\end{figure}


Traçons ensuite la production de $k$ :

\begin{figure}[ht!]
\centering
\includegraphics[scale=0.4]{\pathpic std_enh_pk.png}
\caption{Tracé de l'évolution de la production de $k$ le long de l'axe $y$. Le modèle amélioré de paroi donne de résultats très proches de la DNS. Les résultats du modèle de paroi standard sont bons mais à partir de $y \approx 50 \ m$.}
\label{std_enh_P_k}
\end{figure}

Enfin nous traçons la viscosité turbulente. Nous rappelons que l'équation de la viscosité dans ce modèle est $\nu_t = C_\mu \frac{K^2}{\varepsilon} $

\begin{figure}[ht!]
\centering
\includegraphics[scale=0.45]{\pathpic std_enh_nut.png}
\caption{Tracés de la viscosité turbulente pour les deux modèles de paroi comparés au profil de viscosité donné par la DNS.}
\label{}
\end{figure}
Le modèle \keps $ $ ne donne pas une très bonne approximation de la viscosité turbulente même en loi de paroi améliorée. Nous pouvons donner une piste d'explication sur ces écarts de valeurs par le fait que dans le calcul de $\nu_t$ dans ce modèle, Fluent cumule les erreurs faites sur le carré de $k$, sur $\varepsilon$ et sur la constante $C_\mu$. Ainsi, même si le logiciel prend en compte une \textit{sublayer}, les erreurs seront toujours présentes.

\pagebreak

\dgreen \subsubsection*{Conclusions sur le modèle } \black
Ce qui ressort des trois figures précédentes (\ref{std_enh_k}), (\ref{std_enh_eps}) et (\ref{std_enh_P_k}), est que le modèle de loi de paroi standard (pour le \keps) n'arrive pas à donner des valeurs précises par rapport à celles de la DNS, l'écart est assez grand aux alentours des petits $y$, puis il rattrape la courbe de la DNS. Toutefois, ce modèle standard avec une loi de paroi standard peut être utilisé en première approximation, puisqu'il donne l'ordre de grandeurs des différentes grandeurs étudiées. La loi de paroi améliorée donne des valeurs plus proches et plus satisfaisantes. \\
Ces différences s'expliquent par le fait qu'elle prend en compte un raffinement du maillage, et donc contient plus d'informations aux niveaux des points où les différentes grandeurs ou leurs dérivées varient abruptement. \\

L'approche du modèle \keps $ $ est justifiée lorsque nous prenons la loi améliorée de paroi, et même au niveau du temps de calcul, il n'y a que très peu de différences entre les deux lois. L'inconvénient majeur est qu'il ne permet pas d'avoir accès aux stress de Reynolds qui peuvent être des quantités intéressantes selon ce que l'on cherche à analyser. 

La courbe de ce modèle avec la loi de paroi améliorée donnant de très bonnes approximations en règle général, nous allons l'utiliser pour donner un ordre de grandeur de la précision des prochains modèles, en plus de la comparaison avec la DNS. Nous continuons notre analyse des modèle avec le test des modèles RNG et REA pour ce problème. 

\npurple
\subsection{Modèles RNG et REAlizable} \black
Ces modèles cherchent toujours à résoudre les équations RANS en utilisant des équations différentes pour l'évolution de l'\eci $ $ $k$ et de son taux de dissipation $\varepsilon$, mais dans ces deux modèles, on ajoute des calculs supplémentaires pour réduire le nombre de constantes expérimentales présentes dans le modèle \keps $ $ standard en considérant d'autres équations.\\
 Dans le cas du modèle Realizable (que nous abrégerons dans la suite par REA), l'équation du taux $\varepsilon$ est déduite de l'équation de l'enstrophy qui est l'équivalent du tenseur de Reynolds du point de vue de la vorticité $\omega$. Ces deux quantités sont reliées par la viscosité (ici turbulente dans le modèle \keps) par la formule 
 \begin{equation*}
  \varepsilon = \nu_t \overline{\omega^2}
  \end{equation*} 
On s'affranchit ainsi des constantes $\sigma_\epsilon$, $C_{1 \varepsilon}$ et $C_{2 \varepsilon}$. Ce modèle fait également varier la constante $C_\mu$ dans l'équation de la viscosité turbulente (\ref{mut}) en l'ajustant par rapport au taux de cisaillement ($\frac{dU_x}{dy}$ ici) et du taux de rotation $\omega$. \\

\noindent En ce qui concerne le modèle RNG, toutes les constantes du modèle \keps $ $ standard sont calculées à partir de la théorie des groupes (utilisant les propriétés d'invariances par changement de base galiléennes). Ce modèle devrait être très précis. \\
Pour ces deux nouveaux modèles, nous ne comparerons les résultats que pour un modèle amélioré de paroi. \\

On commence à analyser les résultats de ces méthodes avec le tracé de la vitesse $U_x$ calculé par ces deux modèles :
 
 \begin{figure}[ht!]
 \centering
 \includegraphics[scale=0.5]{\pathpic stdENH_REA_RNG_U.png}
 \caption{\footnotesize{Les modèles RNG et REA donnent de très bonnes approximations pour la vitesse, en paroi améliorée. Les courbes sont très proches de la bleue, le modèle REA donne ici des résultats légèrement plus précis. Figure également dans ce graphique, la courbe (blanche) de vitesse obtenue avec le modèle de paroi amélioré pour le modèle \keps $ $ standard.}}
 \label{RNG_REA_Ux}
 \end{figure}
Les vitesses numériques obtenues grâce aux modèles REA et RNG sont très proches de celle calculée par la DNS, légèrement plus proches que celles données par la modèle \keps $ $ standard  (loi améliorée). Notons néanmoins une erreur légèrement plus grande dans la zone de transition du profil de vitesse, aux alentours des $y = 50 \ m$.  \\

\noindent On veut également voir si les modèles REA et RNG permettent d'avoir accès précisément au taux de cisaillement $\frac{dU_x}{dy}$.

\begin{figure}[ht!]
\centering
\includegraphics[scale=0.5]{\pathpic stdENH_REA_RNG_gradu.png}
\caption{Tracé du gradient de vitesse $\frac{dU_x}{dy}$ pour les modèles RNG et REA comparé avec la simulation du gradient par la DNS et par le modèle standard en loi de paroi améliorée. Les courbes sont très proches et se superposent. Il semblerait quand même que le modèle REA \textit{fite} mieux la courbe de la DNS (blanche).}
\label{REA_RNG_gradu}
\end{figure}

 \noindent Comme on s'y attendait au vue de la figure (\ref{RNG_REA_Ux}), le gradient est parfaitement approximé par les deux modèles. Nous voyons à certaines position de $y$ que le modèle REA est très légèrement plus précis que les deux autres modèles représentés sur la figure (\ref{REA_RNG_gradu}). \\
 Au vue des deux figures précédentes, il semblerait que ces deux modèles soient légèrement plus précis que le modèle \keps $ $ standard en loi de paroi améliorée. \\
On analyse à présent les quantités turbulentes et nous commençons avec la courbe de l'évolution de l'énergie cinétique $k$ pour les deux modèles (plus le modèle standard à titre de comparaison).
 \begin{figure}[ht!]
\centering
\includegraphics[scale=0.5]{\pathpic stdENH_REA_RNG_k.png}
\caption{}
\label{REA_RNG_k}
\end{figure}

Proche du pic d'énergie cinétique ($y \approx 50\ m$), les deux modèles RNG et REA donnent une meilleure approximation que le modèle standard et notamment pour les $y > 300 \ m$, le modèle REA est légèrement plus précis que le RNG et le \keps $ $ en modèle de paroi amélioré. Remarquons également que pour les $y > 100 \ m$, le modèle \keps $ $ est plus proche de la courbe de DNS que ne l'est le modèle RNG. \\

\pagebreak

On continue notre étude avec le tracé du taux de dissipation $\varepsilon$ de $k$ :
\begin{figure}[ht!]
\centering
\includegraphics[scale=0.42]{\pathpic stdENH_RNG_REA_eps.png}
\caption{}
\label{REA_RNG_eps}
\end{figure}

Ce graphique est la parfaite illustration qu'il manque des informations pour approximer la valeur du taux de dissipation proche de la parois. Les deux modèles donnent des courbes très proches et il est difficile de dire qui des deux est physiquement plus significatif. De plus, ils sont légèrement plus précis que le modèle standard, surtout dans l'intervalle $\left] 0, 70 \right[$. Les informations manquantes ici pourraient être des fluctuations des tensions Reynolds qui n'ont pas été prise en compte dans le calcul du taux de dissipation . \\
Nous continuons notre analyse. Le graphe suivant est celui de la production d'énergie $k$ :

\begin{figure} [ht!]
\centering 
\includegraphics[scale=0.4]{\pathpic stdENH_REA_RNG_Pk.png}
\caption{\footnotesize{Tracés des courbes de production d'énergie cinétique pour les modèles REA, RNG et standard en loi de paroi amélioré. Ces courbes sont à comparer avec la courbe de la DNS (blanche)}}
\label{REA_RNG_Pk}
\end{figure}

La production de $k$ parait être calculée avec beaucoup de justesse pour les modèles en cours avec une légère finesse supplémentaire pour le modèle REA. \\
Si ces deux modèles donnent de très bonnes approximations des grandeurs regardées, il reste quand même très proche des résultats du modèle \keps $ $ en loi paroi améliorée. La seule grandeur qui n'avait pas été très bien \textit{fitté} par le modèle \keps $ $ standard était la viscosité turbulente. Traçons $\nu_t$ :

\begin{figure}[ht!]
\centering
\includegraphics[scale=0.4]{\pathpic stdENH_REA_RNG_nut.png}
\caption{\footnotesize{Tracés des courbes de viscosité turbulentes pour les différents modèles évoqués et la DNS.}}
\label{stdENH_REA_RNG_nut}
\end{figure}

La figure précédente est très intéressante pour plusieurs raisons. Premièrement elle nous permet de discuter de la différence entre les deux modèles dans leurs constructions même : le modèle REA ajuste la constante $C_\mu$ en fonction des taux de cisaillement et de la possible rotation présente dans l'écoulement. Or lorsque le gradient tend vers zéro (\cad pour $y > 400 \ m$ voir figure (\ref{REA_RNG_gradu})), il semblerait que le calcul de cette constante soit mal géré. Et c'est justement dans cette configuration que le modèle RNG gagne en fiabilité par rapport au REA, puisqu'il ne prend aucune constante telle qu'elle, s'affranchissant donc d'éventuels problèmes de variation de ces constantes. Le même argument sur la constante $C_\mu$ s'applique à la comparaison entre le RNG et le \keps $ $ considéré ici sauf qu'on peut l'étendre aux autres constantes présentes dans le modèle \keps. \\

\noindent Nous allons à présent discuter brièvement de ces différents modèles du point de vue industriel \cad analyser quel(s) modèle(s) donn(ent) un meilleur ratio précision / temps de calcul.\\
Le modèles \keps $ $ avec une loi de parois standard a nécessité 1 minute et 6 secondes de calcul pour 1250 itérations alors que l'amélioration du modèle de paroi réduit ces chiffres à 55 secondes de calculs et 850 itérations. La différence de précision était très remarquable à tous les niveaux.\\
Venaient ensuite les modèles RNG et REA avec une loi de paroi améliorée. Le modèle RNG a calculé pendant 43 secondes et a convergé au bout de 780 itérations. Le modèle REA augmente ces chiffres pour finir ses calculs au bout de 55 secondes et de 1050 itérations. Les deux étaient très proches vis-à-vis des courbes de la DNS. Ils étaient sensiblement plus précis que le modèle standard avec loi de paroi amélioré tant au niveau concret (meilleur fit) qu'au niveau physique se séparant de quelques voire toutes les constantes propres aux modèles \keps $ $ standard.\\
Ce qui ressort de cette analyse c'est qu'il semblerait que le modèle RNG soit celui qui aurait le meilleur ratio précision/temps de calcul.\\
 
Malgré tout, une ombre vient assombrir le tableau : ces modèles aussi précis soit-il ne permettent pas de calculer les tensions de Reynolds et simulent leur action au travers certaines constantes et d'équations de transports. Nous allons comparer ces résultats avec des modèles capables de calculer ces tensions : les RSM.
\brik
  \section{Les modèles RSM :} \black
  Les modèles RSM pour Reynolds Stress Model sont les modèles les plus physiques dans ce sens qu'ils prennent en compte les effets d'anisotropie de la viscosité turbulente, les effets des tensions de Reynolds moyennés entre autres choses. En effet, en plus des équations RANS, les RSM résolvent les équations d'évolutions des tensions de Reynolds moyens y incorporant par la même occasion des termes de diffusion turbulent et de transport visqueux pas forcément isotropes. Ces équations supplémentaires s'écrivent sous forme indicielle : 
  \begin{equation}
  \frac{\partial \rho \overline{u_i u_j}}{\partial t} + \frac{\partial }{\partial x_k} \left( \overline{U_k} \overline{u_i u_j}\right) = P_{ij} + F_{ij} + D^T_{ij} + \Phi _{ij} - \varepsilon _{ij}
  \end{equation}
Avec $P_{ij}$ la facteur traduisant le terme de production des stress de Reynolds par les gradients de vitesse moyenne notamment, $F_{ij}$ le terme de transport visqueux, $D^T_{ij}$ celui représentant la diffusion turbulente, $\Phi_{ij}$ la corrélation pression déformation et $\varepsilon_{ij}$ la dissipation. \\
Nous allons donc commencer l'analyse de la précision et de la justesse des résultats issus de ce modèle en considérant plusieurs loi de parois : dans un premier temps, nous allons analyser le RSM pour des lois de paroi standard et améliorées. Puis dans un second temps, nous ajouterons un terme d'\textit{echo} de paroi. Enfin, nous comparerons avec la simulation contenant le plus de conditions et qui a nécessité plus de 5 min de calcul : le RSM avec \textit{echo} de paroi et \textit{conditions aux limites à partir de K}. Nous discuterons de ces deux conditions lorsque nous les rencontrerons.

\npurple
\subsection{loi de paroi standard et améliorée} \black
Comme dans la première section, nous allons comparer les résultats du modèle RSM basique pour deux loi de parois. L'utilité de ces tests est de regarder l'impact d'un sous maillage proche de la paroi, pour ce modèle. La première courbe que nous traçons est la vitesse. Nous allons également comparer avec les résultats du modèle \keps $ $ standard pour les mêmes lois de parois. Les résultats sont disponibles sur la figure (\ref{stdENH_RSM_std_enh_Ux}).  \\
Nous constatons pour la vitesse $U_x$ que les lois de parois standards et améliorés pour le modèle RSM donnent des résultats moins précis que le modèle standard de \keps. \\

\begin{figure}[ht!]
\centering
\includegraphics[width = 13cm, height= 7.5cm]{\pathpic stdENH_RSM_std_enh_Ux.png}
\caption{Tracé des vitesses calculées avec le \keps $ $ et le RSM pour deux lois de paroi : la standard et l'améliorée. Les tracés des calculs du \keps $ $ nous servent à évaluer la précision du RSM.}
\label{stdENH_RSM_std_enh_Ux}
\end{figure}

\pagebreak

Nous allons donc rajouter la condition "écho de paroi" que nous appellerons par la suite écho. Nous traçons dans la figure (\ref{stdENH_RSM_std_echo_Ux}) l'évolution de la vitesse pour le modèle epsilon standard amélioré, le RSM standard qui donnait de meilleur résultat que l'amélioré (pour la vitesse) et le tracé de $U_x$ pour le RSM écho.

\begin{figure}[ht!]
\centering
\includegraphics[width = 13cm, height= 7.5cm]{\pathpic stdENH_RSM_std_echo_Ux.png}
\caption{Tracé des vitesses pour le RSM avec un modèle de paroi standard (plus précis pour la vitesse que pour le modèle du paroi amélioré voir (\ref{stdENH_RSM_std_enh_Ux}) comparé avec des résultats assez précis du \keps $ $ modèle amélioré de paroi. La valeur de référence étant la DNS (bleue).}
\label{stdENH_RSM_std_echo_Ux}
\end{figure}

Le modèle d'écho de paroi à l'air d'aider le RSM dans ces calculs et nous allons voir dans la suite que cette tendance se confirme. Nous allons essayer de comprendre ce qu'il y a derrière ce modèle et ce qu'il implique ou encore représente au niveau de la physique de l'écoulement.\\
Ce modèle de conditions aux limites prend en compte qu'une partie des fluctuations de pression est réfléchie sur la paroi  au lieu d'être absorbée par celle ci. Ce modèle ajoute donc un terme dans le tenseur des corrélations pression-déformations $\Phi_{ij}$. Pour le comprendre intuitivement, on peut utiliser la méthode des images et donc le terme supplémentaire dans $\Phi_{ij}$ pourrait s'écrire comme une combinaison linéaire des autres termes de ce tenseur. En somme, l'introduction de ces conditions aux limites permet d'étendre la gamme de phénomène physique simulé et donc d'avoir des résultats plus "physiques".\\
Une autre condition aux parois existe : les conditions aux limites à partir de $k$ l'énergie cinétique (nous abrégerons cette option par BCK).\\
Comparons le RSM avec et sans cette condition (les effet d'echo de parois sont activés).
  
  \pagebreak
  
  \begin{figure}[ht!]
  \centering
  \includegraphics[scale=0.5]{\pathpic RSM_echo_BC_Ux.png}
  \caption{Tracé des modèles RSM avec écho de paroi, avec ou sans raffinement du meshing pour $k$. Les résultats sont comparés avec la courbe de la DNS (Verte).}
  \label{RSM_echo_BC_U}
  \end{figure}
  
\noindent L'ajout de l'option BCK dans notre calcul Fluent fait doubler voire tripler le temps de calcul. En fait, cette option déclenche un raffinement de maillage pour le calcul de l'énergie cinétique turbulente $k$. Ce raffinement est très proche de celui effectué sur la vitesse par la loi de paroi amélioré et il est basé sur des modèles différents dans la \textit{sublayer}, l'\textit{inner layer} ou dans le reste de l'écoulement. La figure (\ref{RSM_echo_BC_U}) ne montre pourtant pas de changement sensible sur l'approximation de la vitesse moyenne $U_x$. Nous verrons des changements significatifs lorsque que nous considérerons des grandeurs turbulentes.\\
Les résultats obtenues pour la vitesse en écho ou BCK, sont très proches de ceux des modèles REA, RNG, ou \keps $ $ améliorée. On s'attend donc à ce que le gradient de cette vitesse soit bien approximé par le RSM.

\begin{figure}[ht!]
\centering
\includegraphics[scale=0.55]{\pathpic all_gradu.png}
\caption{Tracé de toutes les approximations du gradient de U calculer pour les différents modèles vus dans ce rapport. }
\label{all_gradu}
\end{figure}

\noindent Les RSM avec modèle de paroi standard et amélioré donnent des approximations plutôt bonnes si ce n'est quelques valeurs, les RSM avec modèles écho et BCK fittent parfaitement la courbe de la DNS. \\
Comme pour les autres modèles nous allons à présent étudier les approximations des grandeurs turbulentes par les RSM pour loi de paroi standard, amélioré, avec écho de paroi et raffinement du maillage pour $k$ et $u$.\\
 Dans chacun des cas, nous utiliserons les courbes de DNS comme références mais également les résultats obtenus avec le RNG ou le REA puisqu'ils étaient très bons.\\
 
 \pagebreak
 
 On trace alors les approximations de $k$.
 
\begin{figure}[ht!]
\centering
\includegraphics[height = 5.8 cm, width = 8.7 cm ]{\pathpic RSM_std_enh_k.png} \hfill 
\includegraphics[height = 5.8 cm, width = 8.7 cm]{\pathpic RSM_std_Echo_REA_k.png} 
 \caption{À gauche comparaison entre les RSM sans echo ni BCK et la DNS. Les résultats ne sont pas très précis en revanche le modèle de paroi standard semble être plus proche que l'amélioré. Nous le comparons (à droite) avec le modèle REA et le modèle écho.}
 \label{RSM_k}
 \end{figure}

\noindent Les modèles de paroi standard ou amélioré sont justes mais pas vraiment précis. Ils nous permettent d'avoir qu'une tendance l'évolution de $k$. Nous activons alors les effets d'écho des parois et nous obtenons sur la figure de droite, la courbe verte qui approche mieux que le modèle REA le plus précis (avec RNG) que nous avions étudié jusque là. Nous allons maintenant activer l'option BCK et nous comparons cette courbe avec l'option écho.

\begin{figure}[ht!]
\centering
\includegraphics[scale=0.4]{\pathpic RSM_echo_BC_k.png}
\caption{Tracé de $k$ lorsque les options écho et écho-BCK sont activés. Les résultats sont les plus précis obtenus dans ce projet. Notons que le pic pour $ y \approx 40\ m$ n'est pas très bien approximer par l'ensemble de nos modèles.}
\label{RSM_k2}
\end{figure}
\noindent Pour l'approximation de $k$, les tracés écho et écho-BCK donnent des résultats quasi similaires, approximant bien le tracé de la DNS, le pic d'énergie $k$ n'est toujours pas vraiment présent pour ces modèles. \\
Étudions de la même manière les courbes des RSM pour la production de $k$ : $P_k$ :

\begin{figure}[ht!]
\centering
\includegraphics[height = 5.5 cm, width = 8.7 cm ]{\pathpic RSM_std_enh_REA_pk.png} \hfill 
\includegraphics[height = 5.5 cm, width = 8.7 cm]{\pathpic RSM_enh_echo_REA_Pk.png} 
 \caption{À gauche comparaison de $P_k$ entre les RSM sans echo ni BCK, le modèle REA et la DNS. Les résultats ne sont pas très précis ici non plus pour les RSM. Nous comparons ensuite dans la figure de droite les modèle RSM enh (pour enhanced) REA et RSM écho.}
 \label{RSM_Pk}
 \end{figure}
 Ici non plus les RSM en loi de paroi standard et amélioré ne sont pas très précis. En revanche, avec l'option écho, le RSM et le REA se superposent presque avec la DNS. 
 
 Essayons avec l'option écho-BCK et comparons avec le modèle REA  : 

\begin{figure}[ht!]
\centering
\includegraphics[scale=0.5]{\pathpic RSM_BC_REA_Pk.png}
\caption{Tracé de $P_k$ pour le modèle REA et écho-BCK sont activés. Ici le écho-BCK n'est pas plus précis que le REA.}
\label{RSM_Pk2}
\end{figure}

\noindent Un résultat surprenant sur la figure (\ref{RSM_Pk2}) : malgré les options écho-BCK et tous les termes physiques ajoutés, le RSM donne une plus mauvaise approximation de la production de $k$ que REA. \\
Traçons le taux de dissipation : 

\begin{figure}[ht!]
\centering
\includegraphics[height = 5 cm, width = 8.7 cm ]{\pathpic RSM_std_enh_eps.png} \hfill 
\includegraphics[height = 5 cm, width = 8.7 cm]{\pathpic RSM_echo_BC_eps.png} 
 \caption{À gauche comparaison entre les RSM sans echo ni BCK et la DNS. La dissipation est largement surestimée pour $y<20 \ m$ dans les RSM. À droite nous comparons les modèles écho et écho-BCK vis-à-vis de la DNS.}
 \label{RSM_eps1}
 \end{figure}

\begin{figure}[ht!]
\centering
\includegraphics[scale=0.5]{\pathpic k_e_REA_RNG_enh_RSM_echo_Bc_eps.png}
\caption{Comparaison RNG, REA et RSM écho, écho-BCK. Les quatre courbes sont très proches et se superposent souvent. Subsistent encore des erreurs dans l'approximation (par rapport à la DNS courbe cyan) pour les $y<30 \ m$.}
\label{RSM_eps2}
\end{figure} 
 
 Ces trois premiers tracés confirment que les modèles RSM écho et écho-BCK sont très affûtés si ce n'est quelques erreurs. Par contre, les modèles de loi de paroi standard et amélioré n'ont pas l'air compatible avec la précision imposée par le raffinement du modèle RSM même standard. 

\pagebreak  

 Nous terminons le tracé des grandeurs communes aux différents modèles. Pour cela nous tracé un pèle-mêle des courbes de viscosité turbulente :
 
\begin{figure}[ht!]
\centering
\includegraphics[scale=0.5]{\pathpic all_nut.png}
\caption{}
\label{all_nut}
\end{figure} 
\noindent Concernant la viscosité turbulente, les RSM ne sont pas plus précis que certains modèles \keps $ $ étudiés dans ce projet et ce même si on prend en compte les effets d'écho de parois ou encore un sous maillage pour l'énergie cinétique turbulente. Ce résultat est un peu surprenant et nous pouvons qu'émettre l'hypothèse suivante : les constantes dans les modèles \keps $ $ ont été calculées à partir d'expériences, et même si ces constantes ne sont pas exactement juste, elles prennent en compte une grande gamme de fluctuations et pas seulement les moyennes de ces fluctuations. Cette hypothèse peut aussi être appliquée à tous les résultats pour lesquels les modèles REA ou RNG ou parfois même le \keps $ $ standard avec loi de paroi amélioré donne des résultats autant voire plus précis.\\
  
 \npurple \subsection{Tracés des tensions de Reynolds}  \black
 \noindent La particularité du RSM est qu'il prend en compte plus de physique dans ses calculs. Il calcule les évolutions des tensions de Reynolds et permet d'analyser leurs évolutions. Nous allons tracer ces tensions pour les différents modèles de paroi du RSM que nous avons considéré jusque là. Dans cette partie nous allons tracer toutes les tensions de Reynolds (sauf $\overline{u_2u_3}$ et $\overline{u_1u_3}$) et pour chacune d'elle, nous comparerons les RSM avec les modèles de paroi standard et améliorée puis les RSM avec écho et écho-BCK en modèle de paroi amélioré. Nous interpréterons les différentes figures et discuterons des modèles après toutes les figures.\\
 
 \begin{figure}[ht!]
 \centering
 \includegraphics[height = 6 cm, width = 8.7 cm]{\pathpic RSM_std_enh_uu.png} \hfill
 \includegraphics[height = 6 cm, width = 8.7 cm]{\pathpic RSM_echo_BCK_uu.png}
 \caption{Comparaisons des modèles de paroi standard, amélioré, écho et écho-BCK, pour la tension $\overline{u_1^2}$ avec la courbe de DNS.}
 \label{uu}
 \end{figure} 
 
  \begin{figure}[ht!]
 \centering
 \includegraphics[height = 6 cm, width = 8.7 cm]{\pathpic RSM_std_enh_vv.png} \hfill
 \includegraphics[height = 6 cm, width = 8.7 cm]{\pathpic RSM_echo_BC_vv.png}
 \caption{Tracé des modèles de paroi standard, amélioré, écho et écho-BCK, pour la tension $\overline{u_2^2}$ avec la courbe de DNS.}
 \label{vv}
 \end{figure}
 
  \begin{figure}[ht!]
 \centering
 \includegraphics[height = 6 cm, width = 8.7 cm]{\pathpic RSM_std_enh_ww.png} \hfill
 \includegraphics[height = 6 cm, width = 8.7 cm]{\pathpic RSM_echo_BC_ww.png}
 \caption{Comparaison des modèles de paroi standard et amélioré (à gauche) ainsi qu'écho et écho-BCK (à droite), pour la tension $\overline{u_3^2}$ avec la courbe de DNS.}
 \label{ww}
 \end{figure}
 
\pagebreak 
 
 \begin{figure}[ht!]
 \centering
 \includegraphics[height = 6 cm, width = 8.7 cm]{\pathpic RSM_std_enh_uv.png} \hfill
 \includegraphics[height = 6 cm, width = 8.7 cm]{\pathpic RSM_echo_BC_uv.png}
 \caption{Comparaison des modèles de paroi standard et amélioré (à gauche) ainsi qu'écho et écho-BCK (à droite), pour la tension $\overline{u_1u_2}$ avec la courbe de DNS.}
 \label{uv}
 \end{figure}
 

\noindent De manière générale, les RSM standards ne donnent pas de valeurs précises mais l'ordre de grandeurs. Les tensions $\overline{u_2^2}$ et $\overline{u_3^2}$ sont surestimées par rapport aux valeurs de la DNS. Pour la tension $\overline{u_1^2}$, les valeurs sont dans un premier temps inférieur à celles de la DNS puis passent supérieures. Les valeurs de $\overline{u_1u_2}$ sont sous estimées. \\
Ces différents sens de convergence (par valeurs supérieures ou inférieures) pourraient influer sur les estimations à partir de ces modèles peu précis vis-vis du tenseur $\overline{u_iu_j}$.\\

Dès lors que l'on ajoute l'option écho, les résultats sont beaucoup plus précis. L'ajout de l'option de sous maillage pour $k$ (BCK) n'améliore que très peu les résultats puisqu'en général, les courbes écho et écho-BCK se superposent, mais d'un point de vue physique, le modèle avec l'option BCK contient plus d'informations. \\

 Avant de conclure ce projet, nous allons essayer de discuter sur les modèles RSM d'un point de vue industriel \cad évaluer le rapport qualité des résultats - temps de calcul. Sur un sujet dont la géométrie est simple, les modèles RSM utilisés ici nécessitent de 2 minutes de calcul à 10 minutes, effectuant respectivement entre 2200 itérations et 9800 itérations (valeurs arrondies). \\
 Le RSM à loi de paroi standard calcule pendant 3 min 18 pour des résultats dont la précision n'est pas énorme. Lorsque l'on améliore la loi de paroi, le temps de calcul est réduit, 2 min 14, mais les résultats étaient parfois pire qu'avec le modèle standard. Ces deux modèles ne sont donc pas une solution envisageable. \\
 Lorsque l'on actionne l'option écho, les résultats étaient très satisfaisant et assez étonnamment, Fluent calcule pendant 2 min 19 sur 2200 itérations soit autant qu'en RSM standard avec loi de parois amélioré. Enfin, si l'on active l'option BCK, le temps de calcul quadruple pour culminer à environ 10 min pour 9800 itérations. Ce temps de calcul supplémentaire n'impacte pas la précision des calculs, si bien que les courbes écho-BCK et écho se superposent pour presque tous les cas.\\
L'option écho se dégage donc clairement dans une logique de précision par rapport au temps de calcul.\\
 
 \brik
 \section*{Conclusion} \black
Au cours de cette étude, nous avons étudié différents modèles de Fluent avec des lois de parois différentes sur un problème à géométrie simple. Le but était d'essayer de déterminer quel serait le modèle le plus efficace \cad $ $ lequel donnerait des résultats précis, avec du sens physique et ce le plus rapidement. \\
Les modèles qui se sont dégagés sont les modèles RNG et RSM avec écho, les deux prenant un modèle de paroi amélioré. \\
En effet, le premier (RNG) n'a nécessité que 43 secondes de calculs et a convergé au bout de 780 itérations. De plus, il a corrigé les modèles \keps$ $ standard puisqu'il recalcule les constantes qui sont injectées dans le cas classique du \keps $ $ voir équations (\ref{k_keps}), (\ref{eps_keps}) et (\ref{keps_constantes}). Les grandeurs calculées approchent avec une grande précision les valeurs réelles (DNS) et est autant précis voire plus que les modèles RSM toute option ajoutée. 
Cependant, puisque c'est un modèle \keps$ $, il ne cherche pas à prendre en compte des grandeurs comme les tensions de Reynolds et du point de vue physique cela peut être insatisfaisant. Cela explique pourquoi nous avons également retenu le modèle RSM, option écho. \\
Nous avons établi que ce dernier  était très précis tant au niveau concret qu'au niveau physique. Il nécessite environ 2 min 20 de calcul pour 2200 itérations.
Ainsi, au profit d'un plus grand sens physique, on peut utiliser le modèle RSM-écho mais si on recherche la précision sans toutes les informations on privilégiera le modèle \keps $ $ RNG.\\
En ouverture, nous pourrons tester nos conjectures sur des problèmes à géométrie plus complexes et également tester les limites des différents modèles.
 
\end{document}
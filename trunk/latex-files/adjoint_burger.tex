\documentclass[a4paper,12pt]{article} 
\usepackage[utf8x]{inputenc}
\usepackage[french]{babel}
\usepackage{mathtools}
\usepackage{amsmath, amssymb, amsfonts}
\usepackage{textcomp}
\usepackage[nointegrals]{wasysym}			% Collection de symboles mathématiques
\usepackage{ifthen}
\usepackage{tabularx}	 				% Gestion avancée des tableaux
\usepackage{longtable}		
%\usepackage{cleveref}

\usepackage{enumitem}
\usepackage{wrapfig}
%\usepackage[squaren]{SIunits}
%\usepackage[T1]{fontenc}				% Indispendable, présent dans tous les codes exemples
\usepackage[linkcolor=red,colorlinks=true, citecolor=SaddleBrown, urlcolor=MidnightBlue]{hyperref} 	% Hyper ref
\usepackage{listings}					% Pour citer du code
\usepackage[justification=centering]{caption}
\usepackage{sistyle} 
\usepackage{numprint}
\usepackage{wrapfig}
\usepackage{cite}	
\usepackage{url} 					% Pour citer les sites internet dans la
%\usepackage{cleveref}
\usepackage{setspace}

\usepackage{graphicx}		 			% Inclusion des figures
\graphicspath{{./pic/}}
\usepackage[svgnames]{xcolor}			%https://www.latextemplates.com/svgnames-colors

%%% Commandes utiles définies%
\newcommand{\argmin}{\mathop{\mathrm{argmin}}}

\newcommand{\bepar}[1]{
	\left( #1 \right)  
}

\newcommand{\becro}[1]{
	\left[ #1 \right]  
}

\newcommand{\beacc}[1]{
	\left\{ #1 \right \}  
}

\newcommand{\norm}[1]{
	\left \vert \left \vert #1 \right \vert  \right \vert
}

\newcommand{\uin}[2]{
	\underline{\texttt{u}}_{#1}^{#2}
}
\newcommand{\ui}[1]{
	\underline{\texttt{u}}_{#1}^{n}
}
\newcommand{\dij}[1]{
	\delta_{#1,j}
}
\usepackage{listings}					% Pour citer du code
%%%%%%%%%%%%%%%%%%%
%%% Élément pour citer des codes %%%
\lstset{
language=Python,
basicstyle=\ttfamily\bfseries\small, %
identifierstyle=\bfseries\color{black}, %
keywordstyle=\color{blue}, %
stringstyle=\color{black!90}, %
commentstyle=\it\color{black!70}, %
columns=flexible, %
tabsize=4, %
extendedchars=true, %
showspaces=false, %
showstringspaces=false, % %
numberstyle=\small, %
breaklines=true, %
breakautoindent=true, %
captionpos=b,
otherkeywords={cross_val_score},
keywords=[0]{cv},
keywordstyle=[0]{\color{red}},
}
%%%%%%%%%%%%%%%%%%%%%
\title{\navy \textbf{Calculs méthode adjoint burger : \color{black}}}%%%%%%%%%%%%%%%%%%%%
\date{}
%\usepackage{multicol}
%\usepackage{etoolbox}
%\patchcmd{\thebibliography}{\section*{\refname}}
%    {\begin{multicols}{2}[\section*{\refname}]}{}{}
%\patchcmd{\endthebibliography}{\endlist}{\endlist\end{multicols}}{}{}
\usepackage[authoryear]{natbib}
\usepackage{geometry}
\geometry{hmargin=2cm, vmargin=2cm}

%%%%%%%%%%%%%%%%%%%%
%%% Couleurs %%%
\xdefinecolor{brick}{named}{DarkRed}
\xdefinecolor{navy}{named}{Navy}
\xdefinecolor{midblue}{named}{MidnightBlue}
\xdefinecolor{dsb}{named}{DarkSlateGray}
\xdefinecolor{dgreen}{named}{DarkGreen}
\xdefinecolor{indian}{named}{IndianRed}

%%% 	Raccourcis 	%%%
\newcommand{\keps}{$k-\varepsilon$}
\newcommand\bk{\color{black}}
\newcommand\brick{\color{brick}}
\newcommand\navy{\color{navy}}
\newcommand\midblue{\color{midblue}}
\newcommand\dsb{\color{dsb}}
\newcommand{\dgreen}{\color{dgreen}}
\newcommand{\dpurple}{\color{indian}}
\newcommand\red{\color{red}}

%%%%%%%% Cigles
\newcommand{\rap}{par rapport}
\newcommand{\cad}{c'est-à-dire}
\newcommand{\vav}{vis-à-vis}

%%%%%%%% Autres

%%%%%%%%%%%%%%%%%%%
% Syntax: \colorboxed[<color model>]{<color specification>}{<math formula>}
\newcommand*{\colorboxed}{}
\def\colorboxed#1#{%
  \colorboxedAux{#1}%
}
\newcommand*{\colorboxedAux}[3]{%
  % #1: optional argument for color model
  % #2: color specification
  % #3: formula
  \begingroup
    \colorlet{cb@saved}{.}%
    \color#1{#2}%
    \boxed{%
      \color{cb@saved}%
      #3%
    }%
  \endgroup
}
\renewcommand{\sectionmark}[1]{\markright{#1}}
\usepackage{fancyhdr}
\pagestyle{fancy}
\lhead{\textbf{Nathaniel} \brick \textbf{\textsc{Saura}}}
\rhead{\markright}
\cfoot{\thepage}
\renewcommand{\headrulewidth}{0.4pt}

\numberwithin{equation}{section} %%%% To count the equation like Section.Number

\usepackage{accents}
\newcommand{\vect}[1]{\accentset{\Rightarrow}{#1}}

\usepackage{multicol}		% Pour utiliser \hfill et découper une partie de son texte en colonnes
\setlength{\columnseprule}{0.1pt}
\def\columnseprulecolor{\color{red}}
\setlength{\columnsep}{1.5cm}

% Numéro Roman pour le texte
\makeatletter
\newcommand*{\rom}[1]{\expandafter\@slowromancap\romannumeral #1@}
\makeatother

\begin{document}
\maketitle 
\setcounter{section}{1}
\noindent On a le problème exact :
\begin{equation} \tag{1 - exact}
\partial_t u + \partial_x \frac{u^2}{2} - \nu \partial_{xx}u = 0 \label{pb_ex}
\end{equation}
Et le modèle inexact :
\begin{equation} \tag{2 - inexact}
\mathcal{R} = \partial_t u + \beta \partial_x u - \nu \partial_{xx}u = 0 \label{pb_inex}
\end{equation}
Avec $\beta$ le terme que l'on cherche à inférer.\\

\noindent  On discrétise \eqref{pb_inex} en utilisant Crank Nicholson pour le terme en dérivée seconde :
\begin{equation} \setcounter{equation}{1}
\mathcal{R}^n_i = u^{n+1}_i - u^n_i + r \beta  \bepar{u_{i+1} - u_i} - \frac{\alpha}{2} \bepar{u^{n+1}_{i+1} - 2u^{n+1}_i + u^{n+1}_{i-1} + u^{n}_{i+1} - 2 u^{n}_{i} + u^{n}_{i-1}}  = 0 \label{inex_discr}
\end{equation}
Avec $\displaystyle r = \frac{\Delta t}{\Delta x}$ et $ \displaystyle\alpha = \frac{\nu\Delta t}{\Delta x ^2}$.\\
 \\
 \noindent La méthode adjoint consiste à écrire 
 \begin{equation}
\frac{d\mathcal{J}}{d\beta} = \partial_\beta \mathcal{J} - \partial_u \mathcal{J} \bepar{\partial_u \mathcal{R}}^{-1}\partial_\beta\mathcal{R} \label{adjoint}
 \end{equation}
 
\noindent On écrit $\mathcal{J}$ comme une Ridge regression. On introduit $U^{n+1}_{\text{obs}}$ la solution du problème \eqref{pb_ex} utilisant Lax-Wendroff, et $U^{n+1}_\beta$ la solution de \eqref{pb_inex} utilisant Crank Nicholson.\\
 \begin{equation}
 \mathcal{J}^n = \frac{1}{2} \bepar{\bepar{\bepar{U^{n+1}_{\text{obs}}}^\text{LW} - \bepar{U^{n+1}_\beta}^\text{CN}}^T \text{C}_\text{obs}^{-1} \bepar{\bepar{U^{n+1}_{\text{obs}}}^\text{LW} - \bepar{U^{n+1}_\beta}^\text{CN}} + \lambda \bepar{\beta^n - \beta_{\text{p}}}^T \text{I}_d \bepar{\beta^n - \beta_\text{p}}}
 \end{equation}
 
 
\noindent À chaque itération, par le biais d'une minimisation de $\mathcal{J}^n$, on cherchera $\beta^n$ tel que $U^n_{\text{obs}}$ et $U^n_\beta$ soient le plus proches possible, ainsi on fait converger la solution de \eqref{pb_inex} vers celle de \eqref{pb_ex} à l'itération n, pour tout n.\\
   
\noindent  On exprime les termes de \eqref{adjoint} : 
\begin{equation}
\frac{\partial \mathcal{J}^n }{\partial \beta^n_j} = \lambda \delta_{ij} \bepar{\beta_i^n - \beta_i^{\text{p}}}
\end{equation}
\red
\begin{equation}
\frac{\partial \mathcal{J}^n}{\partial u_j^n} = -\text{C}_\text{obs}^{-1} \bepar{U^n_{\text{obs, j}} - U^n_{\beta, j}} ^*
\end{equation}
\bk
\begin{equation}
\frac{\partial \mathcal{R}^n_i }{u_j^n} = \delta_{ij}\bepar{\alpha -\bepar{1 + r \beta_i^n}} + \delta_{i+1, j} \bepar{r \beta^n_i - \frac{\alpha}{2}} + \delta_{i-1, j} \bepar{-\frac{\alpha}{2}} 
\end{equation}
\begin{equation}
\frac{\partial \mathcal{R}_i^n}{\partial \beta_i^n} = \delta_{ij} r \bepar{u^n_{i+1} - u^n_i}
\end{equation}
Les différentes matrices sont évaluées puis calcule de DJ et procédure d'optimisation utilisant BFGS, Nelder-Meld, Powell, CG etc.\\
\red $^*$ Cette équation semble fausse.
\bk
\pagebreak

On rappelle l'écriture du Lax-Wendroff (LW) :
\begin{equation}
\partial_t \vec{W} +  \partial_x \vec{F} = \nu \partial_{x,x} \vec{W}
\end{equation}
(Ici en 1d) Avec $W = u_x = u $ et $\vec{F} = F$ le vecteur flux associé soit ici $\displaystyle \frac{u^2}{2}$.\\
Le LW est un schéma à deux temps dont la maille est en diamant :
\begin{equation} \tag{\'etape 1}
\uin{i + \frac{1}{2}}{n + \frac{1}{2}} = \frac{1}{2} \bepar{\uin{i+1}{n} + \uin{i}{n}} - \frac{r}{2} \bepar{\underline{\texttt{f}}_{i+1}^n - \underline{\texttt{f}}_{i}^n} \label{et1}
\end{equation}
\begin{equation} \tag{\'etape 2}
\uin{i}{n+1} = \uin{i}{n} - r \bepar{\underline{\texttt{f}}_{i+\frac{1}{2}}^{n + \frac{1}{2}} - \underline{\texttt{f}}_{i-\frac{1}{2}}^{n + \frac{1}{2}}} \label{et2}
\end{equation}
Or puisque $\displaystyle \underline{\texttt{f}}_{i+1}^n = \frac{\uin{i+1}{n} ²}{2}$ on réécrit l'\ref{et1} en fonction de $\uin{i}{n}$ seulement, aux points $\bepar{ i - \frac{1}{2},\, n+\frac{1}{2}}$ et $\bepar{ i + \frac{1}{2},\, n+\frac{1}{2}}$   :
\begin{equation}
\uin{i - \frac{1}{2}}{n+\frac{1}{2}} = \bepar{\frac{1}{2}\bepar{\uin{i}{n}+\uin{i-1}{n}} - \frac{r}{4}\bepar{\bepar{\uin{i}{n}}^2 - \bepar{\uin{i-1}{n}}^2}}
\end{equation}
\begin{equation}
\uin{i + \frac{1}{2}}{n+\frac{1}{2}} = \bepar{\frac{1}{2}\bepar{\uin{i+1}{n}+\uin{i}{n}} - \frac{r}{4}\bepar{\bepar{\uin{i+1}{n}}^2 - \bepar{\uin{i}{n}}^2}}
\end{equation}
On exprime alors les carrés de ces deux expressions, et on réécrit la deuxième étapes avec les nouvelles expressions. On obtient alors :
\begin{align*}
\uin{i}{n+1} =\  &\uin{i}{n} \\
 -\ & \frac{r}{8} \bepar{\bepar{\ui{i+1}}^2 + 2\, \ui{n}\bepar{\ui{i+1} - \ui{i-1}} - \bepar{\ui{i-1}}^2} \\
+ & \frac{r^2}{8}\bepar{\bepar{\ui{i+1}}^3 +\ui{i}\bepar{\bepar{\ui{i+1}}^2 + \bepar{\ui{i-1}}^2} - \bepar{ \ui{i}}^2\bepar{\ui{i+1} + \ui{i-1}} + \bepar{\ui{i-1}}^3 } \\
-\ & \frac{r^3}{32} \bepar{\bepar{\ui{i+1}}^4 -2\,\bepar{\ui{i}}^2 \bepar{\bepar{\ui{i+1}}^2 - \bepar{\ui{i-1}}^2} - \bepar{\ui{i-1}}^4}
\end{align*}
On dérive alors cette expression :
\begin{align*}
\frac{\partial \bepar{\uin{i}{n+1}}^\text{LW}}{\partial \ui{j} } &= \\
 \dij{i} \\
-\ & \frac{r}{8} \bepar{2\, \ui{i+1} \dij{i+1} + 2\, \dij{i}\bepar{\ui{i+1} - \ui{i-1}} + 2\, \ui{i}\bepar{\dij{i+1}- \dij{i-1}} - 2\, \ui{i-1} \dij{i-1}} \\
+\ & \frac{r^2}{8} \bepar{ 3\bepar{\ui{i+1}}^2\dij{i+1} + \dij{i}\bepar{\bepar{\ui{i+1}}^2 + \bepar{\ui{i-1}}^2} + 2\ui{i}\bepar{\ui{i+1}\dij{i+1} + \ui{i-1}\dij{i-1}}} \\
+\ & \frac{r^2}{8} \bepar{3\bepar{\ui{i-1}}^2\dij{i-1} -2\dij{i}\ui{i}\bepar{\ui{i+1} + \ui{i-1}} - \bepar{\ui{i}}^2\,\bepar{\dij{i+1} + \dij{i-1}}} \\
-\ & \frac{r^3}{8}\bepar{\bepar{\ui{i+1}}^3 \dij{i+1} -\ui{i}\dij{i}\bepar{\bepar{\ui{i+1}}^2 - \bepar{\ui{i-1}}^2}- \bepar{\ui{i}}^2\bepar{\ui{i+1}\dij{i+1} - \ui{i-1}\dij{i-1}} -\bepar{\ui{i-1}}^3\dij{i-1} }
\end{align*}
On peut écrire ceci de manière matricielle comme une matrice tridiagonale dont les éléments sont :
\begin{equation} 
 1 - \frac{r}{8}\, \becro{2\bepar{\ui{i+1} - \ui{i-1}} - r \bepar{\bepar{\ui{i+1}}^2 + -2\ui{i}\bepar{\ui{i+1} + \ui{i-1}} + \bepar{\ui{i-1}}^2} -r^2\ui{i}\bepar{\bepar{\ui{i+1}}^2 - \bepar{\ui{i-1}}^2 }}
 \red \tag{ $\dij{i}$ }
 \end{equation} 
 
 \begin{equation} 
 \frac{r}{8}\becro{ 2\bepar{\ui{i}+\ui{i-1}} + r\bepar{\ui{i-1}\bepar{2\ui{i} + 3\ui{i-1}} -\bepar{\ui{i}}^2} -r^2\, \ui{i-1}\bepar{\bepar{\ui{i}}^2 - \bepar{\ui{i-1}}^2} }
 \red \tag{$\dij{i-1}$ }
 \end{equation}
 
 \begin{equation}
 - \frac{r}{8}\, \becro{2\bepar{\ui{i+1} + \ui{i}} - r\,\bepar{\ui{i+1} \bepar{2\ui{i} + 3\ui{i+1}} -\bepar{\ui{i}}^2} + r^2\ui{i+1}\bepar{\bepar{\ui{i+1}}^2 - \bepar{\ui{i}}^2}}
 \red \tag{$\dij{i+1}$ }
 \end{equation}
 
 \noindent On exprime à présent la dérivée $\partial_{\ui{j}} \bepar{\uin{i}{n+1}}^\text{CN}$ :
 \begin{align*}
 \frac{\partial \bepar{\uin{i}{n+1}}^\text{CN}}{\partial \ui{j} } & = \dij{i}\bepar{1 - \alpha +  \beta r} + \dij{i+1}\bepar{\frac{\alpha}{2} - \beta r} + \dij{i-1}\frac{\alpha}{2}
 \end{align*}
%Finalement  
% \begin{equation} \tag{$\partial \mathcal{J}^n_{u_j^n}$ }
% \frac{\partial \mathcal{J}^n_i}{\partial u_j^n} = \frac{\partial \bepar{\bepar{U^{n+1}_{\text{obs}}}^\text{LW} - \bepar{U^{n+1}_\beta}^\text{CN}}^T \text{C}_\text{obs}^{-1} \bepar{\bepar{U^{n+1}_{\text{obs}}}^\text{LW} - \bepar{U^{n+1}_\beta}^\text{CN}}}{\partial u_j^n} \label{djuj}
% \end{equation}
% En remplaçant $\bepar{U^{n+1}_{\text{obs}}}^\text{LW} - \bepar{U^{n+1}_\beta}^\text{CN}$ par \textbf{X}, on peut écrire \ref{djuj} comme $\displaystyle \frac{d\textbf{X}^T\textbf{A}\textbf{X}}{d\textbf{X}}$ et utilisé le résultat suivant
% \begin{equation*}
% \frac{d\textbf{X}^T\textbf{A}\textbf{X}}{d\textbf{X}} = \textbf{X}^T\bepar{\textbf{A} + \textbf{A}^T}
% \end{equation*}
\end{document}
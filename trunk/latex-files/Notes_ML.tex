\documentclass[a4paper,12pt]{article} 
\usepackage[utf8x]{inputenc}
\usepackage[french]{babel}
\usepackage{mathtools}
\usepackage{amsmath}
\usepackage{amsfonts}
\usepackage{amssymb}
\usepackage{graphicx}		 			% Inclusion des figures 
\usepackage{textcomp}
\usepackage[nointegrals]{wasysym}			% Collection de symboles mathématiques
\usepackage{multicol}					% Pour utiliser \hfill
\usepackage{ifthen}
\usepackage{tabularx}	 				% Gestion avancée des tableaux
%\usepackage{cleveref}

\usepackage{enumitem}
\usepackage{wrapfig}
%\usepackage[squaren]{SIunits}
%\usepackage[T1]{fontenc}				% Indispendable, présent dans tous les codes exemples
\usepackage[linkcolor=DarkRed,colorlinks=true, citecolor= DarkGreen, urlcolor=MidnightBlue]{hyperref} 	% Hyper ref
\usepackage{listings}					% Pour citer du code
\usepackage[justification=centering]{caption}
\usepackage{sistyle} 
\usepackage{numprint}
\usepackage{wrapfig}
\usepackage{cite}	
\usepackage{url} 					% Pour citer les sites internet dans la
%\usepackage{cleveref}
\usepackage{setspace}

\usepackage[svgnames]{xcolor}			%https://www.latextemplates.com/svgnames-colors

\newcommand{\bepar}[1]{
	\left( #1 \right)  
}

\newcommand{\becro}[1]{
	\left[ #1 \right]  
}

\newcommand{\rbk}[1]{\color{red}\textit{#1} \color{black}  
}

\usepackage{listings}					% Pour citer du code
%%%%%%%%%%%%%%%%%%%
%%% Élément pour citer des codes %%%
\lstset{
language=Python,
basicstyle=\ttfamily\bfseries\small, %
identifierstyle=\bfseries\color{black}, %
keywordstyle=\color{blue}, %
stringstyle=\color{black!90}, %
commentstyle=\it\color{black!70}, %
columns=flexible, %
tabsize=4, %
extendedchars=true, %
showspaces=false, %
showstringspaces=false, % %
numberstyle=\small, %
breaklines=true, %
breakautoindent=true, %
captionpos=b,
otherkeywords={cross_val_score},
keywords=[0]{cv},
keywordstyle=[0]{\color{red}},
}
%%%%%%%%%%%%%%%%%%%%%
\title{\color{red}Notes Machine Learning \color{black}}%%%%%%%%%%%%%%%%%%%%
\date{}
%\usepackage{multicol}
%\usepackage{etoolbox}
%\patchcmd{\thebibliography}{\section*{\refname}}
%    {\begin{multicols}{2}[\section*{\refname}]}{}{}
%\patchcmd{\endthebibliography}{\endlist}{\endlist\end{multicols}}{}{}
\usepackage[authoryear]{natbib}

\usepackage{geometry}
\geometry{hmargin=2cm, vmargin=2cm}

%%%%%%%%%%%%%%%%%%%%
%%% Couleurs %%%
\xdefinecolor{brick}{named}{DarkRed}
\xdefinecolor{navy}{named}{Navy}
\xdefinecolor{midblue}{named}{MidnightBlue}
\xdefinecolor{dsb}{named}{DarkSlateGray}
\xdefinecolor{dgreen}{named}{DarkGreen}

%%% 	Raccourcis 	%%%
\newcommand{\keps}{$k-\varepsilon$}
\newcommand\bk{\color{black}}
\newcommand\brick{\color{brick}}
\newcommand\navy{\color{navy}}
\newcommand\midblue{\color{midblue}}
\newcommand\dsb{\color{dsb}}
\newcommand{\dgreen}{\color{dgreen}}
\newcommand\red{\color{red}}

%%%%%%%% Cigles
\newcommand{\rap}{par rapport }
\newcommand{\cad}{c'est-à-dire}

%%%%%%%% Autres

%%%%%%%%%%%%%%%%%%%
% Syntax: \colorboxed[<color model>]{<color specification>}{<math formula>}
\newcommand*{\colorboxed}{}
\def\colorboxed#1#{%
  \colorboxedAux{#1}%
}
\newcommand*{\colorboxedAux}[3]{%
  % #1: optional argument for color model
  % #2: color specification
  % #3: formula
  \begingroup
    \colorlet{cb@saved}{.}%
    \color#1{#2}%
    \boxed{%
      \color{cb@saved}%
      #3%
    }%
  \endgroup
}
\renewcommand{\sectionmark}[1]{\markright{#1}}
\usepackage{fancyhdr}
\pagestyle{fancy}
\lhead{\textbf{Nathaniel} \brick \textbf{\textsc{Saura}}}
\rhead{\markright}
\cfoot{\thepage}
\renewcommand{\headrulewidth}{0.4pt}

\numberwithin{equation}{section} %%%% To count the equation like Section.Number



\begin{document}
\maketitle

\tableofcontents
\bibliographystyle{apalike}
\bibliography{bibliotheque}

\newpage 

\section{Vocabulaires \& Notions}
\subsection{Formulations ou citations}
\navy \paragraph*{Machine Learning} :\bk \\
\cite{rasmussen2006gaussian} définissent le \textit{surpervised learning} comme l'apprentissage d'une correspondance entrées/sortie (on parlera de \textit{mapping}), à partir de données préalables. Ces données sont regroupées en paires dans un ensemble communément appelée Dataset : \\ 
\begin{equation}
\mathcal{D} = \left\{ \left( x_i, y_i \right) : i = 1,. . ., n \right\}
\end{equation}
 

\navy \paragraph*{Processus Gaussien (GP)} :\bk \\
Cette idée généralise la notion de distribution Gaussienne. De manière générale, lorsqu'une distribution de probabilité nous renseigne sur une variable ou un vecteur \textit{aléatoire}, un processus stochastique régit les \textit{propriétés} de la fonction.  \\

\noindent Dans un problème d'apprentissage, on doit passer de $\mathcal{D}$, \cad $ $ une connaissance \textit{spécifique}, à une connaissance \textit{généralisée} avec laquelle on sera capable de \textit{mapper} une entrée avec une sortie cohérente, selon le problème étudié. On construit donc une fonctionnelle liant entrées et sorties. Cette fonctionnelle peut se construire de plusieurs façons. \\
L'approche Neural Network consiste à n'utiliser que des combinaisons linéaires des entrées de chaque couche. Chaque site recevant cette entrée emet un signal à la couche de neuronnes suivante. Ce signal est une fonction non linéaire de l'entrée (voir plus loin ou autres notes sur les différentes fonctions d'activation).\\
Mais de maniere générale les relations entre les entrées sont toujours linéaires.\\

\noindent L'approche du processus Gaussien consiste à adapter la fonction liant les entrées. C'est-à-dire, initialement toutes les fonctions sont possibles, mais le processus Gaussiens nous permet de donner plus de poids à telle ou telle fonction en spécifiant les propriétés recherchées.\\
Même si initialement toutes les fonctions étaient possibles, le fait d'en choisir une en particulier n'influe pas sur le résultat final.\\


\subsection{Notions}
\begin{itemize}
\item[--] \textbf{Homoscédasticité} : Une des hypothèses fondamentales de la régression linéaire. \\
Se dit lorsque la variance des erreurs stochastiques de la régression est la même pour chque observation : $\displaystyle \text{\textbf{Var}}[\varepsilon_i] = \sigma^2, \, \forall i$.
\item[--] \textbf{Hétéroscédasticité} : Variance des variables est différente (s'oppose à l'homoscédasticité) : $\displaystyle \text{\textbf{Var}}[\varepsilon_i] = \sigma_i^2$ par forcément égale à $\sigma_j^2$ pour $i \ne j$.
\end{itemize}

\section{Notes -- Mueller \cite{muller2016introduction}}
\brick \subsection{Classification}\bk
\subsubsection{Binaires}
\navy \paragraph*{LogisticRegression} :\bk \\
\indent Mesure la similitude entre une donnée et les classes avec une norme euclidienne.\\
Utilise la fonction logistique $f$ qui associe à toute entrée, valeur comprise entre 0 et 1.\\
Lors du training, sont déterminées les valeurs de $a$ et $b$ qui apparaissent dans le calcul de la «probabilité d'appartenance» d'une donnée à une des deux classes : $P = f\bepar{a+bx}$\\

\noindent Lors du training, on peut pénaliser \cad $ $ décomplexifier le training set pour améliorer les prédictions. Cette amélioration est due au fait qu'en pénalisant, on empêche le modèle de prendre en compte le bruit dans les données, les exceptions ou les particularités.
\subsubsection{Regression}
\brick \subsection{Évaluer son modèle : Cross Validation (CV)} \bk
\subsubsection{CV et Stratification (SCV)}
\textit{k-fold cross-validation} : k nombre de partitions voulues. Ces partitions ont approximativement la même taille et sont appelées \textit{folds}. À chque «tour», on reserve $1$ partition pour le test et on consacre les $k-1$ partitions restantes pour le training.
À chacun de ces cas, on calcule le score sur le test set. Pour avoir une meilleure idée de la capacité de production du modèle qu'avec model.score ou model.predict\_proba, on pourra prendre la moyenne des scores des différents \textit{folds}.
\begin{lstlisting}
	scores = cross_val_score(logreg, iris.data, iris.target, cv=5)	
	Cross validation scores : [ 1., 0.96666667, 0.93333333, 0.9, 1. ]	
	Average cross-validation score : 0.96
\end{lstlisting}

\noindent Cette méthode permet d'avoir une meilleure vue d'ensemble de la base de données puisque l'on a accès aux différents scores des \textit{folds}.
Cela nous renseigne aussi sur la sensitivité du modèle (overfitting) et sur les meilleurs ou pires scénarii possibles.\\
On optimise l'utilisation des données puisqu'on n'est plus limité au 80-20 du train\_test\_split.\\
Évidemment, le coût de calcul est aussi plus élevé. \\
Pour les classifier, il est recommandé d'utiliser la stratification puisqu'elle préserve les proportions des différentes classes pour la découpe de \textit{folds} et donc augmente la capacité de généralisation du modèle. Pour les problèmes de regression, cela n'a pas de sens.\\

\subsubsection{KFold CV}
On peut ensuite construire une cross validation pour mieux contrôler ce qu'il se passe pendant la séparation.
On utilise le \textit{splitter} Kfold. Ces trois arguments sont \color{red}\textit{n\_splits} \bk qui correspond au nombre de \textit{folds} voulu, \color{red} \textit{shuffle} \bk qui est un booléen et \color{red}\textit{random\_state} \bk qu'on fixera à zéro pour avoir des résultats reproductibles. On initialsera \color{red} cv \bk à kfold :
\begin{lstlisting}
from sklearn.model_selection import KFold
kfold = KFold(n_splits=3, shuffle=True, random_state=0)
scores = cross_val_score(logreg, iris.data, iris.target, cv=kfold)
Cross validation - KFold embedded scores : [ 0.9   0.96  0.96]
\end{lstlisting}

\subsubsection{Suffle Split CV (SSCV)}
On définit \rbk{train\_size} et \rbk{test\_size} respectivement la taille des éléments du training et de celle de ceux du test. On splitera en \rbk{n\_splits} la base de données et on sélectionnera aléatoirement \rbk{train\_size} élements du split comme étant la training  et les \rbk{test\_size} comme étant réservé pour le test.\\
On pourra éventuellement réitérer cette procédure. On pourra également ne considérer qu'une partie de la base de données.
\begin{lstlisting}
from sklearn.model_selection import ShuffleSplit
SSCV	=   ShuffleSplit(test_size = 0.5, train_size = 0.5, n_splits = 10)
SSCV scores : 
[ 0.96        0.90666667  0.96        0.94666667  0.77333333  0.98666667
  0.89333333  0.96        0.93333333  0.96      ]

\end{lstlisting}

\subsubsection{Groupe - CV (GCV)}
Méthode utilisée lorsqu'un groupe de données est fortement corrélé par exemple reconnaissance des sentiments à partir de visages identiques représentant différentes émotions ou bien dans le \textit{speech recognition}, plusieurs messages mais parfois voix identiques.\\
Dans ce genre cas, utiliser la SCV peut s'avérer être une mauvaise idée. Le modèle pourrait chercher à comparer les \textit{visages} et non les sentiments.\\
Pour capter les sentiments, il faut s'assurer que dans le training set se trouvent les visages \textit{différents} exprimant la \textit{même} émotion.\\

\noindent Le \textsc{gcv} s'utilise grâce à un tableau indiquant la classe à laquelle les éléments appartiennent. De cette façon, nous sommes sûrs d'apprendre la bonne information.
\begin{lstlisting}
	from sklearn.model_selection import GroupKFold	
	#Assume the first three, second four etc belong to the same group :
	groups = [0, 0, 0, 1, 1, 1, 1, 2, 2, 3, 3, 3]
	scores = cross_val_score(logreg, X, y, groups, cv=GroupKFold(n_splits=3))
			[ 0.75        0.8         0.66666667]\end{lstlisting}

\brick \subsection{Optimiser son modèle : Grid Search (GS) et NestedCV} \bk
Pour améliorer la capacité de généralisation d'un modèle, il faut ajuster ses paramètres. Selon la dataset et selon le modèle, ajustement différent.\\
Le \textit{grid search} consiste à parcourir les paramètres du modèle afin de trouver la meilleur combinaison de paramètres possible. \\
On peut alors procéder de deux façons : «à la main» ou utilisant méthode sklearn.
\subsubsection{Simple Grid Search}
\noindent \color{green!40!black!70} \textbf{\textit{Exemple :}} \bk \\
SVM : Séparation à vaste marge contient un argument $\gamma$ qui établit la marge maximale et $C$ qui va caractériser la pénalisation.\\
On peut alors boucler sur des valeurs de $\gamma$ et sur $C$, calculer le score du modèle et retenir le meilleur score avec les meilleurs paramètres. On peut rapidement se retrouver à faire un overfitting.\\
Une façon d'éviter ce problème est de \textit{re-splitter} les données. \\
C'est important de garder un validation test en plus d'un test set.\\

\subsubsection{GridSearchCV}
On peut définir les paramètres que l'on veut tester dans un dictionnaire dont les clés sont les paramètres du modèles que l'on optimiser, les valeurs sont les valeurs de ces derniers que l'on veut considérer.
\begin{lstlisting}
param_grid = {'C': C_list,
               'gamma': gamma_list }
\end{lstlisting}
Lors de l'appelle du module \textit{GridSearchCV}, on rajoutera alors  param\_grid dans les arguments. C'est la seule fois que l'on appelle ce dictionnaire.
\begin{lstlisting}
from sklearn.model_selection import GridSearchCV
grid_search =   GridSearchCV(SVC(), param_grid, cv=5) # Objet a entrainer et evaluer

X_train, X_test, y_train, y_test = train_test_split(X, y, random_state=0)
grid_search.fit(X_train, y_train)
grid_search.score(X_test, y_test)
	score: 0.973684210526 
\end{lstlisting}
Cette méthode se créera également un ensemble de validation comme discuté.\\
On peut également accéder aux paramètres ayant menant aux meilleurs résultats mais également l'estimateur lui même avec respectivement \red \textit{grid\_search.best\_params\_}  \bk et \red \textit{grid\_search.best\_estimator\_}\bk: 
\begin{lstlisting}
#Best params
{'C': 100, 'gamma': 0.01}

#Best estimator :
SVC(C=100, cache_size=200, class_weight=None, coef0=0.0,
  decision_function_shape='ovr', degree=3, gamma=0.01, kernel='rbf',
  max_iter=-1, probability=False, random_state=None, shrinking=True,
  tol=0.001, verbose=False)

\end{lstlisting}

\noindent Ces opérations peuvent rapidement devenir coûteuse. Il faut donc bien choisir quelles gammes explorer.\\
Dans certains des cas, il est possible que certains modèles ou plutôt certains noyaux de certains modèles nécessitent plus d'arguments que d'autres.\\
Le \textit{search\_grid} peut alors défini comme une liste de dictionnaires, et on spécifiera à la main les arguments et leurs valeurs selon les noyaux.\\
Enfin, on pourra spécifier une autre CV dans l'arguments \rbk{cv}. Par défault, GridSearchCV prend du KFoldCV stratifié en classification et du KFoldCV en régression. On pourra cependant lui spécifier un Shuffe Spit ou autre.\\
Enfin, on pourra paralléliser cette recherche en modifiant la valeur de l'argument \rbk{n\_job}; l'initialiser à -1 mobilisera tous les processeurs disponibles.

\subsubsection{NestedCV}
La \rbk {Nested} CV permet de réduire la dépendance entre split et résultats, en réitérant un certain nombre de fois, et de manière différente, le splitting des données.\\
Sur chacun de ces split, on lance un \rbk{grid sreach} afin de catcher les meilleurs paramètres.\\
En fin de programme, on calcule la moyenne des scores optimaux pour les CV successives.
On pourra s'inspirer de la fonction \rbk{nested\_cv} and dataasets\_mglearn provenant de \textit{Introduction to Machine Learning with Python}.

\pagebreak 
\end{document}

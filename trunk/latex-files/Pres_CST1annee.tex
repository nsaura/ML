\documentclass[10pt,
			   xcolor=svgnames,
			   hyperref={linkcolor=red, citecolor = DarkGreen, colorlinks=true, urlcolor=Navy}]{beamer}
			   
\usepackage[english]{babel}
\usepackage[utf8x]{inputenc}

\usepackage{mathtools}
\usepackage{amsmath, amssymb, amsfonts}

\usepackage{numprint}

\usepackage{tikz}
\def\checkmark{\tikz\fill[scale=0.4](0,.35) -- (.25,0) -- (1,.7) -- (.25,.15) -- cycle;}

\mode<presentation>
{
  \usetheme{Copenhagen}      % or try Darmstadt, Madrid, Warsaw, ...
  \usecolortheme{beaver}
} 

\usepackage{multicol}
\setlength{\columnseprule}{0.pt}

\usepackage{float}
\usepackage[authoryear]{natbib}

\graphicspath{{./pic/}}

\usepackage[font=scriptsize, center]{caption}

%%% Commandes utiles définies
\newcommand{\argmin}{\mathop{\mathrm{argmin}}}

\newcommand{\bepar}[1]{
	\left( #1 \right)  
}
\newcommand{\becro}[1]{
	\left[ #1 \right]  
}

\newcommand{\norm}[1]{
	\left \vert \left \vert #1 \right \vert  \right \vert
}

\usepackage{siunitx}

\usepackage{pifont}
\newcommand{\bwarrow}{\item[\color{DarkRed} \ding{227}]}
\newcommand{\warrow}{\item[\color{blue!50!black!70} \tiny{\ding{109}}]}
\newcommand{\sarrow}{\item[\color{blue!50!black!70!orange!60} \tiny{\ding{55}}]}


\usepackage{geometry}
\geometry{hmargin=1.cm, vmargin=0cm}

\xdefinecolor{bviolet}{named}{BlueViolet}

%%%%%%%%%%%%%%%%%%%%
%%% Couleurs %%%
\xdefinecolor{brick}{named}{DarkRed}
\xdefinecolor{navy}{named}{Navy}
\xdefinecolor{midblue}{named}{MidnightBlue}
\xdefinecolor{dsb}{named}{DarkSlateGray}
\xdefinecolor{dgreen}{named}{DarkGreen}

%%% 	Raccourcis 	%%%
\newcommand{\keps}{$k-\varepsilon$}
\newcommand\bk{\color{black}}
\newcommand\brick{\color{brick}}
\newcommand\navy{\color{navy}}
\newcommand\midblue{\color{midblue}}
\newcommand\dsb{\color{dsb}}
\newcommand{\dgreen}{\color{dgreen}}
\newcommand\red{\color{red}}

\usepackage{setspace}

    \expandafter\def\expandafter\insertshorttitle\expandafter{%
       \insertshorttitle\hfill%
       \insertframenumber\,/\,\inserttotalframenumber}

% Things for first slide :
\title[First year CST defence]{\small \bk First year CST defence: \\ \color{red!80!black!85!} \Large Data driven turbulence modeling for turbulent boundary layer control}
\author[Saura Nathaniel]{\large{Author:} \normalsize Saura Nathaniel\\ \vspace{3mm}
\large{Supervised by:} \normalsize Gomez Thomas
}

\date{$5^{\text{th}}$ July 2018}
\institute{
\begin{center}
	\begin{minipage}[!ht]{0.9\textwidth}
	\centering
	\includegraphics[scale=0.75]{logo_lmfl_border.png}
	\end{minipage}
\end{center}
}

% Begin 
\begin{document}
\begin{frame}
  \titlepage
\end{frame}

\begin{frame}{Main idea through few observations}

\begin{block}{Well known question : cost computation or precision ?}
\begin{itemize}
\item[$\bullet$] DNS provides perfect solutions but needs a lot of time computation
\item[$\bullet$] RANS models provide quicker more or less precise results 
\end{itemize} 
\end{block}
\vspace{1cm}
\begin{block}{ML algorithms exploding development (in every field)}
\begin{itemize}
\item[$\bullet$] Gaussian Process
\item[$\bullet$] RForests, NNetworks (Supervised) for classification or regression tasks (more and more papers in fluid mechanics)
\item[$\bullet$] Reinforcement Learning (few papers in fluid mechanics so far)
\end{itemize}
\end{block}

\end{frame}

\begin{frame}{Main idea through few observations}
\begin{block}{Different implementations}
\begin{itemize}
	\item[$\bullet$] Data-Driven augmentation (use of high fidelity datasets) with a combination of Bayesian Inference and Gaussian Process (or other ML methods)
	\item[$\bullet$] Machine Learning to fully recover physical behaviour
\end{itemize}
\end{block} 

\tableofcontents

\end{frame}

\section{Bayesian inversion framework (BIF)}
\begin{frame}{Infer beta through inversion}
\begin{block}{Modified closure equation }
The main idea of RANS models is to add a closure equation. The main idea of BI is to add a flexible term in this closure to correct the error injected.  
\center{$\displaystyle \mathcal{F}\bepar{\color{BlueViolet} \beta(\mathbf{x}, t) \bk P_{\text{roduction}}  , D_{\text{issipation} }, T_{\text{ransport}}} = 0 $}
\end{block}
$$ \beta_M = \argmin \mathcal{J}$$
$$\mathcal{J} = \becro{\frac{1}{2} \bepar{\text{Obs} - h\bepar{\beta}}^T\textbf{C}^{-1}_m\bepar{\text{Obs} - h\bepar{\beta}} + \bepar{\beta - \beta_{\text{p}}}^T\textbf{C}_\beta^{-1}\bepar{\beta - \beta_{\text{p}}}}$$

\begin{block}{Importance and meaning of $\beta$}
\begin{itemize}
\item[$\bullet$] Discrepancy vector minimizing the distance between DNS and RANS solutions (can be seen as a vector of latent variables)
\item[$\bullet$] Maximizing $\color{BlueViolet} \beta \bk | \text{Obs}$ 
\end{itemize}
\end{block}

\end{frame}
\begin{frame}{Bayesian Inversion Framework (BIF)}
Framework proposed by Duraisamy \textit{et. al} in \citep{parish2016paradigm}
\begin{multicols*}{2}
\noindent
	\begin{figure}[!ht]
	\centering
	\includegraphics[scale=0.3]{singh.png}
	\caption{Figure extracted from \citep{singh2017machine}}
	\end{figure}
	
	\columnbreak

	\begin{itemize}
	\item[1 --] Infer descrepancy \color{BlueViolet} \textit{information} \bk $\beta$ as the MAP solution \\[5mm]
	
	\item[2 --]	\textit{Generalize} this information using ML methods to have a \color{BlueViolet} \textit{knowledge} \bk: $\beta = \mathcal{F}\bepar{\eta_1, \eta_2,...}$ \\[5mm]
 
	\item[3 --] Inject ML predicted correction into model's equation(s) \\[5mm]
	
	\end{itemize}
			
\end{multicols*}
\end{frame}

\begin{frame}{BIF example : 1D heat conduction with radiative and conductive heat sources}
\begin{block}{Real and Model equations}
\begin{itemize}
\item[$\bullet$] Real equation :
\begin{center}
$\displaystyle \frac{d^2T}{dz^2} + \varepsilon(T)\bepar{T^4_\infty - T^4} + h\bepar{T_\infty - T} = 0$
\end{center}
\item[$\bullet$] Model equation :
\begin{center}
$\displaystyle \frac{d^2T}{dz^2} + \varepsilon_0\color{BlueViolet}\beta(z)\bk\bepar{T^4_\infty - T^4}= 0$
\end{center}
\end{itemize} 
\end{block}

\begin{itemize}
\item[\dgreen \checkmark] BFGS method to minimize $\mathcal{J}$ $\rightarrow$ $\beta_{\text{MAP}}$ \& $\text{Hess}^{-1}_{\beta_{\text{MAP}}}$
\item[\dgreen \checkmark] Create $\beta_{\text{final}}$ distribution : \\
\begin{center}
$ \beta_{\text{final}} \equiv \beta \sim \mathcal{N}\bepar{\beta_{\text{MAP}}, {\text{Hess}^{-1}_{\beta_{\text{MAP}}}}}$
\end{center}
\end{itemize}

\end{frame}

\begin{frame}{BIF : Few figures based on inversion ($T_\infty = \ang{15}$C)}
\begin{multicols}{2}
\noindent
	\begin{figure}[H]
	\centering
	\includegraphics[scale=0.25]{Pres_Evolution_beta_map.png}
	\caption{Evolution of beta through the minimization process}
	\end{figure}

\columnbreak
	
	\vspace*{-1cm}
	\begin{figure}[H]
	\centering
	\includegraphics[height=4cm, width=6cm]{Pres_T_15_Full_Beta_Comp.png}
	\end{figure}
	
	\vspace*{-2.5cm}
	\begin{figure}[H]
	\centering
	\includegraphics[height=4cm, width=6cm]{Evolution_de_l'erreur_T_inf_15.png}
	\end{figure}		


	
\end{multicols}
\end{frame}

\begin{frame}{BIF : Next Step : Machine Learning}
Reproduce the previous calculs for T = 5:5:50 \\[1cm]

Construction $X$ and $y$ in $\mathcal{D} = \left\lbrace \ \underbrace{\bepar{T_{\text{inf}},T(x_i)}}_{X_i}, \ \underbrace{\beta_i}_{y_i}\ \right\rbrace$ for each T.\\
In matrix notation : 

\begin{multicols}{2}
\noindent
$$ X = \left[ \begin{array}{c} \cdot \\ \cdot \\ T_{\text{inf}},\ T(x) \\ \cdot \\ \cdot
			  \end{array}
	   \right]
$$
\columnbreak
$$ y = \left[ \begin{array}{c} \cdot \\ \cdot \\ \beta(x) \\ \cdot \\ \cdot
			  \end{array}
	   \right]
$$
\end{multicols}
Neural Network generalization : \\
\begin{center}
 $\beta_1(x), ..., \beta_n(x) \rightarrow \beta_{\text{ML}} = \mathcal{F}\bepar{\eta_1 = T_{\text{inf}}, \eta_2= T}$
\end{center}
\end{frame}

\begin{frame}{(BIF) : Prediction with NN }
$T_\infty = \ang{28}$C 
\begin{figure}[H]
	\centering
	\includegraphics[scale=0.5]{T_True_vs_T_ML_N_sample_5_T_inf_28_Adam-mean.png}
\end{figure}

\end{frame}

\begin{frame}
$T_\infty = \ang{55}$C 
\begin{figure}[H]
	\centering	
	\includegraphics[scale=0.5]{TBeta_True_vs_TBeta_ML_N_sample_=_5_T_inf_=_55.png}
\end{figure}

\end{frame}

\begin{frame}
$T_\infty = 35 + 20\sin(2\pi x)$ 
\begin{figure}[H]
	\centering	
	\includegraphics[scale=0.5]{T_True_vs_T_ML_N_sample_5_T_inf_35_+_20sin_2piz_adam-mean.png}
\end{figure}
\end{frame}


\begin{frame}{(BIF) Viscous Burgers Equation 1D}
	
	\begin{block}{Equations}
		\begin{itemize}
			\item[$\bullet$] Real equation :
				\begin{center}
					$\displaystyle \frac{\partial u}{\partial t} + u \frac{\partial u}{\partial x} - \nu\frac{\partial^2 u}{\partial x^2} = 0$
				\end{center}
			\item[$\bullet$] (Inference) Model equation (1) :
				\begin{center}
					$\displaystyle \frac{\partial u}{\partial t} + u \color{BlueViolet} \beta(x,t) \bk - \nu \frac{\partial^2 u}{\partial x^2} = 0$
				\end{center}
		\end{itemize} 
	\end{block}

	\begin{itemize}
		\item[\checkmark] Real solution obtained with Lax Wendroff Scheme
		\item[\checkmark] Infered solution obtained with Crank Nicholson Scheme
		\red \item[\red \checkmark \bk] One inference at each time iteration \bk
	\end{itemize}

\end{frame}

\begin{frame}{(BIF) VBE 1D : Inference step}
	\begin{block}{A look at the future}
		At each iteration $n$ we compute the $\beta_{\text{MAP}}$ minimizing a cost function that depends on the solution at $n+1$ : \\ 
		\begin{center}
			$\displaystyle \mathcal{J}^n = \frac{1}{2} \bepar{\Delta_{_{\text{LW-CN}}}U^{n+1}}^T \text{C}_\text{obs}^{-1} \bepar{\Delta_{_{\text{LW-CN}}}U^{n+1}} + \lambda \bepar{\beta^n -
			\beta_{\text{p}}}^T \text{I}_d \bepar{\beta^n - \beta_\text{p}}$
		\end{center}
		Where \begin{center}$\Delta_{_{\text{LW-CN}}}U^{n+1} = \bepar{U^{n+1}_{\text{obs}}}^\text{LW} - \bepar{U^{n+1}_\beta}^\text{CN}$ \end{center}
		And \\
		\begin{center}
			$\displaystyle \beta_{\text{MAP}}^n = \argmin \mathcal{J}^n$
		\end{center}
	\end{block}
	
	\begin{itemize}
		\item[\checkmark] Initialize the problem with $u_0(x,t) = \sin((2\pi x)/L)$ in L-length domain
		\item[\checkmark] Periodic boundary condition
	\end{itemize} 
\end{frame}

\begin{frame}{(BIF) VBE 1D : Inference step (figures)}
\vspace{-1cm}
\begin{multicols}{2}
\noindent
	\begin{figure}[H]
		\centering
		\includegraphics[width = 6cm, height= 6cm]{nu0_0250_CFL0_40_Nx_52_InferenceVSTrue_it5.png}
		\vspace{-0.5cm}	
		\caption{Left : Beta at time $n$ Right : Comparaison between reel solution and inferred solution at iteration 1(red) and 6(blue)}
	\end{figure}

\columnbreak

	\begin{figure}[H]
		\centering
		\includegraphics[width = 6cm, height=6cm]{nu0_0250_CFL0_40_Nx_52_InferenceVSTrue_it35.png}
		\vspace{-0.5cm}	
		\caption{Left : Beta at time $n$ Right : Comparaison between reel solution and inferred solution at iteration 31(red) and last(blue)}
	\end{figure}

\end{multicols}

\end{frame}


\begin{frame}{(BIF) VBE 1D : Machine Learning step}
	More difficult now : what are $\eta$ to use in $\beta = \mathcal{F}(\eta)$ ?
	\begin{itemize}
		\item[$\bullet$]$\beta^{n+1} = \bepar{U_{i-1}^n, U_i^n, U^n_{i+1}}$ 
		
		\item[$\bullet$] $\beta^{n+1} = \bepar{X_{i-1}^n, X_i^n, X^n_{i+1}, U_{i-1}^n, U_i^n, U^n_{i+1}}$ 		
		
		\item[$\bullet$] $\beta^{n+1} = \bepar{X_{i-1}^n, X_i^n, X^n_{i+1}, 
		U_{_{\bk i-1}}^{\red n-1}, U_{\bk i}^{\red n-1}, U_{_{\bk i+1}}^{\red n-1}, 
		U_{i-1}^n, U_i^n, U^n_{i+1}}$ 		
		
		\item[$\bullet$] \scriptsize$\beta^{n+1} = \bepar{X_{i-1}^n, X_i^n, X^n_{i+1}, 
		U_{_{\bk i-1}}^{\navy n-2}, U_{\bk i}^{\navy n-2}, U_{_{\bk i+1}}^{\navy n-2},
		U_{_{\bk i-1}}^{\red n-1}, U_{\bk i}^{\red n-1}, U_{_{\bk i+1}}^{\red n-1}, U_{i-1}^n, U_i^n, U^n_{i+1}}$ 		\\[0.54cm]
	\end{itemize}
	\normalsize

	Problem linked to this framework :
	\begin{itemize}
		\sarrow Inversion can take long time to give $\beta_{\text{MAP}}$
		\sarrow The need to produce various inferences to construct wide datasets
	\end{itemize}
	
	\begin{exampleblock}{Idea}
		Can we use NN brute force to infer the behaviour of U through time ?
	\end{exampleblock}	
	No more inference needed
\end{frame}

\section{Neural network brute force (NNBF)}
\begin{frame}{Building the datasets}

	\begin{itemize}
		\item[$\bullet$] Different sinusoides as initial conditions

		\begin{itemize}
			\sarrow Different amplitudes (between 1 and 2) -- 15 cases
			\sarrow Random phases
 		\end{itemize}
	
	\begin{figure}[H]
		\vspace{-0.3cm}		
		\centering
		\includegraphics[scale=0.3]{Initialisation_cases.png}
	\end{figure}			
		
 		\item[$\bullet$] Lax Wendroff scheme to solve each initial condition
	\end{itemize}
\end{frame}

\begin{frame}{Strategy employed here}
	We want to find the functionnal linking the solution $U^{n+1}$ and some n$^{\text{th}}$ parameters :\\
	\begin{center}
	$ \displaystyle U^{n+1} = \mathcal{F}\bepar{\eta^n}$ \\
	\end{center}
	Construction $X$ and $y$ iteration by iteration, case by case.\\

	\begin{multicols}{2}
	\noindent
	$$ X = \left[ \begin{array}{c} \cdot \\ \cdot \\ U^n_{i-1},U^n_{i}, U^n_{i+1}, \frac{U^n_{i+1} - U^n_{i-1}}{2\Delta x}  \\ \cdot \\ \cdot
				  \end{array}
		   \right]
	$$
	\columnbreak
	$$ y = \left[ \begin{array}{c} \cdot \\ \cdot \\ U^{n+1}_i \\ \cdot \\ \cdot
			  \end{array}
	   \right]
	$$
	\end{multicols}
	
	\begin{itemize}
		\item[\checkmark] $\displaystyle \frac{U^n_{i+1} - U^n_{i-1}}{2\Delta x}$ to better catch chocs \\
		\item[\checkmark] Each line of X describes the local velocity field around the i$^{\text{th}}$ point and its local variation linked with the value of the velocity at the same i$^{\text{th}}$ point
	\end{itemize}		
	
\end{frame}

\begin{frame}{Training of the algorithm}
	\begin{figure}[H]
		\centering
		\includegraphics[scale=0.45]{Cost_Evolution_:_loglog_and_lin.png}
	\end{figure}
\end{frame}

\begin{frame}{Graph architecture}
We constructed a NN with the following properties :
	\begin{itemize}
	\sarrow \red 6 \bk hidden layers with \red 80 \bk hidden nodes
	\sarrow \red SeLu \bk \citep{klambauer2017self} as activation function with default params\\
	\begin{equation*}
\text{SeLu} = 1. 0507 \,
\Big \{
		\begin{array}{c l}
		x & \text{if } x > 0 \\
		1 . 6732 \bepar{ e^x - 1} & \text{if } x \leq 0
		\end{array}						
\end{equation*}
	\sarrow \navy Adam \bk optimizer \citep{kingma2014adam} (better and the most used in literature)
	\sarrow \navy Lasso \bk cost function $$ L = \norm{ y_{\text{tr}} - y_{\text{pred}}}^2_{L_2} + \lambda \norm{w}_{L_1}$$	
	\end{itemize} 
	
%	TensorFlow graph that can be visualized with Tensorboard :
%	
%	\begin{multicols}{2}	
%		\noindent		
%		\begin{figure}[H]
%		\centering
%		\includegraphics[scale=0.2]{Graphs.png}
%		\end{figure}
%	
%	\columnbreak
%			\begin{figure}[H]
%			\centering
%			\includegraphics[scale=0.3]{SELU.png}
%			\caption{From Wikipedia}
%			\end{figure}
%		
%		\end{itemize}

\end{frame}

\begin{frame}{Brute Force on VBE 1D I}
	New initial velocity field (first we deal with the \color{orange} orange \bk one) : 
	\begin{figure}[H]
	\centering
	\includegraphics[scale=0.4]{Initialisation_cases_Andu1.png}
	\end{figure}
\end{frame}

\begin{frame}{Brute Force on VBE 1D I}
	\begin{figure}[H]
	\centering
	\includegraphics[scale=0.4]{Pres_First_Iteration_1.png}
	\end{figure} 
\end{frame}

\begin{frame}{Brute Force on VBE 1D I}
	\begin{figure}[H]
	\centering
	\includegraphics[scale=0.5]{Pres_Tenth_Iteration_1.png}
	\end{figure} 
\end{frame}

\begin{frame}{Brute Force on VBE 1D I}
	\begin{figure}[H]
	\centering
	\includegraphics[scale=0.35]{Pres_50th_Iteration_1.png}
	\end{figure} 
\end{frame}

\begin{frame}{Brute Force on VBE 1D I}
	\begin{figure}[H]
	\centering
	\includegraphics[scale=0.42]{Pres_Last_Iteration_1.png}
	\end{figure} 
\end{frame}

\begin{frame}{Brute Force on VBE 1D II}
	New initial velocity field (we deal now with the \color{cyan} cyan \bk one) : 
\vspace{-0.3cm}	
	\begin{figure}[H]
	\centering
	\includegraphics[scale=0.5]{Initialisation_cases_Last.png}
	\end{figure}
\end{frame}

\begin{frame}{Brute Force on VBE 1D II}
	\begin{figure}[H]
	\centering
	\includegraphics[scale=0.5]{Pres_First_Iteration_2.png}
	\end{figure} 
\end{frame}

\begin{frame}{Brute Force on VBE 1D II}
	\begin{figure}[H]
	\centering
	\includegraphics[scale=0.5]{Pres_Tenth_Iteration_2.png}
	\end{figure} 
\end{frame}

\begin{frame}{Brute Force on VBE 1D II}
	\begin{figure}[H]
	\centering
	\includegraphics[scale=0.5]{Pres_50th_Iteration_2.png}
	\end{figure} 
\end{frame}

\begin{frame}{Brute Force on VBE 1D II}
	\begin{figure}[H]
	\centering
	\includegraphics[scale=0.5]{Pres_Last_Iteration_2.png}
	\end{figure} 
\end{frame}

\begin{frame}{Conclusion and beyond NN Brute Force}
VBE is not the simpliest equation to evaluate NNBF because of sharp chocs are generated. But this is \navy challenging \bk. \\
 
	\begin{block}{Ideas out of NNBF :}	
		\begin{itemize}
			\item[$\bullet$] Give good approximations but can be better
			\item[$\bullet$] \textit{Accumulative errors phenomena} that impose to differ \red local \bk error (on the on going iteration) and \red global \bk errors (growing through time iterations)
			\item[$\bullet$] Genetic Algorithm can be used to get the best architecture but \red infinite \bk feature combinations possible\\[0.5cm]
		\end{itemize}
	\end{block}
\vspace{3mm}
Various directions to solve the global error such as modifying the cost function. \red We choosed Deep Reinforcement Learning (DRL)\bk

\end{frame}

\begin{frame}{DRL : How to reduce global and local errors (on goning)}
	\begin{exampleblock}{Solution : Learn !}
	
		
		The solution is to use Reinforcement learning to learn the physics
	\begin{itemize}
		\item[\checkmark] Optimize weights given a policy (conservation laws)
		\item[\checkmark] Ensure to minimize local error (per step) 
		\item[\checkmark] Golbal error should be small at the end of the process
	\end{itemize}
	\end{exampleblock}	
	\begin{itemize}
		\item[$\bullet$] Deep Reinforcement Learning (DQL) : 
		\begin{figure}[!ht]
		\centering
		\includegraphics[scale=0.2]{RL_scheme.png}
		\end{figure}
		\item[$\bullet$] A strategy has been established and wait to be tested
		
		\item[$\bullet$] More soon .. ;)
		
	\end{itemize}

\end{frame}
\begin{frame}
\begin{center}
	\color{FireBrick} \textbf{\textit{\Large{Thank you for your attention}}} \bk
\end{center}
\end{frame}

\begin{frame}
\bibliographystyle{apalike}
\bibliography{bibliotheque}
\end{frame}

\end{document}
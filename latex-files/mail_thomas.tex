\documentclass[a4paper,12pt]{report} 
\usepackage[utf8x]{inputenc}
\usepackage[french]{babel}
\usepackage{mathtools}
\usepackage{amsmath}
\usepackage{amsfonts}
\usepackage{amssymb}
\usepackage{textcomp}
\usepackage[nointegrals]{wasysym}			% Collection de symboles mathématiques
\usepackage{multicol}					% Pour utiliser \hfill
\usepackage{ifthen}
\usepackage{tabularx}	 				% Gestion avancée des tableaux
%\usepackage{cleveref}

\usepackage{float}

\usepackage{enumitem}
\usepackage{wrapfig}
%\usepackage[squaren]{SIunits}
%\usepackage[T1]{fontenc}				% Indispendable, présent dans tous les codes exemples
\usepackage[linkcolor=DarkGreen,colorlinks=true, citecolor= DarkGreen, urlcolor=MidnightBlue]{hyperref} 	% Hyper ref
\usepackage{listings}					% Pour citer du code
\usepackage[justification=centering]{caption}
\usepackage{sistyle} 
\usepackage{numprint}
\usepackage{wrapfig}
\usepackage{cite}	
\usepackage{url} 					% Pour citer les sites internet dans la
%\usepackage{cleveref}
\usepackage{setspace}

\usepackage{graphicx}		 			% Inclusion des figures
\graphicspath{{./pic/}, {./../figures/ML}}
\usepackage[svgnames]{xcolor}			%https://www.latextemplates.com/svgnames-colors

%%% Commandes utiles définies
\newcommand{\argmin}{\mathop{\mathrm{argmin}}}

\newcommand{\bepar}[1]{
	\left( #1 \right)  
}

\newcommand{\becro}[1]{
	\left[ #1 \right]  
}

\newcommand{\rbk}[1]{\color{red}\textit{#1} \color{black}  
}

\usepackage{listings}					% Pour citer du code
%%%%%%%%%%%%%%%%%%%
%%% Élément pour citer des codes %%%
\lstset{
language=Python,
basicstyle=\ttfamily\bfseries\small, %
identifierstyle=\bfseries\color{black}, %
keywordstyle=\color{blue}, %
stringstyle=\color{black!90}, %
commentstyle=\it\color{black!70}, %
columns=flexible, %
tabsize=4, %
extendedchars=true, %
showspaces=false, %
showstringspaces=false, % %
numberstyle=\small, %
breaklines=true, %
breakautoindent=true, %
captionpos=b,
otherkeywords={cross_val_score},
keywords=[0]{cv},
keywordstyle=[0]{\color{red}},
}
%%%%%%%%%%%%%%%%%%%%%
%\usepackage{multicol}
%\usepackage{etoolbox}
%\patchcmd{\thebibliography}{\section*{\refname}}
%    {\begin{multicols}{2}[\section*{\refname}]}{}{}
%\patchcmd{\endthebibliography}{\endlist}{\endlist\end{multicols}}{}{}
\usepackage[authoryear]{natbib}

\usepackage{geometry}
\geometry{hmargin=2cm, vmargin=2cm}

%%%%%%%%%%%%%%%%%%%%
%%% Couleurs %%%
\xdefinecolor{brick}{named}{DarkRed}
\xdefinecolor{navy}{named}{Navy}
\xdefinecolor{midblue}{named}{MidnightBlue}
\xdefinecolor{dsb}{named}{DarkSlateGray}
\xdefinecolor{dgreen}{named}{DarkGreen}

%%% 	Raccourcis 	%%%
\newcommand{\keps}{$k-\varepsilon$}
\newcommand\bk{\color{black}}
\newcommand\brick{\color{brick}}
\newcommand\navy{\color{navy}}
\newcommand\midblue{\color{midblue}}
\newcommand\dsb{\color{dsb}}
\newcommand{\dgreen}{\color{dgreen}}
\newcommand\red{\color{red}}

%%%%%%%% Cigles
\newcommand{\rap}{par rapport}
\newcommand{\cad}{c'est-à-dire}
\newcommand{\vav}{vis-à-vis}

%%%%%%%% Autres

%% Columns
\usepackage{multicol}		% Pour utiliser \hfill et découper une partie de son texte en colonnes
\setlength{\columnseprule}{0.pt}
%\def\columnseprulecolor{\color{red}}
%\setlength{\columnsep}{1.5cm}

%%%%%%%%%%%%%%%%%%%
% Syntax: \colorboxed[<color model>]{<color specification>}{<math formula>}
\newcommand*{\colorboxed}{}
\def\colorboxed#1#{%
  \colorboxedAux{#1}%
}
\newcommand*{\colorboxedAux}[3]{%
  % #1: optional argument for color model
  % #2: color specification
  % #3: formula
  \begingroup
    \colorlet{cb@saved}{.}%
    \color#1{#2}%
    \boxed{%
      \color{cb@saved}%
      #3%
    }%
  \endgroup
}
\title{\navy \textbf{Notes Machine Learning} \color{black}}%%%%%%%%%%%%%%%%%%%%
\date{}

\renewcommand{\sectionmark}[1]{\markright{#1}}
\usepackage{fancyhdr}
\pagestyle{fancy}
\lhead{\textbf{Nathaniel} \brick \textbf{\textsc{Saura}}}
\rhead{\markright}
\cfoot{\thepage}
\renewcommand{\headrulewidth}{0.4pt}

\numberwithin{equation}{section} %%%% To count the equation like Section.Number

\setlength{\parindent}{0pt}

\begin{document}
\subsubsection{Modification de la gradient descent :}
La GD de base s'écrit :
\begin{equation}
w_{i+1} = w_i - \eta \frac{\partial J}{\partial w_i} \tag{1} \label{eq1}
\end{equation}
Avec J la fonction de coût classique :
\begin{equation*}
J = \| y_{\text{pred}} - y_{\text{true}} \|^2 \tag{2} \label{eq2}
\end{equation*}

Ici on propose une nouvelle forme de gradient qui se propagera durant la back-propagation :
\begin{equation*}
w_{i+1} = w_i - \eta \frac{\partial J}{\partial w_i} - \gamma \frac{\partial \varepsilon_{i}}{\partial w_i} \tag{3} \label{eq3}
\end{equation*}
Avec 
\begin{equation*}
\varepsilon_{i} =  \mathcal{L}_{\text{Burgers}} = \frac{U^{n+1}_i\bepar{w_i} - U^n_i}{\Delta t} + \frac{1}{4} \frac{\bepar{{U^n_{i+1}}}^2 - \bepar{{U^n_{i-1}}}^2}{\Delta x}  \tag{4} \label{eq4}
\end{equation*}
On peut réécrire $$U^{n+1} = P\bepar{U^n\, | w, \theta}$$ C'est a dire la preédiction paramétrée par les poids  et les biais (s'ils existent).\\
Ainsi, $\varepsilon$ peut se réécrire  

\begin{equation*}
\varepsilon_{i} = \frac{P\bepar{U^n\, | w, \theta}_i - U^n_i}{\Delta t} + \frac{1}{4} \frac{\bepar{{U^n_{i+1}}}^2 - \bepar{{U^n_{i-1}}}^2}{\Delta x}  \tag{5} \label{eq5}
\end{equation*}
On peut alors écrire :
\begin{equation*}
\frac{\partial \varepsilon_{i}}{\partial w_i} = \frac{1}{\Delta t} \bepar{\frac{ \partial \, P\bepar{U^n\, | w, \theta}_i}{\partial w_i} - \frac{\partial \, U^n_i}{\partial w_i}} + \frac{1}{4 \Delta x } \bepar{\frac{\partial \bepar{U^n_{i+1}}^2}{\partial w_i} -\frac{\partial \bepar{{U^n_{i-1}}}^2}{\partial w_i}}  \tag{6} \label{eq6}
\end{equation*}
\red Il faudrait alors établir si $$\frac{\partial \, U^n_i}{\partial w_i} = 0\, ,  \forall  i$$ 

\bk Finalement la nouvelle forme de la GD s'écrit :
\begin{equation*}\boxed{
w_{i+1} = w_i - \eta \frac{\partial J}{\partial w_i} - \gamma \bepar{\frac{1}{\Delta t} \bepar{\frac{ \partial \, P\bepar{U^n\, | w, \theta}_i}{\partial w_i} - \frac{\partial \, U^n_i}{\partial w_i}} + \frac{1}{4 \Delta x } \bepar{\frac{\partial \bepar{U^n_{i+1}}^2}{\partial w_i} -\frac{\partial \bepar{{U^n_{i-1}}}^2}{\partial w_i}}}}{\navy}
\end{equation*}
\end{document}
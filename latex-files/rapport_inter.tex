\documentclass[a4paper,12pt]{report} 
\usepackage[utf8x]{inputenc}
\usepackage[french]{babel}
\usepackage{mathtools}
\usepackage{amsmath, amssymb, amsfonts}
\usepackage{textcomp}
\usepackage[nointegrals]{wasysym}			% Collection de symboles mathématiques
\usepackage{ifthen}
\usepackage{tabularx}	 				% Gestion avancée des tableaux
\usepackage{longtable}		
%\usepackage{cleveref}

\usepackage{enumitem}
\usepackage{wrapfig}
%\usepackage[squaren]{SIunits}
%\usepackage[T1]{fontenc}				% Indispendable, présent dans tous les codes exemples
\usepackage[linkcolor=Indigo,colorlinks=true, citecolor=SaddleBrown, urlcolor=MidnightBlue]{hyperref} 	% Hyper ref
\usepackage{listings}					% Pour citer du code
\usepackage[justification=centering]{caption}
\usepackage{sistyle} 
\usepackage{numprint}
\usepackage{wrapfig}
\usepackage{cite}	
\usepackage{url} 					% Pour citer les sites internet dans la
%\usepackage{cleveref}
\usepackage{setspace}

\usepackage{graphicx}		 			% Inclusion des figures
\graphicspath{{./pic/}}
\usepackage[svgnames]{xcolor}			%https://www.latextemplates.com/svgnames-colors


\usepackage{tikz} %Pour entourer un terme 
\newcommand*\circled[1]{\tikz[baseline=(char.base)]{
   \node[shape=circle,draw,inner sep=1pt] (char) {#1};}}
   
%%% Commandes utiles définies%
\newcommand{\argmin}{\mathop{\mathrm{argmin}}}

\newcommand{\bepar}[1]{
	\left( #1 \right)  
}

\newcommand{\becro}[1]{
	\left[ #1 \right]  
}

\newcommand{\beacc}[1]{
	\left\{ #1 \right \}  
}

\newcommand{\norm}[1]{
	\left \vert \left \vert #1 \right \vert  \right \vert
}

\newcommand{\rbk}[1]{\color{red}\textit{#1} \color{black}  
}

\usepackage{listings}					% Pour citer du code
%%%%%%%%%%%%%%%%%%%
%%% Élément pour citer des codes %%%
\lstset{
language=Python,
basicstyle=\ttfamily\bfseries\small, %
identifierstyle=\bfseries\color{black}, %
keywordstyle=\color{blue}, %
stringstyle=\color{black!90}, %
commentstyle=\it\color{black!70}, %
columns=flexible, %
tabsize=4, %
extendedchars=true, %
showspaces=false, %
showstringspaces=false, % %
numberstyle=\small, %
breaklines=true, %
breakautoindent=true, %
captionpos=b,
otherkeywords={cross_val_score},
keywords=[0]{cv},
keywordstyle=[0]{\color{red}},
}
%%%%%%%%%%%%%%%%%%%%%
\title{\navy \textbf{Rapport CST fin première année} \color{black}}%%%%%%%%%%%%%%%%%%%%
\date{}
%\usepackage{multicol}
%\usepackage{etoolbox}
%\patchcmd{\thebibliography}{\section*{\refname}}
%    {\begin{multicols}{2}[\section*{\refname}]}{}{}
%\patchcmd{\endthebibliography}{\endlist}{\endlist\end{multicols}}{}{}
\usepackage[authoryear]{natbib}

\usepackage{geometry}
\geometry{hmargin=2cm, vmargin=2cm}

%%%%%%%%%%%%%%%%%%%%
%%% Couleurs %%%
\xdefinecolor{brick}{named}{DarkRed}
\xdefinecolor{navy}{named}{Navy}
\xdefinecolor{midblue}{named}{MidnightBlue}
\xdefinecolor{dsb}{named}{DarkSlateGray}
\xdefinecolor{dgreen}{named}{DarkGreen}
\xdefinecolor{indian}{named}{IndianRed}

%%% 	Raccourcis 	%%%
\newcommand{\keps}{$k-\varepsilon$}
\newcommand\bk{\color{black}}
\newcommand\brick{\color{brick}}
\newcommand\navy{\color{navy}}
\newcommand\midblue{\color{midblue}}
\newcommand\dsb{\color{dsb}}
\newcommand{\dgreen}{\color{dgreen}}
\newcommand{\dpurple}{\color{indian}}
\newcommand\red{\color{red}}

%%%%%%%% Cigles
\newcommand{\rap}{par rapport}
\newcommand{\cad}{c'est-à-dire}
\newcommand{\vav}{vis-à-vis}
\newcommand{\NS}{Navier-Stokes}
\newcommand{\turb}{turbulence}
\newcommand{\inst}{instabilité}
\newcommand{\taur}{$\tau_{ij}$ }
%%%%%%%% Autres

%%%%%%%%%%%%%%%%%%%
% Syntax: \colorboxed[<color model>]{<color specification>}{<math formula>}
\newcommand*{\colorboxed}{}
\def\colorboxed#1#{%
  \colorboxedAux{#1}%
}
\newcommand*{\colorboxedAux}[3]{%
  % #1: optional argument for color model
  % #2: color specification
  % #3: formula
  \begingroup
    \colorlet{cb@saved}{.}%
    \color#1{#2}%
    \boxed{%
      \color{cb@saved}%
      #3%
    }%
  \endgroup
}

\renewcommand{\sectionmark}[1]{\markright{#1}}

\usepackage{fancyhdr}
\pagestyle{fancy}
\lhead{\textbf{Nathaniel} \brick \textbf{\textsc{Saura}}}
\rhead{\markright}
\cfoot{\thepage}
\renewcommand{\headrulewidth}{0.4pt}

\numberwithin{equation}{section} %%%% To count the equation like Section.Number

\usepackage{accents}
\newcommand{\vect}[1]{\accentset{\Rightarrow}{#1}}

\usepackage{multicol}		% Pour utiliser \hfill et découper une partie de son texte en colonnes
\setlength{\columnseprule}{0.1pt}
\def\columnseprulecolor{\color{red}}
\setlength{\columnsep}{1.5cm}

% Numéro Roman pour le texte
\makeatletter
\newcommand*{\rom}[1]{\expandafter\@slowromancap\romannumeral #1@}
\makeatother

\begin{document}
\maketitle
\section*{Introduction du sujet}
La turbulence est l'état d'un fluide dont l'écoulement possède des caractéristiques stochastiques, parfois chaotiques, qui passionne les chercheurs et les ingénieurs depuis plusieurs siècles.\\ Même si Léonard De Vinci au $\text{XV}^{\text{ème}}$ siècle avait mentionné ce caractère aléatoire et décrit quantitativement le phénomène par des observations, Osborne Reynolds en 1883 amorça la description qualitative et mathématique que nous utilisons aujourd'hui; il fut suivi par d'autres grands noms de la mécanique des fluides comme Richardson, Kolmogorov puis d'autres. \\
%Il construisit le nombre caractéristique dit de Reynolds qui compare les effets d'inertie et les effets visqueux. \\

\noindent Cette discipline est une des branches de la mécanique des fluides les plus actives qui attire énormément de chercheurs d'horizons plus ou moins éloignés de la mécanique des fluides ; Kolmogorov sus-mentionné était avant tout mathématicien.\\
Ils existent de nombreux facteurs qui jouent sur la popularité de la discipline, sans être exhaustif, il est possible d'en développer deux : dans un premier temps c'est une théorie qui traite d'écoulement que l'on retrouve autour de soi dans notre vie de tous les jours, mais également dans les océans, le ciel et dans l'espace. Le deuxième facteur clé découle du premier : au fur et à mesure que de nouvelles technologies impliquant ce genre d'écoulement émergent, de l'intelligence provenant d'autres domaines de la Physique, de l'Informatique ou des Mathématiques devient nécessaire et même indispensable. \\

Au travers de ses équations, la théorie de la Turbulence entend modéliser les fluides dont les effets d'inerties ne sont pas négligeables devant les effets visqueux. Les équations de Navier-Stokes constituent la base de cette discipline. Si on considère une parcelle de fluide et que l'on y effectue un bilan de quantité de mouvement, on obtient dans un cas incompressible \footnote{Sans forces extérieures, avec $\rho$ la masse volumique (constante)	, $\nu$ la viscosité cinématique de l'écoulement et $p$  le champ de pression} :\\
\begin{equation*}
\frac{\partial u_i}{\partial t} + \underset{\text{\red Inertie -  terme non linéaire \bk}}{\text{\red \circled{\bk $u_j\, \frac{\partial u_i}{\partial x_j}$}}} \bk = - \frac{1}{\rho} \frac{\partial p_i}{\partial x_j} + \underset{\text{\navy Visqueux \bk}}{\text{\navy \circled{\bk $\nu \frac{\partial^2 u_i }{\partial x_j^2}$}}} \bk 
%\tag{Navier-Stokes} 
\end{equation*}
Les termes d'inerties (de dimension de Joule par unité de volume) apparaissent dans le terme non-linéaire des équations de Navier-Stokes traduisant l'apport local d'énergie cinétique du fluide autour de la parcelle contrôlée.\\

Reynolds construisit un nombre sans dimension comparant les effets d'inertie sur les 
effets visqueux; il est appelé nombre de Reynolds, abrégé Reynolds; il peut s'écrire comme $$ \text{Re} = \bepar{u_j\, \frac{\partial u_i}{\partial x_j}} / \bepar{\nu \frac{\partial^2 u_i }{\partial x_j^2}} = \frac{\text{Effets inertiels} }{\text{Effets visqueux}} $$ \\
Ce nombre  caractérise l'état de l'écoulement. Pour un Reynolds faible (jusqu'à quelques centaines voire quelques miliers selon les cas), l'écoulement est dit laminaire. Lorsqu'il est au delà de la dizaine de milliers, l'écoulement est dit turbulent. \\
Entre les deux états, on parlera d'état de transition : des instabilités naissent dans l'écoulement rendant le champ de vitesse fluctuant. Plus le Reynolds augmente et plus ces fluctutations sont fortes, jusqu'à arriver à un champ de vitesse complétement aléatoire (turbulent).	 \\
La viscosité d'un fluide filtre en quelque sorte le champ de vitesse de ses fluctuations, dont l'intensité peut être déduite de l'inertie de l'écoulement. On peut dire également que la vsicosité absorbe l'énergie cinétique en la dissipant en chaleur. \\

Par analyse dimensionnelle, le nombre de Reynolds peut être construit comme le rapport $$ \text{Re} = \frac{\text{UL}}{\nu}$$ La notion clé dans la théorie de la turbulence est que les échelles spatiales L, temporelles T et leurs combinaisons (comme la vitesse U = L/T) se décomposent en un nombre très élevé de sous-échelles.
Il faut alors préciser que dans le cadre de la turbulence, lorsqu'il est question de nombre de Reynolds élevé, il est ici fait allusion au nombre de Reynolds de la plus grande échelle spatiale L, pour une échelle de vitesse U donnée. \\
La première structure turbulente qui apparaît est alors un tourbillon dont les échelles caractéristiques sont similaires à celles utilisées pour le calcul du Reynolds global. En prime, il est nécessaire de construire un temps de retournement du tourbillon $\tau_{\text{retournement}} = L/U$. \\
Plus le Reynolds est grand et moins le fluide sera capable d'absorber les instabilités induites par l'inertie du fluide. En conséquence, à haut Reynolds, de l'énergie est constamment injectée dans la zone d'intérêt. Cette énergie est à l'origine de la multitude des échelles spatiales (et donc de toutes les autres) ; on parle de cascade\footnote{L'idée de cascade fut proposée par Richardson, mais sa description s'est avérée fausse. Kolmogorov en 1941 mis en évidence la casacade de Richardson de manière plus mathématique, sans pour autant utiliser cette appellation.} ou de brisure des échelles, c'est-à-dire que l'énergie cinétique des premières structures de turbulence (à la suite d'instabilité) est distribuée de façon incohérente à des structures de taille plus petite. Ce processus est répété jusqu'à ce que le fluide soit capable de dissiper en chaleur l'énergie cinétique liées aux structures ainsi crées. \\

D'un point de vue pratique, les équations de \NS $ $ ne sont actuellement solubles «à la main» uniquement pour des cas relativemenent simple. Pour la majorité des écoulements turbulents, il est nécessaire d'utiliser des outils numériques.\\
La méthode la plus proche de la Physique (puisqu'elle résout les équations de \NS $ $ elles-mêmes) est la méthode dite directe, ou DNS (Direct Numerical Simulation). Cette méthode est extrêmement coûteuse puisqu'au fur et à mesure que le processus de turbulence se développe et se propage dans l'écoulement, il est nécessaire de raffiner le maillage sur lequel les équations sont résolues. \\
Cette méthode est souvent utilisée pour produire des résultats de référence pour l'établissement de modèles numériques moins couteux et plus ou moins précis.\\
Dans un ordre de précision, les méthodes dites de LES (Large Eddy Scales) suivent les DNS. Schématiquement, cette catégorie de méthodes consiste à définir un seuil de prise en compte des échelles\footnote{Alors que la DNS entend résoudre les équations de \NS $ $ sur toutes les échelles.}, accélérant ainsi les calculs. Pour assurer une prise en compte de la physique des plus petites échelles, la communauté utilisant ces méthodes construisent des modèles basés sur des termes modélisant la physique des échelles non prises en compte.\\
 Enfin, il existe une dernière catégorie de méthodes consistant à construire des équations modélisant le(s) transport(s) de l'énergie cinétique, du taux de dissipation de cette énergie, parfois les deux, ou bien le transport d'autres quantités (comme l'intermittence ou la viscosité turbulente). On parle de modèle à un ou deux équations.\\
Ces modèles sont basés sur trois idées : la première est une idée utilisée dans la théorie de la turbulence qui consiste à décomposer les trois compsantes spatiales des champs de vitesse $U_i$ et de pression $P_i$ en champs moyens (temporels) $\overline{U_i}$ et $\overline{P_i}$ et en champs fluctuants $u_i$ et $p_i$ ; 
on parle de Décomposition de Reynolds. On peut écrire l'équation de \NS $ $ pour le champ moyen, on parlera d'équation RANS pour Reynolds Averaged \NS $ $ : $$ \frac{\partial \overline{U}_i}{\partial t} + \overline{U}_j \frac{\partial \overline{U}_i}{\partial x_j} = -\frac{1}{\rho} \frac{\partial \overline{P}}{\partial x_j} + \nu \frac{\partial^2 \overline{U}_i}{\partial x_j^2}  - \frac{\partial \, \overline{u_iu_j}}{\partial x_j}$$ \\
Lorsque l'on opère à cette décomposition, il s'avère que le transport d'énergie associée à la partie fluctuante influe dans l'évolution du champ moyen. Ce flux d'énergie représentée par le tenseur d'ordre deux $-\rho \overline{u_iu_j}$ est appelée tenseur de Reynolds. On le note $\tau_{ij}$.\\
Pour modéliser \taur, il est nécesaire de modéliser le tenseur d'ordre trois $-\overline{u_iu_ju_k}$ dont la modélisation nécessite celle du tenseur d'ordre quatre $-\overline{u_iu_ju_ku_l}$ et ainsi de suite. \\
Le fait de d'avoir besoin d'un tenseur d'ordre n+1 pour la modélisation de celui d'ordre n schématise la cascade sus-décrite et indique que les fluctuations aux différentes échelles sont fortement couplées.\\
  Il s'avère en fin de compte que le système formé des équations RANS et de celles du champ flucutant est ouvert c'est-à-dire qu'il y a plus d'inconnu que d'équations, il faut alors en rajouter une appelée l'équation de fermeture.\\
  
  \noindent Dans les méthodes à une ou deux équations, on utilise l'hypothèse de Boussinesq. Celle-ci se base sur le fait que la création de nouvelles échelles augmente la stochasticité du champ de vitesse et donc de la friction au sein de l'écoulement.\\
  Il est proposé alors de construire un terme de viscosité de turbulence à partir d'une analagie à la viscosité cinématique. La formule proposée par Boussinesq en 1890 s'écrit alors : $$ \tau_{ij} = \rho \nu_T\bepar{\frac{\partial \overline{U}_i}{\partial x_j} + \frac{\partial \overline{U}_j}{\partial x_i}} - \frac{2}{3} \rho \overline{u_iu_i} \delta_{ij}$$


\end{document}
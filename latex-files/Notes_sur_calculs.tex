\documentclass[a4paper,12pt]{article} 
\usepackage[utf8x]{inputenc}
\usepackage[french]{babel}
\usepackage{mathtools}
\usepackage{amsmath}
\usepackage{amsfonts}
\usepackage{amssymb}
\usepackage{graphicx}		 			% Inclusion des figures 
\usepackage{textcomp}
\usepackage[nointegrals]{wasysym}			% Collection de symboles mathématiques
\usepackage{multicol}					% Pour utiliser \hfill
\usepackage{ifthen}
\usepackage{tabularx}	 				% Gestion avancée des tableaux
%\usepackage{cleveref}

\usepackage{enumitem}
\usepackage{wrapfig}
%\usepackage[squaren]{SIunits}
%\usepackage[T1]{fontenc}				% Indispendable, présent dans tous les codes exemples
\usepackage[linkcolor=DarkRed,colorlinks=true, citecolor= DarkGreen, urlcolor=MidnightBlue]{hyperref} 	% Hyper ref
\usepackage{listings}					% Pour citer du code
\usepackage[justification=centering]{caption}
\usepackage{sistyle} 
\usepackage{numprint}
\usepackage{wrapfig}
\usepackage{cite}	
\usepackage{url} 					% Pour citer les sites internet dans la
%\usepackage{cleveref}
\usepackage{setspace}

\usepackage[svgnames]{xcolor}			%https://www.latextemplates.com/svgnames-colors

\newcommand{\bepar}[1]{
	\left( #1 \right)  
}

\newcommand{\becro}[1]{
	\left[ #1 \right]  
}

\newcommand{\rbk}[1]{\color{red}\textit{#1} \color{black}  
}

\newcommand{\parfracD}[3]{
	\frac{\partial^2 #1}{\partial #2 \partial #3}
}
\newcommand{\parfrac}[2]{
	\frac{\partial #1}{\partial #2}
}
\newcommand{\kro}[2]{
\delta_{#1,#2}
}
\usepackage{listings}					% Pour citer du code
%%%%%%%%%%%%%%%%%%%
%%% Élément pour citer des codes %%%
\lstset{
language=Python,
basicstyle=\ttfamily\bfseries\small, %
identifierstyle=\bfseries\color{black}, %
keywordstyle=\color{blue}, %
stringstyle=\color{black!90}, %
commentstyle=\it\color{black!70}, %
columns=flexible, %
tabsize=4, %
extendedchars=true, %
showspaces=false, %
showstringspaces=false, % %
numberstyle=\small, %
breaklines=true, %
breakautoindent=true, %
captionpos=b,
otherkeywords={cross_val_score},
keywords=[0]{cv},
keywordstyle=[0]{\color{red}},
}
%%%%%%%%%%%%%%%%%%%%%
\title{\color{red}Notes sur calculs \color{black}}%%%%%%%%%%%%%%%%%%%%
\date{}
\usepackage{multicol}
\usepackage{etoolbox}
\patchcmd{\thebibliography}{\section*{\refname}}
    {\begin{multicols}{2}[\section*{\refname}]}{}{}
\patchcmd{\endthebibliography}{\endlist}{\endlist\end{multicols}}{}{}


\usepackage{geometry}
\geometry{hmargin=2cm, vmargin=2cm}

%%%%%%%%%%%%%%%%%%%%
%%% Couleurs %%%
\xdefinecolor{brick}{named}{DarkRed}
\xdefinecolor{navy}{named}{Navy}
\xdefinecolor{midblue}{named}{MidnightBlue}
\xdefinecolor{dsb}{named}{DarkSlateGray}
\xdefinecolor{dgreen}{named}{DarkGreen}

%%% 	Raccourcis 	%%%
\newcommand{\keps}{$k-\varepsilon$}
\newcommand\bk{\color{black}}
\newcommand\brick{\color{brick}}
\newcommand\navy{\color{navy}}
\newcommand\midblue{\color{midblue}}
\newcommand\dsb{\color{dsb}}
\newcommand{\dgreen}{\color{dgreen}}
\newcommand\red{\color{red}}

%%%%%%%% Cigles
\newcommand{\rap}{par rapport }
\newcommand{\cad}{c'est-à-dire}
\newcommand{\epsz}{\varepsilon_0}
\newcommand{\tinf}{T^4_{\text{inf}}}
\newcommand{\CP}[2]{C^{-1}_{\text{pri}_{\, \mathbf{#1, #2}}}}
\newcommand{\CO}[2]{C^{-1}_{\text{obs}_{\, \mathbf{#1, #2}}}}
\newcommand{\BP}[1]{\beta_{\text{P}_#1}}
\newcommand{\Jj}{\mathcal{J}}
\newcommand{\Rr}{\mathcal{R}}
%%%%%%%% Autres

%%%%%%%%%%%%%%%%%%%
% Syntax: \colorboxed[<color model>]{<color specification>}{<math formula>}
\newcommand*{\colorboxed}{}
\def\colorboxed#1#{%
  \colorboxedAux{#1}%
}
\newcommand*{\colorboxedAux}[3]{%
  % #1: optional argument for color model
  % #2: color specification
  % #3: formula
  \begingroup
    \colorlet{cb@saved}{.}%
    \color#1{#2}%
    \boxed{%
      \color{cb@saved}%
      #3%
    }%
  \endgroup
}
\renewcommand{\sectionmark}[1]{\markright{#1}}
\usepackage{fancyhdr}
\pagestyle{fancy}
\lhead{\textbf{Nathaniel} \brick \textbf{\textsc{Saura}}}
\rhead{\markright}
\cfoot{\thepage}
\renewcommand{\headrulewidth}{0.4pt}

\numberwithin{equation}{section} %%%% To count the equation like Section.Number

\begin{document}
\section{Sur l'erreur de la formule 14 p.762}
\noindent On a :
\begin{equation}
H_{i,j} = \parfracD{\mathcal{J}}{\beta_i}{\beta_j} + \navy \psi_m\parfracD{\mathcal{R}_m}{\beta_i }{\beta_j} \bk+ \mu_{i,m}\frac{\partial \mathcal{R}_m}{\partial \beta_j} + \red \nu_{i,m}\parfracD{\mathcal{J}}{T_n}{\beta_j} + \nu_{i,n} \psi_m \parfracD{\mathcal{R}_m}{T_n}{\beta_j} \bk \label{hijf}
\end{equation} 
Avec :
\begin{equation}
\nu_{i,n}\frac{\partial \mathcal{R}_m}{\partial T_n} = - \frac{\partial \mathcal{R}_m}{\partial \beta_i} \label{nu}
\end{equation}
Et :
\begin{equation}
\mu_{i,m}\frac{\partial \mathcal{R}_m}{\partial T_k} = - \bepar{ \navy \parfracD{\mathcal{J}}{\beta_i}{T_k} + \bk \psi_m \parfracD{\mathcal{R}_m}{\beta_i}{T_k} + \nu_{i,n} \parfracD{\mathcal{J}}{T_n}{T_k} + \nu_{i,n} \psi_m \parfracD{\mathcal{R}_m}{T_n}{T_k} } \label{muim}
\end{equation}

\noindent Pour retrouver cette équation on va se servir de deux équations :
\begin{equation}
\frac{d\mathcal{J}}{d\beta_j} = \parfrac{\mathcal{J}}{\beta_j} + \psi^T \parfrac{\mathcal{R}_m}{\beta_j} \label{eq1}
\end{equation}
\begin{equation}
\psi^T\, \parfrac{\mathcal{R}_m}{T_k} = -\, \parfrac{\mathcal{J}}{T_k} \label{eq2}
\end{equation}
On dérive alors \eqref{eq1} par rapport à $\beta_i$ pour obtenir la hessienne au point $i,j$:
\begin{equation}
H_{i,j} = \parfracD{\Jj}{\beta_i}{\beta_j} + 
\parfrac{\psi_m}{\beta_i} \parfrac{\Rr_m}{\beta_j} + \psi_m\, \parfracD{\Rr_m}{\beta_i}{T_k} \label{hess}
\end{equation}
On utilise alors \eqref{eq2} que l'on dérive par rapport à $\beta_i$
\begin{equation}
\parfrac{\psi_m}{\beta_i}\, \parfrac{\Rr_m}{T_k} + \psi_m\, \parfracD{\Rr_m}{\beta_i}{T_k} = -\, \parfracD{\Jj}{\beta_i}{T_k}
\end{equation}
On peut donc exprimer $\displaystyle \parfrac{\psi_m}{\beta_i}$ :
\begin{equation}
\parfrac{\psi_m}{\beta_i} = -\, \bepar{\parfracD{\Jj}{\beta_i}{T_k} + \psi_m\, \parfracD{\Rr_m}{\beta_i}{T_k}}\bepar{\parfrac{\Rr_m}{T_k}}^{-1}
\end{equation}
\begin{equation}
H_{i,j} = \parfracD{\Jj}{\beta_i}{\beta_j} -\, \bepar{\parfracD{\Jj}{\beta_i}{T_k} + \psi_m\, \parfracD{\Rr_m}{\beta_i}{T_k}}\bepar{\parfrac{\Rr_m}{T_k}}^{-1} \parfrac{\Rr_m}{\beta_j} + \psi_m\, \parfracD{\Rr_m}{\beta_i}{T_k}
\end{equation}
On peut poser 
\begin{equation}
\mu_{i,m} \parfrac{\Rr_m}{T_k} = - \parfracD{\Jj}{\beta_i}{T_k} - \psi_m \parfracD{\Rr_m}{\beta_i}{T_k} -\nu_{i,n} \parfracD{\Jj}{T_n}{T_k} - \nu_{i,n}\psi_m \,\parfracD{\Rr_m}{T_n}{T_k} \label{muu}
\end{equation}
Avec $\nu$ défini par \eqref{nu}.\\
Les auteurs ont injecté \eqref{muu} dans \eqref{hijf} et ont rajouté les termes en rouge pour compenser. Cependant, les termes $ \displaystyle \nu_{i,n} \parfracD{\Jj}{T_n}{T_k} \text{ et } \nu_{i,n}\psi_m \parfracD{\Rr_m}{T_n}{T_k}$ne sont pas compensés et appraissent comme des termes en trop.

\pagebreak

\section{Hessienne (obsolète)} 
\red Il est posible que les calculs avec le passage en indice sont faux puisque les contractions et/ou les transpositions ne sont pas prises en compte. \bk \\
En l'état les différentes formules ne permettent pas l'obtention d'une Hessienne (inverse) décomposable selon Cholesky. \\
\begin{equation}
H_{i,j} = \parfracD{\mathcal{J}}{\beta_i}{\beta_j} + \navy \psi_m\parfracD{\mathcal{R}_m}{\beta_i }{\beta_j} \bk+ \mu_{i,m}\frac{\partial \mathcal{R}_m}{\partial \beta_j} + \navy \nu_{i,m}\parfracD{\mathcal{J}}{T_n}{\beta_j} \bk+ \nu_{i,n} \psi_m \parfracD{\mathcal{R}_m}{T_n}{\beta_j} 
\end{equation} 
Avec :
\begin{equation}
\nu_{i,n}\frac{\partial \mathcal{R}_m}{\partial T_n} = - \frac{\partial \mathcal{R}_m}{\partial \beta_i} 
\end{equation}
Et :
\begin{equation}
\mu_{i,m}\frac{\partial \mathcal{R}_m}{\partial T_k} = - \bepar{ \navy \parfracD{\mathcal{J}}{\beta_i}{T_k} + \bk \psi_m \parfracD{\mathcal{R}_m}{\beta_i}{T_k} + \nu_{i,n} \parfracD{\mathcal{J}}{T_n}{T_k} + \nu_{i,n} \psi_m \parfracD{\mathcal{R}_m}{T_n}{T_k} }
\end{equation}

\begin{center} 
- - - - -
\end{center}

\noindent On établit les identités suivantes pour $\mathcal{R}_m$:

\begin{equation}
\parfrac{\mathcal{R}_m}{\beta_i} = \kro{m}{j}\bepar{\tinf - T_m^4}\epsz \label{rm_b}
\end{equation}
Donc \navy
\begin{equation}
 \parfracD{\mathcal{R}_m}{\beta_i}{\beta_j} = 0 \label{rm_bb}
\end{equation}\bk
Et
\begin{equation}
\parfracD{\mathcal{R}_m}{\beta_i}{T_k} = -4\epsz T_m^3 \kro{m}{i}\, \kro{m}{k}
\end{equation}

\noindent Enfin :
\begin{equation}
\parfrac{\mathcal{R}_m}{T_k} = \bepar{\frac{\kro{m+1}{k} - 2\kro{m}{k} + \kro{m-1}{k}}{\Delta z^2}} -4 \beta_m(z)\epsz T_m^3 \, \kro{m}{k} \label{r_t}
\end{equation}
Donc : 
\begin{equation}
\parfracD{\mathcal{R}_m}{T_n}{T_k} = -12\, \epsz \beta_m(z) T_m^2 \kro{m}{n}\, \kro{m}{k} \label{r_tt}
\end{equation}

\begin{center} 
- - - - -
\end{center}

\noindent On établit les identités suivantes pour $\mathcal{J}$:

\begin{equation}
\parfrac{\mathcal{J}}{\beta_j} = \CP{j}{k}\bepar{\beta_k - \BP{k}} \label{j_b}
\end{equation}
Donc
\begin{equation}
\parfracD{\mathcal{J}}{\beta_i}{\beta_j}  = \CP{j}{k} \kro{i}{k} \label{j_bb}
\end{equation}
Et : \navy
\begin{equation}
\parfracD{\mathcal{J}}{T_j}{\beta_n} = 0 \label{j_tb} 
\end{equation}\bk

\begin{equation}
\parfrac{\mathcal{J}}{T_k} = \CO{k}{j}\bepar{T_j - d_j} \label{j_t}
\end{equation}
Donc
\begin{equation}
\parfracD{\mathcal{J}}{T_k}{T_n} = \CO{i}{k}\kro{n}{i} \label{j_tt}
\end{equation}

\pagebreak

\noindent On utilise ces résultats pour exprimer chacun des termes des équations \eqref{hijf} , \eqref{nu} et  \eqref{muim}.\\
On commence avec les termes de $\mu$. Le premier est nulle conformément à \eqref{j_tb}. Donc : \\
\begin{equation}
\psi_m \parfracD{\mathcal{R}_m}{\beta_i}{T_k} = -4\, \epsz \psi_m T_m^3 \kro{k}{m} \, \kro{n}{m}
\end{equation}
Puis 
\begin{equation}
\nu_{i,n} \parfracD{\mathcal{J}}{T_n}{T_k} = \nu_{i,n}\, \CO{i}{k}\kro{n}{i} = \nu_{i,i}\, \CO{i}{k}
\end{equation}
Et enfin
\begin{align}
\nu_{i,n} \psi_m \parfracD{\mathcal{R}_m}{T_n}{T_k} &= \nu_{i,n} \psi_m \bepar{-12\, \epsz \beta_m T_m^2 \kro{m}{k}\, \kro{m}{n}} \\
& = -12\, \epsz\, \nu_{i,m} \psi_m \beta_m T_m^2 \kro{m}{k}
\end{align}

\noindent On continue avec les termes de la Hessienne : le deuxième et le quatrième sont nuls conformément aux équations \eqref{rm_bb} et \eqref{j_tb}. On calcule :
\begin{equation}
\parfracD{\mathcal{J}}{\beta_i}{\beta_j} = \CP{j}{k} \kro{i}{k} = \CP{j}{i}
\end{equation}
Puis 
\begin{align}
\mu_{i,m}\frac{\partial \mathcal{R}_m}{\partial \beta_j} &= \mu_{i,m}\, \epsz\bepar{\tinf - T_m^4}\, \kro{j}{m} \\
&= \mu_{i,j}\, \epsz\bepar{\tinf - T_j^4}
\end{align}
Et enfin,
\begin{align}
\nu_{i,n} \psi_m \parfracD{\mathcal{R}_m}{T_n}{\beta_j} &= -4\, \epsz\, \nu_{i,n}\psi_m T_m^3\, \kro{m}{j} \,\kro{m}{n} \\
&= -4\, \epsz\, \nu_{i,j}\psi_j T^3_j\, \kro{j}{j}
\end{align}

\noindent On finit avec le calcule du terme $\nu_{i,n}$ à partir de l'équation \eqref{nu}
\begin{align}
\nu_{i,n} \bepar{ \bepar{\frac{\kro{m+1}{n} - 2\, \kro{m}{n} + \kro{m-1}{n}}{\Delta z^2}} -4T_m^3\, \epsz\, \kro{m}{n}} = -\epsz\bepar{\tinf-T_m^4}\kro{i}{m}
\end{align}
D'un point de vue matricielle, le membre de gauche est une matrice tridiagonale, celle de droite est diagonale ayant pour diagonale le vecteur $\Theta = T_m^4  \,\, \forall m \in \left[1, N\_{\text{discr} -2} \right]$ shifté de $\tinf$ et modulé par $\epsz$.
\end{document}
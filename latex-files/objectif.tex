\documentclass[a4paper,12pt]{article} 
\usepackage[utf8x]{inputenc}
\usepackage[french]{babel}
\usepackage{mathtools}
\usepackage{amsmath}
\usepackage{amsfonts}
\usepackage{amssymb}
\usepackage{graphicx}		 			% Inclusion des figures 
\usepackage{textcomp}
\usepackage[nointegrals]{wasysym}			% Collection de symboles mathématiques
\usepackage{multicol}					% Pour utiliser \hfill
\usepackage{ifthen}
\usepackage{tabularx}	 				% Gestion avancée des tableaux
%\usepackage{cleveref}

\usepackage{enumitem}
\usepackage{wrapfig}
%\usepackage[squaren]{SIunits}
%\usepackage[T1]{fontenc}				% Indispendable, présent dans tous les codes exemples
\usepackage[linkcolor=DarkRed,colorlinks=true, citecolor= DarkGreen, urlcolor=MidnightBlue]{hyperref} 	% Hyper ref
\usepackage{listings}					% Pour citer du code
\usepackage[justification=centering]{caption}
\usepackage{sistyle} 
\usepackage{numprint}
\usepackage{wrapfig}
\usepackage{cite}	
\usepackage{url} 					% Pour citer les sites internet dans la
%\usepackage{cleveref}
\usepackage{setspace}

\usepackage[svgnames]{xcolor}			%https://www.latextemplates.com/svgnames-colors

\newcommand{\bepar}[1]{
	\left( #1 \right)  
}

\newcommand{\becro}[1]{
	\left[ #1 \right]  
}

\usepackage{listings}					% Pour citer du code
%%%%%%%%%%%%%%%%%%%
%%% Élément pour citer des codes %%%
\lstset{
language=Python,
basicstyle=\ttfamily\bfseries\small, %
identifierstyle=\bfseries\color{black}, %
keywordstyle=\color{blue}, %
stringstyle=\color{black!90}, %
commentstyle=\it\color{green!95!yellow!1}, %
columns=flexible, %
tabsize=4, %
extendedchars=true, %
showspaces=false, %
showstringspaces=false, % %
numberstyle=\small, %
breaklines=true, %
breakautoindent=true, %
captionpos=b
}
%%%%%%%%%%%%%%%%%%%%%
\title{Avancées Numériques}%%%%%%%%%%%%%%%%%%%%
\date{}
\usepackage{multicol}
\usepackage{etoolbox}
\patchcmd{\thebibliography}{\section*{\refname}}
    {\begin{multicols}{2}[\section*{\refname}]}{}{}
\patchcmd{\endthebibliography}{\endlist}{\endlist\end{multicols}}{}{}

%% Pour avoir le check
%\usepackage{tikz}
%\def\checkmark{\tikz\fill[scale=0.4](0,.35) -- (.25,0) -- (1,.7) -- (.25,.15) -- cycle;}  



\usepackage{geometry}
\geometry{hmargin=2cm, vmargin=2cm}

%%%%%%%%%%%%%%%%%%%%
%%% Couleurs %%%
\xdefinecolor{brick}{named}{DarkRed}
\xdefinecolor{navy}{named}{Navy}
\xdefinecolor{midblue}{named}{MidnightBlue}
\xdefinecolor{dsb}{named}{DarkSlateGray}
\xdefinecolor{dgreen}{named}{DarkGreen}

%%% 	Raccourcis 		%%%
%% Couleurs
\newcommand{\keps}{$k-\varepsilon$}
\newcommand\bk{\color{black}}
\newcommand\brick{\color{brick}}
\newcommand\navy{\color{navy}}
\newcommand\midblue{\color{midblue}}
\newcommand\dsb{\color{dsb}}
\newcommand{\dgreen}{\color{dgreen}}

%% Commandes
\newcommand{\done}{\color{green} \textbf{\checkmark} \color{black}}
\newcommand{\todo}{\color{red} \textbf{$\times$} \color{black}}
\newcommand{\warn}{\color{orange} \textbf{$\Delta$} \color{black}}

%% Cigles
\newcommand{\rap}{par rapport }
\newcommand{\cad}{c'est-à-dire}

%%%%%%%% Autres

%%%%%%%%%%%%%%%%%%%
% Syntax: \colorboxed[<color model>]{<color specification>}{<math formula>}
\newcommand*{\colorboxed}{}
\def\colorboxed#1#{%
  \colorboxedAux{#1}%
}
\newcommand*{\colorboxedAux}[3]{%
  % #1: optional argument for color model
  % #2: color specification
  % #3: formula
  \begingroup
    \colorlet{cb@saved}{.}%
    \color#1{#2}%
    \boxed{%
      \color{cb@saved}%
      #3%
    }%
  \endgroup
}
\renewcommand{\sectionmark}[1]{\markright{#1}}
\usepackage{fancyhdr}
\pagestyle{fancy}
\lhead{\textbf{Nathaniel} \brick \textbf{\textsc{Saura}}}
\rhead{\markright}
\cfoot{\thepage}
\renewcommand{\headrulewidth}{0.4pt}

\numberwithin{equation}{section} %%%% To count the equation like Section.Number

%%%%%%%%%%%%%%%%%%%
\begin{document}
\maketitle
\section*{Point au 8/12/17}
\navy \subsection*{Fait et à faire } \bk
\begin{itemize}
	\item[1 - \done] Génération des champs de températures des problèmes exacte et inexacte. Enregistrement de ces champs, création de routine d'extraction de ces données ;
	\item[1 - \todo] Réfléchir à un moyen d'automatiser la récupération de l'écart type de ces données ainsi que les équivalents postérieurs aux calculs de \textsc{class\_temp\_ml} ; \\

	\item[2 - \done] Trois routines d'optimisations dont deux efficaces : \textsc{optimization} et \textsc{adjoint\_method} fonctionnant pour une covariance diagonale
	\item[2 - \todo] Comprendre les différences fondamentales entre ces deux méthodes, en quoi elles expliquent les différences sur les résultats.
	\item[- - \todo] Implémenter le générateur de $\beta_{\text{post}}$ et regarder les amplitudes possibles pour $\beta_{\text{map}}$ comme il a été fait pour \textsc{optimization}.
	\begin{itemize}
		\item[--] Calculer la hessienne inverse évaluée en $\beta_{\text{map}} $ et obtenir la covariance $C_{\text{map}}$
		\item[--] Décomposer avec Cholesky $C_{\text{map}}$ et obtenir le générateur $\beta_{\text{post}}$
		\item[--] Calculer les max et min puis tracer  \\
	\end{itemize}
	
	\item[3 - \done] Obtenir le $\beta$ optimisant le problème réduit par rapport au problème de référence ainsi que le $\sigma$ correspondant. \\
	Superposition de l'image de $\beta_{\text{map}}$ par le système défectueux, avec la solution du problème exacte. \\
	
	\item[\warn \warn] Optimiser des temps de calcul dans les différentes méthodes et routines \\
	Optimiser les chemins de sauvegardes et garder trace de graphes
	
\end{itemize}

\pagebreak

\section*{Idées au 19/03/18}
\navy \subsection*{Fait et à faire } \bk
\begin{itemize}
	\item[1 - ] RANS
	\begin{itemize}
		\item[--] RANS invariant Xiao méthode 
	\end{itemize}
	\item[1 - \todo] Réfléchir à un moyen d'automatiser la récupération de l'écart type de ces données ainsi que les équivalents postérieurs aux calculs de \textsc{class\_temp\_ml} ; \\

	\item[2 - \done] Trois routines d'optimisations dont deux efficaces : \textsc{optimization} et \textsc{adjoint\_method} fonctionnant pour une covariance diagonale
	\item[2 - \todo] Comprendre les différences fondamentales entre ces deux méthodes, en quoi elles expliquent les différences sur les résultats.
	\item[- - \todo] Implémenter le générateur de $\beta_{\text{post}}$ et regarder les amplitudes possibles pour $\beta_{\text{map}}$ comme il a été fait pour \textsc{optimization}.
	\begin{itemize}
		\item[--] Calculer la hessienne inverse évaluée en $\beta_{\text{map}} $ et obtenir la covariance $C_{\text{map}}$
		\item[--] Décomposer avec Cholesky $C_{\text{map}}$ et obtenir le générateur $\beta_{\text{post}}$
		\item[--] Calculer les max et min puis tracer  \\
	\end{itemize}
	
	\item[3 - \done] Obtenir le $\beta$ optimisant le problème réduit par rapport au problème de référence ainsi que le $\sigma$ correspondant. \\
	Superposition de l'image de $\beta_{\text{map}}$ par le système défectueux, avec la solution du problème exacte. \\
	
	\item[\warn \warn] Optimiser des temps de calcul dans les différentes méthodes et routines \\
	Optimiser les chemins de sauvegardes et garder trace de graphes
	
\end{itemize}

\end{document}
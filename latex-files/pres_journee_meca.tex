\documentclass[10pt,
			   xcolor=svgnames,
			   hyperref={linkcolor=red, citecolor = DarkGreen, colorlinks=true, urlcolor=Navy}] {beamer}
			   
\usepackage[english]{babel}
\usepackage[utf8x]{inputenc}

\usepackage{tikz}
\def\checkmark{\tikz\fill[scale=0.4](0,.35) -- (.25,0) -- (1,.7) -- (.25,.15) -- cycle;}

\mode<presentation>
{
  \usetheme{Copenhagen}      % or try Darmstadt, Madrid, Warsaw, ...
  \usecolortheme{beaver}
} 

\usepackage{multicol}
\setlength{\columnseprule}{0.pt}

\usepackage{float}
\usepackage[authoryear]{natbib}

\graphicspath{{./pic/}}

\usepackage[font=scriptsize, center]{caption}

%%% Commandes utiles définies
\newcommand{\argmin}{\mathop{\mathrm{argmin}}}

\newcommand{\bepar}[1]{
	\left( #1 \right)  
}
\newcommand{\becro}[1]{
	\left[ #1 \right]  
}

\newcommand{\norm}[1]{
	\left \vert \left \vert #1 \right \vert  \right \vert
}

\usepackage{siunitx}

\usepackage{pifont}
\newcommand{\bwarrow}{\item[\color{DarkRed} \ding{227}]}
\newcommand{\warrow}{\item[\color{blue!50!black!70} \tiny{\ding{109}}]}
\newcommand{\sarrow}{\item[\color{blue!50!black!70!orange!60} \tiny{\ding{55}}]}


\usepackage{geometry}
\geometry{hmargin=1.cm, vmargin=0cm}

\xdefinecolor{bviolet}{named}{BlueViolet}

%%%%%%%%%%%%%%%%%%%%
%%% Couleurs %%%
\xdefinecolor{brick}{named}{DarkRed}
\xdefinecolor{navy}{named}{Navy}
\xdefinecolor{midblue}{named}{MidnightBlue}
\xdefinecolor{dsb}{named}{DarkSlateGray}
\xdefinecolor{dgreen}{named}{DarkGreen}

%%% 	Raccourcis 	%%%
\newcommand{\keps}{$k-\varepsilon$}
\newcommand\bk{\color{black}}
\newcommand\brick{\color{brick}}
\newcommand\navy{\color{navy}}
\newcommand\midblue{\color{midblue}}
\newcommand\dsb{\color{dsb}}
\newcommand{\dgreen}{\color{dgreen}}
\newcommand\red{\color{red}}

\usepackage{setspace}

    \expandafter\def\expandafter\insertshorttitle\expandafter{%
       \insertshorttitle\hfill%
       \insertframenumber\,/\,\inserttotalframenumber}

% Things for first slide :
\title[Machine learning in Physics fields]{An attempt to learn Physics with Neural Networks}
\author{Saura Nathaniel (PHD Student)\\}

\date{$21^{\text{st}}$ November 2018}
\institute{\bf LMFL
\begin{center}
	\begin{minipage}[!ht]{0.9\textwidth}
	\centering
	\includegraphics[scale=0.8]{logo_lmfl_border.png}
	\end{minipage}
\end{center}
}

% Begin 
\begin{document}
\begin{frame}
  \titlepage
\end{frame}

%\begin{frame}{Laminaire, transient and Turbulent flows}{Graphs and equations}
%\begin{figure}[!ht]
%\centering
%\includegraphics[scale=0.2]{Reynolds_observations_turbulence_1883.png}
%\caption{}
%\label{}
%\end{figure}
%\end{frame}

\begin{frame}{Introduction}{Big data in fluid mechanics}
	Within the last 50 years, datasets covering various fluid mechanics aspects have been built. Those are built from
	\begin{itemize}
		\warrow Numerical results :	
		\begin{itemize}
			\sarrow \normalsize High fidelity datasets from Direct Numerical Simulation \normalsize or Lattice Boltzmann method\\[1mm]
			\sarrow Lower fidelity datasets (Large Eddy Simulation ...) 
		\end{itemize}			
		\warrow Experimental results (condidered to be the ground truth)
	\end{itemize}
Computers being more and more powerful, new datasets are frequently added to this databank.	\\[1mm]
	
	\begin{block}{Boom of data driven methods}
		Data driven method denomination embraces every method that relies on prior knowledge from data. \footnote{To a certain extent, Physics can be seen as data-driven.}
	\end{block}
\end{frame}

\begin{frame}{Introduction}{Data-Driven methods and Supervised Machine learning}
	\begin{block}{Data-driven methods}
		It consists in infering \textit{latent variables} \textbf{h} of a model \textbf{m} behind some \textit{visible variables} \textbf{v} datas or at least maximize the conditionnal probability $$p\bepar{h|\,v}$$
		We call this procedure inference or Bayesian Inversion
	\end{block}
	
	\begin{block}{Supervised Machine Learning}	
		It consists in tuning set of parameters $\omega$ to match predefined input $\eta$ and output $y$ pairs called \textbf{train dataset} in the aim to extend this knowledge for prediction purposes. In other words, we want to maximize 
					$$ p(\omega|\, \eta_{\text{train}}, y_{\text{train}})$$
	\end{block}	
	
\end{frame}	

\begin{frame}{Introduction}{Precision or cost computation ?}
	 Current powerful computers allow to investigate highly complex Turbulent flows. Then two choices 
	\begin{itemize}
		\warrow Consider the whole Physics of the problem
		\begin{itemize}
			\item[\dgreen \checkmark] High fidelity results
			\sarrow Need a huge amount of time 
		\end{itemize}
		
		\warrow Simplify the Physics of the problem (coarse grained modeling)
		\begin{itemize}
			\item[\dgreen \checkmark] Results faster obtained
			\sarrow May be not accurate enough
		\end{itemize}
	\end{itemize}
	\vspace{2mm}
	\begin{exampleblock}{Solution : both of them}
		Correct low fidelity but quick model to inject high fidelity information from previous DNS or experimental dataset.\\
	\end{exampleblock}	
	\end{frame}

%\begin{block}{Well known question : cost computation or precision ?}
%\begin{itemize}
%\item[$\bullet$] DNS provides perfect solutions but needs a lot of time computation
%\item[$\bullet$] RANS models provide quicker more or less precise results 
%\end{itemize} 
%\end{block}
%
%\begin{block}{ML algorithms exploding development (in every field)}
%\begin{itemize}
%\item[$\bullet$] Gaussian Process
%\item[$\bullet$] RForests, NNetworks (Supervised) for classification or regression tasks (more and more papers in fluid mechanics)
%\item[$\bullet$] Reinforcement Learning (few papers in fluid mechanics so far)
%\end{itemize}
%\end{block}


\begin{frame}{Bayesian inference in RANNS (low fidelity) models}
	\begin{block}{RANNS models : consider mean flow and recover fluctuation flow}
		The main idea of RANS models is to aonly consider the mean flow \textit{i.e.} large scale. To recover the fluctuation flow, we add closure(s) equation(s) of the forme.  
		\center{$\displaystyle \mathcal{F}\bepar{\bk P_{\text{roduction}}  , D_{\text{issipation} }, T_{\text{ransport}}} = 0 $}
	\end{block}
 
\begin{block}{Baysian inference problem}
\begin{itemize}
	\item[$\bullet$] We add a term $\color{BlueViolet} \beta\bepar{\textbf{x}, t} \bk$ to \textit{drive} the RANNS solution according to high fidelity information\\
	\center{$\displaystyle \mathcal{F}\bepar{\color{BlueViolet} \beta\bepar{\textbf{x}, t} \bk P_{\text{roduction}}  , D_{\text{issipation} }, T_{\text{ransport}}} = 0 $} 
\flushleft	
\end{itemize}
\end{block} 

%\tableofcontents%

\end{frame}

\begin{frame}{Baysian inference problem}
	\begin{itemize}
		\warrow Find $\beta$ as the vector maximizing $p\bepar{\beta|\, d_{\text{\textbf{Obs}}}}$  
		\warrow Equivalent to minimize the cost function 
		\vspace{-5.5mm}
		\center{$$\mathcal{J} = \becro{\frac{1}{2} \bepar{d_{\text{\textbf{Obs}}} - h\bepar{\beta}}^T\textbf{C}^{-1}_m\bepar{d_{\text{\textbf{Obs}}} - h\bepar{\beta}} + \bepar{\beta - \beta_{\text{p}}}^T\textbf{C}_\beta^{-1}\bepar{\beta - \beta_{\text{p}}}}$$
}
	\end{itemize}

	\vspace{2mm}
	$\mathcal{J}$ distance between low and high fidelity model solutions, with
	
	\begin{multicols}{2}
	\noindent

	\begin{itemize}
		\warrow $\textbf{C}^{-1}_m$ inverse of the true data covariance 
		\warrow $\textbf{C}^{-1}_\beta$ inverse of the prior data covariance 
	\end{itemize}

	\columnbreak

	\begin{itemize}
		\warrow $d_{\text{\textbf{Obs}}}$ one of the high fidelity field (\textit{e.g. $\textbf{U}$ )}
		\warrow $h\bepar{\beta}$ same low fidelity field (now depends on $\beta$)    
	\end{itemize}	
	\end{multicols}	

\textit{Maximum A Posteriori} method : gives through an iterative process $\beta_M$  
$$ \beta_M = \argmin \mathcal{J} $$

\end{frame}

\begin{frame}{Example of inference : Thermal problem}
	Begin with a simple problem	
	\begin{block}{Real and Model equations}
		\begin{itemize}
			\item[$\bullet$] Real equation :
			\begin{center}
				$\displaystyle \frac{d^2T}{dz^2} + \varepsilon(T)\bepar{T^4_\infty - T^4} + h\bepar{T_\infty - T} = 0$
			\end{center}
			\item[$\bullet$] Model equation :
			\begin{center}
				$\displaystyle \frac{d^2T}{dz^2} + \varepsilon_0\color{BlueViolet}\beta(z)\bk\bepar{T^4_\infty - T^4}= 0$
			\end{center}
		\end{itemize} 
	\end{block}
	
	\begin{itemize}
		\item[\dgreen \checkmark] BFGS method to minimize $\mathcal{J}$ $\rightarrow$ $\beta_{\text{MAP}}$ \& $\text{Hess}^{-1}_{\beta_{\text{MAP}}}$
		\item[\dgreen \checkmark] Create $\beta_{\text{final}}$ distribution : \\
		\begin{center}
			$ \beta_{\text{final}} \equiv \beta \sim \mathcal{N}\bepar{\beta_{\text{MAP}}, {\text{Hess}^{-1}_{\beta_{\text{MAP}}}}}$
		\end{center}
	\end{itemize}
\end{frame}

\begin{frame}{BIF : Few figures based on inversion ($T_\infty = \ang{15}$C)}
	\begin{multicols}{2}
		\noindent
		\begin{figure}[H]
			\centering
			\includegraphics[scale=0.25]{Pres_Evolution_beta_map.png}
			\caption{Evolution of beta through the minimization process}
		\end{figure}

	\columnbreak
	
		\vspace*{-1cm}
		\begin{figure}[H]
			\centering
			\includegraphics[height=4cm, width=6cm]{Pres_T_15_Full_Beta_Comp.png}
		\end{figure}
	
		\vspace*{-2.5cm}
		\begin{figure}[H]
			\centering
			\includegraphics[height=4cm, width=6cm]{Evolution_de_l'erreur_T_inf_15.png}
		\end{figure}		

	\end{multicols}
\end{frame}

\begin{frame}{(BIF) Viscous Burgers Equation 1D}
	More complex inference problem
	\begin{block}{Equations}
		\begin{itemize}
			\item[$\bullet$] Real equation :
				\begin{center}
					$\displaystyle \frac{\partial u}{\partial t} + u \frac{\partial u}{\partial x} - \nu\frac{\partial^2 u}{\partial x^2} = 0$
				\end{center}
			\item[$\bullet$] (Inference) Model equation (1) :
				\begin{center}
					$\displaystyle \frac{\partial u}{\partial t} + u \color{BlueViolet} \beta(x,t) \bk - \nu \frac{\partial^2 u}{\partial x^2} = 0$
				\end{center}
		\end{itemize} 
	\end{block}

	\begin{itemize}
		\item[\checkmark] Real solution obtained with Lax Wendroff Scheme
		\item[\checkmark] Infered solution obtained with Crank Nicholson Scheme
		\red \item[\red \checkmark \bk] One inference at each time iteration \bk
	\end{itemize}

\end{frame}

\begin{frame}{(BIF) VBE 1D : Inference step}
	\begin{block}{A look at the future}
		At each iteration $n$ we compute the $\beta_{\text{MAP}}$ minimizing a cost function that depends on the solution at $n+1$ : \\ 
		\begin{center}
			$\displaystyle \mathcal{J}^n = \frac{1}{2} \bepar{\Delta_{_{\text{LW-CN}}}U^{n+1}}^T \text{C}_\text{obs}^{-1} \bepar{\Delta_{_{\text{LW-CN}}}U^{n+1}} + \lambda \bepar{\beta^n -
			\beta_{\text{p}}}^T \text{I}_d \bepar{\beta^n - \beta_\text{p}}$
		\end{center}
		Where \begin{center}$\Delta_{_{\text{LW-CN}}}U^{n+1} = \bepar{U^{n+1}_{\text{obs}}}^\text{LW} - \bepar{U^{n+1}_\beta}^\text{CN}$ \end{center}
		And \\
		\begin{center}
			$\displaystyle \beta_{\text{MAP}}^n = \argmin \mathcal{J}^n$
		\end{center}
	\end{block}
	
	\begin{itemize}
		\item[\checkmark] Initialize the problem with $u_0(x,t) = \sin((2\pi x)/L)$ in L-length domain
		\item[\checkmark] Periodic boundary condition
	\end{itemize} 
\end{frame}

\begin{frame}{(BIF) VBE 1D : Inference step (figures)}
\vspace{-1cm}
\begin{multicols}{2}
\noindent
	\begin{figure}[H]
		\centering
		\includegraphics[width = 6cm, height= 6cm]{nu0_0250_CFL0_40_Nx_52_InferenceVSTrue_it5.png}
		\vspace{-0.5cm}	
		\caption{Left : Beta at time $n$ Right : Comparaison between reel solution and inferred solution at iteration 1(red) and 6(blue)}
	\end{figure}

\columnbreak

	\begin{figure}[H]
		\centering
		\includegraphics[width = 6cm, height=6cm]{nu0_0250_CFL0_40_Nx_52_InferenceVSTrue_it35.png}
		\vspace{-0.5cm}	
		\caption{Left : Beta at time $n$ Right : Comparaison between reel solution and inferred solution at iteration 31(red) and last(blue)}
	\end{figure}

\end{multicols}

\end{frame}

\begin{frame}{Feed Forward Neural Network (FFNN)}
		\begin{itemize}
			\warrow Supervised Machine learning algorithm
			\warrow Graph from Inputs to Outputs 
			\warrow In bitween : architectures (layers abb. L) and operations (activation functions, matricial products)
		\end{itemize}

	\begin{multicols}{2}
	\noindent 
	
	\begin{figure}[!ht]
		\centering
		\includegraphics[width=5.5cm, height=3.5cm]{Multi-layer-perceptron.png}	
	\end{figure}	
	
	\columnbreak

		\begin{itemize}
			\sarrow Input L connected to hidden L connected to hidden L until Output L
			\sarrow Hidden layer (HL) composed of hidden nodes (HN)			
		\end{itemize}	
	\end{multicols}

\end{frame}

\begin{frame}{FFNN : light focus on hidden nodes}
	\begin{multicols}{2}
	
		\noindent		
		\begin{figure}[!ht]
		\centering
		\includegraphics[scale=0.4]{hidden_nodes.png}
		\end{figure}
		
		\columnbreak

		\begin{itemize}
			\sarrow Each HN apply $\text{Output}_\text{HN} = \text{act}\bepar{\sum_j w_{ij}\,{\eta_j}_{\text{HN}}}$ 
			\sarrow act function add non linearity to transorm inputs into outputs
			
		\end{itemize}
		\vspace{-4mm}
\begin{figure}[!ht]
		\centering
		\includegraphics[scale=0.23]{The-most-common-nonlinear-activation-functions.png}
		\end{figure}		
		
	\end{multicols}

\end{frame}

\begin{frame}{Training of a Neural Network}
	\begin{itemize}
		\item[$\bullet$] Two phases : construction phase (define archtectures), execution phase
		\item[$\bullet$] Excecution phase : training phase \& testing phase \\[2mm]
	\end{itemize}
 
Consider X input matrice and y target matrice and split them into 
	\begin{multicols}{2}
		\begin{itemize}
			\warrow $\mathcal{D}_\text{train}$ = $\bepar{X_{\text{train}},\, y_{\text{train}}}$ 
		\end{itemize}				
		
	\columnbreak	
		\begin{itemize}		
			\warrow $\mathcal{D}_\text{test}$ = $\bepar{X_{\text{test}},\, y_{\text{test}}}$
		\end{itemize}
	\end{multicols}
	
	\begin{itemize}
		\sarrow Training phase : optimize (through Gradient Descent algorithm) HL weights to maximize the correspondance between $X_{\text{train}},\, y_{\text{train}}$ \\
		\sarrow Testing phase : evaluate the generalization capacity of the NN on $\mathcal{D}_\text{test}$ \\[3mm]
		\item[\checkmark] If testing phase is validated we can go beyond and use the NN with the final weights to predict
	\end{itemize}
\end{frame}	

\begin{frame}{How it can be used}
	\begin{itemize}
		\warrow First thing is to find the good architecure
	\end{itemize}
	\vspace{-3mm}
	\begin{multicols}{2}
	Classic one
	\vspace{-8mm}	
		\begin{figure}[!ht]
			\centering
			\includegraphics[scale=0.3]{1bCQl.png}
		\end{figure}

	\columnbreak
	AutoEncoder
	\vspace{-8mm}	
		\begin{figure}[!ht]
			\centering
			\includegraphics[scale=0.23]{Autoencoder_structure.png}
	\end{figure}	
	
	\end{multicols}
		\vspace{-9mm}	
	More complicated one 
	\vspace{-5mm}	
	\begin{figure}[!ht]
		\centering
		\includegraphics[width=11.2cm, height=2.5cm]{model_plot.png}
	\end{figure}
	
\end{frame}
%\begin{frame}{Bayesian Inversion Framework (BIF)}
%Framework proposed by Duraisamy \textit{et. al} in \citep{parish2016paradigm}
%\begin{multicols*}{2}
%\noindent
%	\begin{figure}[!ht]
%	\centering
%	\includegraphics[scale=0.3]{singh.png}
%	\caption{Figure extracted from \citep{singh2017machine}}
%	\end{figure}
%	
%	\columnbreak
%
%	\begin{itemize}
%	\item[1 --] Infer descrepancy \color{BlueViolet} \textit{information} \bk $\beta$ as the MAP solution \\[5mm]
%	
%	\item[2 --]	\textit{Generalize} this information using ML methods to have a \color{BlueViolet} \textit{knowledge} \bk: $\beta = \mathcal{F}\bepar{\eta_1, \eta_2,...}$ \\[5mm]
% 
%	\item[3 --] Inject ML predicted correction into model's equation(s) \\[5mm]
%	
%	\end{itemize}
%			
%\end{multicols*}
%\end{frame}

%
%
%
%\begin{frame}{BIF : Next Step : Machine Learning}
%Reproduce the previous calculs for T = 5:5:50 \\[1cm]
%
%Construction $X$ and $y$ in $\mathcal{D} = \left\lbrace \ \underbrace{\bepar{T_{\text{inf}},T(x_i)}}_{X_i}, \ \underbrace{\beta_i}_{y_i}\ \right\rbrace$ for each T.\\
%In matrix notation : 
%
%\begin{multicols}{2}
%\noindent
%$$ X = \left[ \begin{array}{c} \cdot \\ \cdot \\ T_{\text{inf}},\ T(x) \\ \cdot \\ \cdot
%			  \end{array}
%	   \right]
%$$
%\columnbreak
%$$ y = \left[ \begin{array}{c} \cdot \\ \cdot \\ \beta(x) \\ \cdot \\ \cdot
%			  \end{array}
%	   \right]
%$$
%\end{multicols}
%Neural Network generalization : \\
%\begin{center}
% $\beta_1(x), ..., \beta_n(x) \rightarrow \beta_{\text{ML}} = \mathcal{F}\bepar{\eta_1 = T_{\text{inf}}, \eta_2= T}$
%\end{center}
%\end{frame}
%
%\begin{frame}{(BIF) : Prediction with NN }
%$T_\infty = \ang{28}$C 
%\begin{figure}[H]
%	\centering
%	\includegraphics[scale=0.5]{T_True_vs_T_ML_N_sample_5_T_inf_28_Adam-mean.png}
%\end{figure}
%
%\end{frame}
%
%\begin{frame}
%$T_\infty = \ang{55}$C 
%\begin{figure}[H]
%	\centering	
%	\includegraphics[scale=0.5]{TBeta_True_vs_TBeta_ML_N_sample_=_5_T_inf_=_55.png}
%\end{figure}
%
%\end{frame}
%
%\begin{frame}
%$T_\infty = 35 + 20\sin(2\pi x)$ 
%\begin{figure}[H]
%	\centering	
%	\includegraphics[scale=0.5]{T_True_vs_T_ML_N_sample_5_T_inf_35_+_20sin_2piz_adam-mean.png}
%\end{figure}
%\end{frame}
%
%
%
%\begin{frame}{(BIF) VBE 1D : Machine Learning step}
%	More difficult now : what are $\eta$ to use in $\beta = \mathcal{F}(\eta)$ ?
%	\begin{itemize}
%		\item[$\bullet$]$\beta^{n+1} = \bepar{U_{i-1}^n, U_i^n, U^n_{i+1}}$ 
%		
%		\item[$\bullet$] $\beta^{n+1} = \bepar{X_{i-1}^n, X_i^n, X^n_{i+1}, U_{i-1}^n, U_i^n, U^n_{i+1}}$ 		
%		
%		\item[$\bullet$] $\beta^{n+1} = \bepar{X_{i-1}^n, X_i^n, X^n_{i+1}, 
%		U_{_{\bk i-1}}^{\red n-1}, U_{\bk i}^{\red n-1}, U_{_{\bk i+1}}^{\red n-1}, 
%		U_{i-1}^n, U_i^n, U^n_{i+1}}$ 		
%		
%		\item[$\bullet$] \scriptsize$\beta^{n+1} = \bepar{X_{i-1}^n, X_i^n, X^n_{i+1}, 
%		U_{_{\bk i-1}}^{\navy n-2}, U_{\bk i}^{\navy n-2}, U_{_{\bk i+1}}^{\navy n-2},
%		U_{_{\bk i-1}}^{\red n-1}, U_{\bk i}^{\red n-1}, U_{_{\bk i+1}}^{\red n-1}, U_{i-1}^n, U_i^n, U^n_{i+1}}$ 		\\[0.54cm]
%	\end{itemize}
%	\normalsize
%
%	Problem linked to this framework :
%	\begin{itemize}
%		\sarrow Inversion can take long time to give $\beta_{\text{MAP}}$
%		\sarrow The need to produce various inferences to construct wide datasets
%	\end{itemize}
%	
%	\begin{exampleblock}{Idea}
%		Can we use NN brute force to infer the behaviour of U through time ?
%	\end{exampleblock}	
%	No more inference needed
%\end{frame}
%
%\section{Neural network brute force (NNBF)}
%\begin{frame}{Building the datasets}
%
%	\begin{itemize}
%		\item[$\bullet$] Different sinusoides as initial conditions
%
%		\begin{itemize}
%			\sarrow Different amplitudes (between 1 and 2) -- 15 cases
%			\sarrow Random phases
% 		\end{itemize}
%	
%	\begin{figure}[H]
%		\vspace{-0.3cm}		
%		\centering
%		\includegraphics[scale=0.3]{Initialisation_cases.png}
%	\end{figure}			
%		
% 		\item[$\bullet$] Lax Wendroff scheme to solve each initial condition
%	\end{itemize}
%\end{frame}
%
%\begin{frame}{Strategy employed here}
%	We want to find the functionnal linking the solution $U^{n+1}$ and some n$^{\text{th}}$ parameters :\\
%	\begin{center}
%	$ \displaystyle U^{n+1} = \mathcal{F}\bepar{\eta^n}$ \\
%	\end{center}
%	Construction $X$ and $y$ iteration by iteration, case by case.\\
%
%	\begin{multicols}{2}
%	\noindent
%	$$ X = \left[ \begin{array}{c} \cdot \\ \cdot \\ U^n_{i-1},U^n_{i}, U^n_{i+1}, \frac{U^n_{i+1} - U^n_{i-1}}{2\Delta x}  \\ \cdot \\ \cdot
%				  \end{array}
%		   \right]
%	$$
%	\columnbreak
%	$$ y = \left[ \begin{array}{c} \cdot \\ \cdot \\ U^{n+1}_i \\ \cdot \\ \cdot
%			  \end{array}
%	   \right]
%	$$
%	\end{multicols}
%	
%	\begin{itemize}
%		\item[\checkmark] $\displaystyle \frac{U^n_{i+1} - U^n_{i-1}}{2\Delta x}$ to better catch chocs \\
%		\item[\checkmark] Each line of X describes the local velocity field around the i$^{\text{th}}$ point and its local variation linked with the value of the velocity at the same i$^{\text{th}}$ point
%	\end{itemize}		
%	
%\end{frame}
%
%\begin{frame}{Training of the algorithm}
%	\begin{figure}[H]
%		\centering
%		\includegraphics[scale=0.45]{Cost_Evolution_:_loglog_and_lin.png}
%	\end{figure}
%\end{frame}
%
%\begin{frame}{Graph architecture}
%	Graph with \red 6 hidden layers \bk with \red 80 hidden nodes \bk in each. Adam optimizer (better and the most used in literature) and \navy Lasso \bk cost function.\\
%	
%	TensorFlow graph that can be visualized with Tensorboard :
%	
%	\begin{multicols}{2}	
%		\noindent		
%		\begin{figure}[H]
%		\centering
%		\includegraphics[scale=0.2]{Graphs.png}
%		\end{figure}
%	
%	\columnbreak
%		\centering
%		\vspace{1cm}
%		\begin{itemize}
%			\sarrow \footnotesize Lasso = $\norm{ y_{\text{tr}} - y_{\text{pred}}}^2_{L_2} + \lambda \norm{w}_{L_1}$		 \\[1.5cm]
%		
%			\sarrow Activation function : selu
%			\begin{figure}[H]
%			\centering
%			\includegraphics[scale=0.3]{SELU.png}
%			\caption{From Wikipedia}
%			\end{figure}
%		
%		\end{itemize}
%	\end{multicols}
%\end{frame}
%
%\begin{frame}{Brute Force on VBE 1D I}
%	New initial velocity field (first we deal with the \color{orange} orange \bk one) : 
%	\begin{figure}[H]
%	\centering
%	\includegraphics[scale=0.4]{Initialisation_cases_Andu1.png}
%	\end{figure}
%\end{frame}
%
%\begin{frame}{Brute Force on VBE 1D I}
%	\begin{figure}[H]
%	\centering
%	\includegraphics[scale=0.4]{Pres_First_Iteration_1.png}
%	\end{figure} 
%\end{frame}
%
%\begin{frame}{Brute Force on VBE 1D I}
%	\begin{figure}[H]
%	\centering
%	\includegraphics[scale=0.5]{Pres_Tenth_Iteration_1.png}
%	\end{figure} 
%\end{frame}
%
%\begin{frame}{Brute Force on VBE 1D I}
%	\begin{figure}[H]
%	\centering
%	\includegraphics[scale=0.35]{Pres_50th_Iteration_1.png}
%	\end{figure} 
%\end{frame}
%
%\begin{frame}{Brute Force on VBE 1D I}
%	\begin{figure}[H]
%	\centering
%	\includegraphics[scale=0.42]{Pres_Last_Iteration_1.png}
%	\end{figure} 
%\end{frame}
%
%\begin{frame}{Brute Force on VBE 1D II}
%	New initial velocity field (we deal now with the \color{cyan} cyan \bk one) : 
%\vspace{-0.3cm}	
%	\begin{figure}[H]
%	\centering
%	\includegraphics[scale=0.5]{Initialisation_cases_Last.png}
%	\end{figure}
%\end{frame}
%
%\begin{frame}{Brute Force on VBE 1D II}
%	\begin{figure}[H]
%	\centering
%	\includegraphics[scale=0.5]{Pres_First_Iteration_2.png}
%	\end{figure} 
%\end{frame}
%
%\begin{frame}{Brute Force on VBE 1D II}
%	\begin{figure}[H]
%	\centering
%	\includegraphics[scale=0.5]{Pres_Tenth_Iteration_2.png}
%	\end{figure} 
%\end{frame}
%
%\begin{frame}{Brute Force on VBE 1D II}
%	\begin{figure}[H]
%	\centering
%	\includegraphics[scale=0.5]{Pres_50th_Iteration_2.png}
%	\end{figure} 
%\end{frame}
%
%\begin{frame}{Brute Force on VBE 1D II}
%	\begin{figure}[H]
%	\centering
%	\includegraphics[scale=0.5]{Pres_Last_Iteration_2.png}
%	\end{figure} 
%\end{frame}
%
%\begin{frame}{Conclusion on Brute Force}
%VBE is not the simpliest equation to solve since sharp chocs
%	\begin{block}{Still can be improved}
%		\begin{itemize}
%			\item[$\bullet$] Good approximations but can be better
%			\item[$\bullet$] Accumulative errors phenomena that impose to differ local error (on the on going iteration) and global errors (accumulated through time iterations)
%			\item[$\bullet$] Genetic Algorithm can be used to get the best architecture but the possible features are infinite \\[0.5cm]
%		\end{itemize}
%	\end{block}
%
%	\begin{exampleblock}{Solution : Learn !}
%	The solution is to use Reinforcement learning to learn the physics
%	\begin{itemize}
%		\item[\checkmark] Auto ajust weights given an environment 
%		\item[\checkmark] Ensure to respect conservation physics law
%		\item[\checkmark] Auto adjust global and local error
%	\end{itemize}
%	\end{exampleblock}
%\end{frame}
%
%\begin{frame}{OnGoing Deep ReiL Framework}
%	
%	\begin{itemize}
%		\item[$\bullet$] Deep Reinforcement Learning (DQL) code in ealy stage of test.
%		\item[$\bullet$] A strategy has been established and wait to be tested
%		
%		\item[$\bullet$] More soon .. ;)
%		
%	\end{itemize}

%\end{frame}
\begin{frame}
\begin{center}
	\color{FireBrick} \textbf{\textit{\Large{Thank you for your attention}}} \bk
\end{center}
\end{frame}
%
%\begin{frame}
%\bibliographystyle{apalike}
%\bibliography{bibliotheque}
%\end{frame}

\end{document}
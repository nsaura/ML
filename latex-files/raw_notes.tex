\documentclass[a4paper,12pt]{article} 
\usepackage[utf8x]{inputenc}
\usepackage[french]{babel}
\usepackage{mathtools}
\usepackage{amsmath, amssymb, amsfonts}
\usepackage{textcomp}
\usepackage[nointegrals]{wasysym}			% Collection de symboles mathématiques
\usepackage{ifthen}
\usepackage{tabularx}	 				% Gestion avancée des tableaux
\usepackage{longtable}		
%\usepackage{cleveref}

\usepackage{enumitem}
\usepackage{wrapfig}
%\usepackage[squaren]{SIunits}
%\usepackage[T1]{fontenc}				% Indispendable, présent dans tous les codes exemples
\usepackage[linkcolor=red,colorlinks=true, citecolor=SaddleBrown, urlcolor=MidnightBlue]{hyperref} 	% Hyper ref
\usepackage{listings}					% Pour citer du code
\usepackage[justification=centering]{caption}
\usepackage{sistyle} 
\usepackage{numprint}
\usepackage{wrapfig}
\usepackage{cite}	
\usepackage{url} 					% Pour citer les sites internet dans la
%\usepackage{cleveref}
\usepackage{setspace}

\usepackage{graphicx}		 			% Inclusion des figures
\graphicspath{{./pic/}}
\usepackage[svgnames]{xcolor}			%https://www.latextemplates.com/svgnames-colors

%%% Commandes utiles définies%
\newcommand{\argmin}{\mathop{\mathrm{argmin}}}

\newcommand{\bepar}[1]{
	\left( #1 \right)  
}

\newcommand{\becro}[1]{
	\left[ #1 \right]  
}

\newcommand{\beacc}[1]{
	\left\{ #1 \right \}  
}

\newcommand{\norm}[1]{
	\left \vert \left \vert #1 \right \vert  \right \vert
}

\newcommand{\uin}[2]{
	\underline{\texttt{u}}_{#1}^{#2}
}
\newcommand{\ui}[1]{
	\underline{\texttt{u}}_{#1}^{n}
}
\newcommand{\dij}[1]{
	\delta_{#1,j}
}
\usepackage{listings}					% Pour citer du code
%%%%%%%%%%%%%%%%%%%
%%% Élément pour citer des codes %%%
\lstset{
language=Python,
basicstyle=\ttfamily\bfseries\small, %
identifierstyle=\bfseries\color{black}, %
keywordstyle=\color{blue}, %
stringstyle=\color{black!90}, %
commentstyle=\it\color{black!70}, %
columns=flexible, %
tabsize=4, %
extendedchars=true, %
showspaces=false, %
showstringspaces=false, % %
numberstyle=\small, %
breaklines=true, %
breakautoindent=true, %
captionpos=b,
otherkeywords={cross_val_score},
keywords=[0]{cv},
keywordstyle=[0]{\color{red}},
}
%%%%%%%%%%%%%%%%%%%%%
\title{\navy \textbf{Raw Notes from Articles: \color{black}}}%%%%%%%%%%%%%%%%%%%%
\date{}
%\usepackage{multicol}
%\usepackage{etoolbox}
%\patchcmd{\thebibliography}{\section*{\refname}}
%    {\begin{multicols}{2}[\section*{\refname}]}{}{}
%\patchcmd{\endthebibliography}{\endlist}{\endlist\end{multicols}}{}{}
\usepackage[authoryear]{natbib}
\usepackage{geometry}
\geometry{hmargin=2cm, vmargin=2cm}

%%%%%%%%%%%%%%%%%%%%
%%% Couleurs %%%
\xdefinecolor{brick}{named}{DarkRed}
\xdefinecolor{navy}{named}{Navy}
\xdefinecolor{midblue}{named}{MidnightBlue}
\xdefinecolor{dsb}{named}{DarkSlateGray}
\xdefinecolor{dgreen}{named}{DarkGreen}
\xdefinecolor{indian}{named}{IndianRed}

%%% 	Raccourcis 	%%%
\newcommand{\keps}{$k-\varepsilon$}
\newcommand\bk{\color{black}}
\newcommand\brick{\color{brick}}
\newcommand\navy{\color{navy}}
\newcommand\midblue{\color{midblue}}
\newcommand\dsb{\color{dsb}}
\newcommand{\dgreen}{\color{dgreen}}
\newcommand{\dpurple}{\color{indian}}
\newcommand\red{\color{red}}

%%%%%%%% Cigles
\newcommand{\rap}{par rapport}
\newcommand{\cad}{c'est-à-dire}
\newcommand{\vav}{vis-à-vis}

%%%%%%%% Autres

%%%%%%%%%%%%%%%%%%%
% Syntax: \colorboxed[<color model>]{<color specification>}{<math formula>}
\newcommand*{\colorboxed}{}
\def\colorboxed#1#{%
  \colorboxedAux{#1}%
}
\newcommand*{\colorboxedAux}[3]{%
  % #1: optional argument for color model
  % #2: color specification
  % #3: formula
  \begingroup
    \colorlet{cb@saved}{.}%
    \color#1{#2}%
    \boxed{%
      \color{cb@saved}%
      #3%
    }%
  \endgroup
}
\renewcommand{\sectionmark}[1]{\markright{#1}}
\usepackage{fancyhdr}
\pagestyle{fancy}
\lhead{\textbf{Nathaniel} \brick \textbf{\textsc{Saura}}}
\rhead{\markright}
\cfoot{\thepage}
\renewcommand{\headrulewidth}{0.4pt}

\numberwithin{equation}{section} %%%% To count the equation like Section.Number

\usepackage{accents}
\newcommand{\vect}[1]{\accentset{\Rightarrow}{#1}}

\usepackage{multicol}		% Pour utiliser \hfill et découper une partie de son texte en colonnes
\setlength{\columnseprule}{0.1pt}
\def\columnseprulecolor{\color{red}}
\setlength{\columnsep}{1.5cm}

% Numéro Roman pour le texte
\makeatletter
\newcommand*{\rom}[1]{\expandafter\@slowromancap\romannumeral #1@}
\makeatother

\begin{document}
\section*{Raw notes}
\subsubsection*{\citep{LIN2003611} on PCA}
The preprocessing could improve not only training efficiency but also prediction capability of ANN models [19,20]. Some researches [17,23] reported that the technique of \red \textbf{Principal Component Analysis (PCA) could efficiently reduce the number of input variables in a dynamic neural network by eliminating redundant input variables} \bk. The efficiency of PCA is highly depended on the characteristics of data set used as input data. In the work of Bakker et al.[16], they used the raw data of electrical capacitance tomography (ECT) measurements. By applying PCA, \red \textbf{the 66 capacitances were reduced to only three principal components} \bk that explained almost 99\% of the variance of the data. In our work, the time intervals of the original bubble passage time series were used as input data. \\
It is expected that time interval data contained less high frequency noises and were less linearly correlated. We found the variance of input data sets in this study changed greatly after the processing of PCA. Therefore, PCA is considered not to be effective in our work.\\
The preliminary study of preprocessing showed that the approach \red \textbf{which is to make the logarithmic transformation of the training data set first and then normalize them with zero mean and unity  standard deviation} \bk was the most efficient data preprocessing method of the viable methods tested.

\subsubsection*{\citep{parish2016reduced} on coarse-grained model}
The continuous interaction between the resolved and unresolved scales leads \red \textbf{to non-local effects that are challenging to understand and model} \bk. Traditional sub-grid models are based on theoretical arguments and physical observations, such as homogeneity and scale invariance, and are known to be inadequate in many problems of engineering interest.

\newpage 

\bibliographystyle{apalike}
\bibliography{bibliotheque}

\end{document}